% This is the Reed College LaTeX thesis template. Most of the work
% for the document class was done by Sam Noble (SN), as well as this
% template. Later comments etc. by Ben Salzberg (BTS). Additional
% restructuring and APA support by Jess Youngberg (JY).
% Your comments and suggestions are more than welcome; please email
% them to cus@reed.edu
%
% See https://www.reed.edu/cis/help/LaTeX/index.html for help. There are a
% great bunch of help pages there, with notes on
% getting started, bibtex, etc. Go there and read it if you're not
% already familiar with LaTeX.
%
% Any line that starts with a percent symbol is a comment.
% They won't show up in the document, and are useful for notes
% to yourself and explaining commands.
% Commenting also removes a line from the document;
% very handy for troubleshooting problems. -BTS

% As far as I know, this follows the requirements laid out in
% the 2002-2003 Senior Handbook. Ask a librarian to check the
% document before binding. -SN

%%
%% Preamble
%%
% \documentclass{<something>} must begin each LaTeX document
\documentclass[12pt,oneside,a4paper]{reedthesis}
% Packages are extensions to the basic LaTeX functions. Whatever you
% want to typeset, there is probably a package out there for it.
% Chemistry (chemtex), screenplays, you name it.
% Check out CTAN to see: https://www.ctan.org/
%%
\usepackage{graphicx,latexsym}
\usepackage{amsmath}
\usepackage{amssymb,amsthm}
\usepackage{longtable,booktabs,setspace}
\usepackage{chemarr} %% Useful for one reaction arrow, useless if you're not a chem major
\usepackage[hyphens]{url}
% Added by CII
\usepackage{hyperref}
\usepackage{lmodern}
\usepackage{float}
\floatplacement{figure}{H}
% Thanks, @Xyv
\usepackage{calc}
% End of CII addition
\usepackage{rotating}

% Next line commented out by CII
%%% \usepackage{natbib}
% Comment out the natbib line above and uncomment the following two lines to use the new
% biblatex-chicago style, for Chicago A. Also make some changes at the end where the
% bibliography is included.
%\usepackage{biblatex-chicago}
%\bibliography{thesis}
% \ifxetex
%   \usepackage{polyglossia}
%   \setmainlanguage{spanish}
%   % Tabla en lugar de cuadro
%   \gappto\captionsspanish{\renewcommand{\tablename}{Tabla}
%           \renewcommand{\listtablename}{Índice de tablas}}
% \else
%   \usepackage[spanish,es-tabla]{babel}
% \fi

% Added by CII (Thanks, Hadley!)
% Use ref for internal links
\renewcommand{\hyperref}[2][???]{\autoref{#1}}
\def\chapterautorefname{chapter}
\def\sectionautorefname{section}
\def\subsectionautorefname{subsection}
% End of CII addition

% Added by CII
\usepackage{caption}
\captionsetup{width=5in}
% End of CII addition

% \usepackage{times} % other fonts are available like times, bookman, charter, palatino

% Syntax highlighting #22

% To pass between YAML and LaTeX the dollar signs are added by CII
\title{Utilización de datos satelitales para la evaluación y mejora de los pronósticos numéricos en alta resolución a muy corto plazo}
\author{Lic. Paola Corrales}
% The month and year that you submit your FINAL draft TO THE LIBRARY (May or December)
\date{BUENOS AIRES, 2023}
\division{Facultad de Ciencias Exactas y Naturales}
\advisor{Vito Galligani}
\institution{Universidad de Buenos Aires}
\degree{Tesis presentada para optar al título de Doctora de la Universidad de Buenos Aires en el Área de Ciencias de la Atmósfera y los Océanos}
%If you have two advisors for some reason, you can use the following
% Uncommented out by CII
\altadvisor{Juan Ruiz}
\consejera{Celeste Saulo}
\place{Centro de Investigaciones del Mar y la Atmósfera. CONICET-UBA}
% End of CII addition

%%% Remember to use the correct department!
\department{Departamento de Ciencias de la Atmósfera y los Océanos}
% if you're writing a thesis in an interdisciplinary major,
% uncomment the line below and change the text as appropriate.
% check the Senior Handbook if unsure.
%\thedivisionof{The Established Interdisciplinary Committee for}
% if you want the approval page to say "Approved for the Committee",
% uncomment the next line
%\approvedforthe{Committee}

% Added by CII
%%% Copied from knitr
%% maxwidth is the original width if it's less than linewidth
%% otherwise use linewidth (to make sure the graphics do not exceed the margin)
\makeatletter
\def\maxwidth{ %
  \ifdim\Gin@nat@width>\linewidth
    \linewidth
  \else
    \Gin@nat@width
  \fi
}
\makeatother

\makeatletter
\renewcommand{\@chapapp}{Capítulo}
\makeatother

% From {rticles}

\renewcommand{\contentsname}{Índice}
% End of CII addition

\setlength{\parskip}{0pt}

% Added by CII

\providecommand{\tightlist}{%
  \setlength{\itemsep}{0pt}\setlength{\parskip}{0pt}}

\Acknowledgements{
Al Consejo Nacional de Investigaciones Científicas y Técnicas por otorgarme la beca doctoral que me permitió hacer el doctorado y este trabajo de tesis, y a la Facultad de Ciencias Exactas y Naturales, en particular al Departamento de Ciencias de la Atmósfera y los Océanos, por brindarme una educación gratuita y de calidad.

Al Centro de Investigaciones del Mar y la Atmósfera por brindarme el lugar de trabajo para realizar las tareas de investigación, proporcionando además los recursos computacionales necesarios para realizar el análisis de los experimentos realizados.

Al Dr.~Luiz Sapucci y su equipo en INPE por recibirme al comienzo del doctorado y brindarme las herramientas necesarias para trabajar con datos en formato bufr. También por mantener el contacto, la colaboración y el intercambio de ideas todos estos años.

Al Dr.~Craig Schwartz por recibirme durante mi visita al Mesoscale and Microscale Meteorology Laboratory, de NCAR y por su generosidad al compartir su conocimiento y experiencia en el uso del sitema de asimilación GSI.

A Steve Nesbitt y Craig Schwartz por bridarme, a traves de sus proyectos, la posibilidad de acceder a los recursos de la supercomputadora Cheyenene de NCAR a lo largo de todo el doctorado, sin los cuales los experimentos no hubieran sido posibles.

Al Servicio Meteorológico Nacional y en particular al equipo de Investigación y Desarrollo por siempre abrirme las puertas para colaborar en distintos proyectos e intercambiar ideas.

A los miembros del jurado, Lluis Fita Borel, Fabrício Pereira Härter y Malaquias Peña Mendez por aceptar a ser jurados y contribuir a la evaluación de este trabajo de tesis.

A mis directores, Vito y Juan por acompañarme esta montaña rusa que es el doctorado. Por su generosidad, paciencia y confianza siempre.

A la comunidad de usuaries de R en latinoamérica y el mundo, de quienes aprendí muchas de las herramientas que me permitieron hacer mi trabajo de manera abierta y reproducible. Por su generosidad al compartir su trabajo, por aprender y enseñar en comunidad y por contribuir a que el mundo, al menos de la programación, sea más diverso. Gracias también por permitirme conocer a personas increibles con las que trabajar y colaborar en distintos proyectos es siempre un placer.

A todes quienes hacen que Expedición Ciencia sea lo que es, por contagiarme el placer por hacer preguntas y mostrarme que hacer ciencia no era inalcansable.

A Pira, Santi, Bel y Jona por estar siempre cerca, acompañándonos en las buenas y en las malas donde sea que estemos todos estos años.

A mi mamá, mi papá y mis hermanos Lucas y Javi por apoyarme incondicionalmente y ayudarme a ser quien soy hoy.

Finalmente a Elio, no me alcanzan las páginas para escribir todas las razones que tengo para agradecerte.
}

\Dedication{

}

\Preface{

}

\Abstract{
\hypertarget{use-of-satellite-data-for-the-evaluation-and-improvement-of-short-term-high-resolution-numerical-forecasts}{%
\section*{Use of satellite data for the evaluation and improvement of short-term high-resolution numerical forecasts}\label{use-of-satellite-data-for-the-evaluation-and-improvement-of-short-term-high-resolution-numerical-forecasts}}
\addcontentsline{toc}{section}{Use of satellite data for the evaluation and improvement of short-term high-resolution numerical forecasts}

In Argentina, extreme weather events cause considerable human and material losses. Many of these phenomena, such as tornadoes, intense gusts, extreme rainfall in short periods of time, large hail, and ligtning, are associated with deep convection. It is, therefore, necessary to advance in the knowledge of these phenomena and in the ability to forecast their occurrence. The forecasting of severe phenomena is a very complex scientific and technological challenge due to the limited predictability at the mesoscale and the difficulty of knowing or diagnosing the state of the atmosphere at small spatial scales and short times (for example, from 1 to 10 km and in the order of minutes). Data assimilation at the mesoscale is an approach that can provide suitable initial conditions for generating high-resolution numerical forecasts and is therefore an evolving area of study.

For data assimilation methods to be successful, observational networks with a sufficient temporal and spatial resolution capable of capturing mesoscale variability must be used. The relative scarcity of conventional observations in South America poses an important challenge that can be solved with the use of other sources of observations such as automatic weather stations, winds derived from satellite observations, and radiances from polar and geostationary satellites in clear sky. In this context, this thesis work sought to quantify and compare the impact of each of the data sets in a mesoscale assimilation system.

The study of radiance assimilation at the regional level, starting from clear sky observations, is even more important in South America since no previous studies have been carried out and the conventional observation network has low spatial resolution. For this reason, special emphasis is placed on direct radiance assimilation and the quality controls necessary to work with these observations. First, the impact of the assimilation of polar satellite observations with sensors sensitive to the infrared and microwave spectrum is studied. Secondly, we study the implementation of the assimilation of observations from the GOES-16 geostationary satellite and the impact of assimilating observations at high spatial and temporal resolution in a regional context.

To achieve the objectives of this thesis, different data assimilation experiments were performed for a case study of an MCS that developed over southern South America during November 22-23, 2018 during the intensive observing period of the RELAMPAGO field campaign. The WRF-GSI-LETFK system was used for the generation of frequently update ensemble-based experiments. While the WRF model is one of the most used and in constant development, with important antecedents in Argentina, the GSI assimilation system, and in particular its LETKF version, has not been tested in Argentina and is one of the original contributions of this thesis.

The results obtained show that the assimilation of observations with high temporal and spatial frequency generates an important impact on the planetary boundary layer correcting the warm and dry bias present in the model and generating a better development of deep convection and precipitation for the case study. The assimilation of the radiance observations produced a better development of convection leading to an increase in accumulated precipitation. The ensemble forecast initialized from each experiment also showed improvements in the precipitation representation. Finally, the implementation of GOES-16 observation assimilation was shown to be possitive producing precipitation forecasts closer to the observed. In particular, the sensitivity experiments generated to analyze the impact of assimilating observations from the three water vapor channels showed that channel 10, associated with water vapor content at low levels, provides almost as much information as simultaneously assimilating the three water vapor channels, particularly when the observations are assimilated using a spatial resolution similar to that of the model (10 km).

\textbf{Key words:} Data assimilation, Mesoescale, Convencional observations, Satellite observations, Forecasts
}

	\usepackage{setspace}\onehalfspacing
\usepackage[spanish,es-tabla]{babel}
	\usepackage{booktabs}
\usepackage{longtable}
\usepackage{array}
\usepackage{multirow}
\usepackage{wrapfig}
\usepackage{float}
\usepackage{colortbl}
\usepackage{pdflscape}
\usepackage{tabu}
\usepackage{threeparttable}
\usepackage{threeparttablex}
\usepackage[normalem]{ulem}
\usepackage{makecell}
\usepackage{xcolor}
% End of CII addition
%%
%% End Preamble
%%
%
\begin{document}

% Everything below added by CII
  \maketitle

\frontmatter % this stuff will be roman-numbered
\pagestyle{empty} % this removes page numbers from the frontmatter
  \begin{resumen}
    \hypertarget{utilizaciuxf3n-de-datos-satelitales-para-la-evaluaciuxf3n-y-mejora-de-los-pronuxf3sticos-numuxe9ricos-en-alta-resoluciuxf3n-a-muy-corto-plazo}{%
    \section*{Utilización de datos satelitales para la evaluación y mejora de los pronósticos numéricos en alta resolución a muy corto plazo}\label{utilizaciuxf3n-de-datos-satelitales-para-la-evaluaciuxf3n-y-mejora-de-los-pronuxf3sticos-numuxe9ricos-en-alta-resoluciuxf3n-a-muy-corto-plazo}}
    \addcontentsline{toc}{section}{Utilización de datos satelitales para la evaluación y mejora de los pronósticos numéricos en alta resolución a muy corto plazo}
    
    En la Argentina, los fenómenos meteorológicos extremos producen cuantiosas pérdidas humanas y materiales. Muchos de estos fenómenos, por ejemplo tornados, ráfagas intensas, precipitaciones extremas en cortos períodos de tiempo, granizo de gran tamaño y actividad eléctrica, están asociados a la ocurrencia de convección profunda. Es por tal motivo necesario avanzar en el conocimiento de estos fenómenos y en la capacidad de pronosticar la ocurrencia de los mismos. El pronóstico de los fenómenos severos es un desafío científico y tecnológico muy complejo debido a la predictibilidad limitada en la mesoescala y a la dificultad de conocer o diagnosticar el estado de la atmósfera en escalas espaciales pequeñas y tiempos cortos (por ejemplo de 1 a 10 km y del orden de los minutos). La asimilación de datos en la mesoescala es un enfoque que puede proporcionar condiciones iniciales adecuadas para generar pronósticos numéricos de alta resolución y, por tanto, es un área de estudio en constante evolución.
    
    Para que los métodos de asimilación de datos tengan éxito, deben utilizarse redes de observación con suficiente resolución temporal y espacial capaces de captar la variabilidad de la mesoescala. La relativa escasez de observaciones convencionales en Sudamérica supone un importante desafío que puede ser resuelto con el uso de otras fuentes de observaciones como estaciones de superficie automáticas, vientos derivados de observaciones satelitales y radianzas de satélites polares y geoestacionarios en cielo despejado. En este contexto, este trabajo de tesis busca cuantificar y comparar el impacto de cada uno de los conjuntos de datos en un sistema de asimilación de mesoescala.
    
    El estudio de la asimilación de radianzas a nivel regional, en primera instancia para cielos despejados, cobra aún mayor importancia en Sudamérica ya que no se conocen estudios realizados previamente y la red de observaciones convencionales tiene baja resolución espacial. Por esta razón, en este trabajo se hace especial énfasis en la asimilación directa de radianzas y los controles de calidad necesarios para trabajar con estas observaciones. En primer lugar se estudia el impacto de la asimilación de observaciones de satélites polares con sensores sensibles al espectro infrarrojo y microondas. Y en segundo lugar, se estudia la implementación de la asimilación de observaciones del satélite geoestacionario GOES-16 y el impacto de asimilar observaciones en alta resolución espacial y temporal en un contexto regional.
    
    Para alcanzar los objetivos de esta tesis, se realizaron distintos experimentos de asimilación de datos aplicados a un estudio de caso de un sistema convectivo de mesoescala que se desarrolló sobre el sur de Sudamérica durante el 22 y 23 de noviembre de 2018 durante el período de observación intensiva de la campaña de campo RELAMPAGO. Se utilizó el sistema WRF-GSI-LETFK para la generación de los experimentos de actualización frecuente y basados en ensambles. Mientras que el modelo WRF es uno de los más utilizados y en constante avance, con importantes antecedentes en Argentina, el sistema de asimilación GSI y en particular su versión de LETKF, no ha sido probado en Argentina y es uno de los aportes originales de esta tesis.
    
    Los resultados obtenidos muestran que la asimilación de observaciones con alta frecuencia temporal y espacial genera un importante impacto en la capa límite planetaria corrigiendo el bias cálido y seco presente en el modelo generando un mejor desarrollo de la convección profunda y la precipitación para el caso de estudio. La asimilación de las observaciones de radianza produjo un mejor desarrollo de la convección conduciendo a un aumento de la precipitación acumulada. El pronóstico por ensambles inicializado a partir de cada experimento también mostró mejoras en la representación de la precipitación. Finalmente la implementación de la asimilación de observaciones de GOES-16 mostró ser adecuada produciendo pronósticos de precipitación más cercanos a lo observado. En particular los experimentos de sensibilidad generados para analizar el impacto de asimilar observaciones de los tres canales de vapor de agua mostraron que el canal 10, asociado al contenido de vapor de agua en niveles bajos, aporta casi tanta información como asimilar simultáneamente los 3 canales de vapor de agua, particularmente cuando las observaciones son asimiladas utilizando una resolución espacial similar a la del modelo (10 km).
    
    \textbf{Palabras claves:} Asimilación de datos, Mesoescala, Observaciones convencionales, Observaciones de satélite, Pronósticos
  \end{resumen}
  \begin{abstract}
    \hypertarget{use-of-satellite-data-for-the-evaluation-and-improvement-of-short-term-high-resolution-numerical-forecasts}{%
    \section*{Use of satellite data for the evaluation and improvement of short-term high-resolution numerical forecasts}\label{use-of-satellite-data-for-the-evaluation-and-improvement-of-short-term-high-resolution-numerical-forecasts}}
    \addcontentsline{toc}{section}{Use of satellite data for the evaluation and improvement of short-term high-resolution numerical forecasts}
    
    In Argentina, extreme weather events cause considerable human and material losses. Many of these phenomena, such as tornadoes, intense gusts, extreme rainfall in short periods of time, large hail, and ligtning, are associated with deep convection. It is, therefore, necessary to advance in the knowledge of these phenomena and in the ability to forecast their occurrence. The forecasting of severe phenomena is a very complex scientific and technological challenge due to the limited predictability at the mesoscale and the difficulty of knowing or diagnosing the state of the atmosphere at small spatial scales and short times (for example, from 1 to 10 km and in the order of minutes). Data assimilation at the mesoscale is an approach that can provide suitable initial conditions for generating high-resolution numerical forecasts and is therefore an evolving area of study.
    
    For data assimilation methods to be successful, observational networks with a sufficient temporal and spatial resolution capable of capturing mesoscale variability must be used. The relative scarcity of conventional observations in South America poses an important challenge that can be solved with the use of other sources of observations such as automatic weather stations, winds derived from satellite observations, and radiances from polar and geostationary satellites in clear sky. In this context, this thesis work sought to quantify and compare the impact of each of the data sets in a mesoscale assimilation system.
    
    The study of radiance assimilation at the regional level, starting from clear sky observations, is even more important in South America since no previous studies have been carried out and the conventional observation network has low spatial resolution. For this reason, special emphasis is placed on direct radiance assimilation and the quality controls necessary to work with these observations. First, the impact of the assimilation of polar satellite observations with sensors sensitive to the infrared and microwave spectrum is studied. Secondly, we study the implementation of the assimilation of observations from the GOES-16 geostationary satellite and the impact of assimilating observations at high spatial and temporal resolution in a regional context.
    
    To achieve the objectives of this thesis, different data assimilation experiments were performed for a case study of an MCS that developed over southern South America during November 22-23, 2018 during the intensive observing period of the RELAMPAGO field campaign. The WRF-GSI-LETFK system was used for the generation of frequently update ensemble-based experiments. While the WRF model is one of the most used and in constant development, with important antecedents in Argentina, the GSI assimilation system, and in particular its LETKF version, has not been tested in Argentina and is one of the original contributions of this thesis.
    
    The results obtained show that the assimilation of observations with high temporal and spatial frequency generates an important impact on the planetary boundary layer correcting the warm and dry bias present in the model and generating a better development of deep convection and precipitation for the case study. The assimilation of the radiance observations produced a better development of convection leading to an increase in accumulated precipitation. The ensemble forecast initialized from each experiment also showed improvements in the precipitation representation. Finally, the implementation of GOES-16 observation assimilation was shown to be possitive producing precipitation forecasts closer to the observed. In particular, the sensitivity experiments generated to analyze the impact of assimilating observations from the three water vapor channels showed that channel 10, associated with water vapor content at low levels, provides almost as much information as simultaneously assimilating the three water vapor channels, particularly when the observations are assimilated using a spatial resolution similar to that of the model (10 km).
    
    \textbf{Key words:} Data assimilation, Mesoescale, Convencional observations, Satellite observations, Forecasts
  \end{abstract}

  \begin{acknowledgements}
    Al Consejo Nacional de Investigaciones Científicas y Técnicas por otorgarme la beca doctoral que me permitió hacer el doctorado y este trabajo de tesis, y a la Facultad de Ciencias Exactas y Naturales, en particular al Departamento de Ciencias de la Atmósfera y los Océanos, por brindarme una educación gratuita y de calidad.
    
    Al Centro de Investigaciones del Mar y la Atmósfera por brindarme el lugar de trabajo para realizar las tareas de investigación, proporcionando además los recursos computacionales necesarios para realizar el análisis de los experimentos realizados.
    
    Al Dr.~Luiz Sapucci y su equipo en INPE por recibirme al comienzo del doctorado y brindarme las herramientas necesarias para trabajar con datos en formato bufr. También por mantener el contacto, la colaboración y el intercambio de ideas todos estos años.
    
    Al Dr.~Craig Schwartz por recibirme durante mi visita al Mesoscale and Microscale Meteorology Laboratory, de NCAR y por su generosidad al compartir su conocimiento y experiencia en el uso del sitema de asimilación GSI.
    
    A Steve Nesbitt y Craig Schwartz por bridarme, a traves de sus proyectos, la posibilidad de acceder a los recursos de la supercomputadora Cheyenene de NCAR a lo largo de todo el doctorado, sin los cuales los experimentos no hubieran sido posibles.
    
    Al Servicio Meteorológico Nacional y en particular al equipo de Investigación y Desarrollo por siempre abrirme las puertas para colaborar en distintos proyectos e intercambiar ideas.
    
    A los miembros del jurado, Lluis Fita Borel, Fabrício Pereira Härter y Malaquias Peña Mendez por aceptar a ser jurados y contribuir a la evaluación de este trabajo de tesis.
    
    A mis directores, Vito y Juan por acompañarme esta montaña rusa que es el doctorado. Por su generosidad, paciencia y confianza siempre.
    
    A la comunidad de usuaries de R en latinoamérica y el mundo, de quienes aprendí muchas de las herramientas que me permitieron hacer mi trabajo de manera abierta y reproducible. Por su generosidad al compartir su trabajo, por aprender y enseñar en comunidad y por contribuir a que el mundo, al menos de la programación, sea más diverso. Gracias también por permitirme conocer a personas increibles con las que trabajar y colaborar en distintos proyectos es siempre un placer.
    
    A todes quienes hacen que Expedición Ciencia sea lo que es, por contagiarme el placer por hacer preguntas y mostrarme que hacer ciencia no era inalcansable.
    
    A Pira, Santi, Bel y Jona por estar siempre cerca, acompañándonos en las buenas y en las malas donde sea que estemos todos estos años.
    
    A mi mamá, mi papá y mis hermanos Lucas y Javi por apoyarme incondicionalmente y ayudarme a ser quien soy hoy.
    
    Finalmente a Elio, no me alcanzan las páginas para escribir todas las razones que tengo para agradecerte.
  \end{acknowledgements}
  \hypersetup{linkcolor=black}
  \setcounter{secnumdepth}{5}
  \setcounter{tocdepth}{5}
  \tableofcontents

  \listoftables

  \listoffigures

\mainmatter % here the regular arabic numbering starts
\pagestyle{fancyplain} % turns page numbering back on

\hypertarget{introducciuxf3n}{%
\chapter{Introducción}\label{introducciuxf3n}}

\hypertarget{pronostico-de-eventos-severos}{%
\section{Pronostico de eventos severos}\label{pronostico-de-eventos-severos}}

La predicción de fenómenos meteorológicos extremos es de particular importancia ya que pueden producir cuantiosas pérdidas humanas y materiales. En Argentina, una gran cantidad de estos fenómenos están asociados a la ocurrencia de convección profunda entre los que se cuentan tornados, ráfagas intensas, precipitaciones extremas en cortos períodos de tiempo, granizo de gran tamaño y actividad eléctrica. Es por tal motivo necesario avanzar en el conocimiento de estos procesos y la dinámica de la convección y en la capacidad de pronosticar la ocurrencia de los mismos.

La simulación numérica de la atmósfera, es decir, la integración de las ecuaciones que rigen la evolución del sistema atmosférico es la base para el pronostico numérico del tiempo en diversas escalas temporales desde horas a semanas. Los software que llevan a cabo esta integración, conocidos como modelos numéricos para el pronóstico del tiempo han experimentado un amplio desarrollo motivado por el aumento sostenido en la capacidad de cómputo disponible y el desarrollo del conocimiento sobre procesos atmosféricos.

La naturaleza caótica de la atmósfera impone un límite para diagnosticar estados futuros conocido como límite de predictabilidad (Lorenz, 1965). Esto es así aún si los modelos numéricos fueran perfectos y pudieran representar todos los procesos atmosféricos de manera explicita ya que es imposible conocer con total exactitud las condiciones atmosféricas en un tiempo dado. Este límite de predictabilidad puede variar de acuerdo a los procesos atmosféricos representados por el modelo. Sin embargo, aún teniendo en cuenta el estado de desarrollo actual de los modelos numéricos el límite de predictabilidad es de alrededor de 2 semanas para procesos altamente predecibles (Kalnay, 2002).

El pronóstico de fenómenos extremos asociados a convección profunda es a su vez un desafío científico y tecnológico muy complejo debido a la predictabilidad limitada en la mesoescala. En particular, en la mesoescala la convección húmeda esta dentro de los fenómenos más impredecibles debido a que se asocia a inestabilidad con tasas de crecimiento de las perturbaciones muy altas particularmente ante la ausencia de forzantes como la topografía (Hohenegger and Schar, 2007). A lo anterior se suma la dificultad de conocer o diagnosticar el estado de la atmósfera en escalas espaciales pequeñas y tiempos cortos (por ejemplo de 1 a 10 km y del orden de los minutos).

Uno de los métodos que pueden utilizarse para el pronóstico de fenómenos meteorológicos severos es la utilización de modelos numéricos de la atmósfera que resuelvan explícitamente la convección profunda. Diversos estudios han comprobado que estos modelos agregan valor al pronóstico a corto plazo y que en muchos casos proveen información sobre el modo de organización de las celdas convectivas y su intensidad (Aksoy et al., 2010; Stensrud et al., 2013). Diferentes técnicas de nowcasting que se utilizan para generar pronósticos a muy corto plazo han demostrado que es posible generar pronósticos de eventos convectivos muy precisos en escalas temporales cortas y espaciales pequeñas, sin embargo estos pronósticos pierden calidad rápidamente (Sun et al., 2014). No obstante la metodología para generar pronósticos que se utilice, la capacidad de los modelos numéricos en anticipar la ubicación y tiempo de ocurrencia de eventos extremos asociados a convección es muy limitada si no se cuenta con una detallada información sobre el estado de la atmósfera en la escala de las tormentas en el momento en el que se inicializan los pronósticos numéricos (Clark et al., 2009).

\hypertarget{asimilaciuxf3n-de-datos-para-estimar-condiciones-iniciales}{%
\section{Asimilación de datos para estimar condiciones iniciales}\label{asimilaciuxf3n-de-datos-para-estimar-condiciones-iniciales}}

Ante la necesidad de contar con condiciones iniciales de calidad para inicializar pronósticos numéricos, la asimilación de datos es una metodología usada ampliamente en los principales centros de pronóstico a nivel mundial para generar la mejor estimación de las condiciones iniciales necesarias para integrar un modelo numérico. En este sentido Gustafsson et al. (2018) muestran que los pronósticos inicializados a partir de condiciones iniciales generadas usando asimilación de datos son de mejor calidad que aquellos generados escalando pronósticos globales de baja resolución.

La asimilación de datos en la mesoescala es particularmente desafiante ya que los procesos que se quieren representar tienen una componente no lineal muy importante. Debido a que el crecimiento de los errores de los modelos es rápido, los efectos no lineales aparecen más rápidamente alejándose de las hipótesis de linealidad usadas en los métodos de asimilación actuales. Por otro lado la iniciación y desarrollo de convección profunda involucra procesos en distintas escalas por lo que es un desafío muy importante incorporar observaciones que representen procesos adecuadamente.

Por lo tanto, para que los métodos de asimilación de datos tengan éxito, deben utilizarse redes de observación con suficiente resolución temporal y espacial capaces de captar la variabilidad en las escalas que se quieren resolver. En la mesoescala, y en particular a la hora de representar fenómenos de convección húmeda, será necesario contar con observaciones con una frecuencia temporal del orden de unos minutos a una hora y una escala espacial entre 1 y 10 km para incorporar información en esas escalas. En este sentido la asimilación de observaciones de superficie con alta frecuencia temporal y observaciones satelitales tienen en general un impacto positivo en los pronósticos como se discute a continuación.

Por ejemplo, Wheatley and Stensrud (2010) investigaron el impacto de la asimilación de datos de presión de superficie en un sistema de asimilación de datos basado en conjuntos de mesoescala, pero encontró un impacto limitado en dos estudios de caso relacionados con sistemas convectivos de mesoescala. Ha and Snyder (2014) demostraron que la asimilación de la temperatura y la temperatura del punto de rocío de las redes de estaciones meteorológicas de superficie de alta resolución mejoraba sistemáticamente la estructura de la capa límite planetaria simulada y mejoraba los pronósticos de precipitaciones a 12 hs sobre los Estados Unidos. Chang et al. (2017), Bae and Min (2022) y Chen et al. (2016) informaron sobre los efectos beneficiosos de la asimilación de observaciones de estaciones meteorológicas de superficie en un sistema de asimilación de datos de alta resolución utilizando las metodologías de asimilación de datos variacionales y filtro de Kalman encontrando impactos positivos en el pronóstico de la temperatura y la humedad en la capa límite planetaria y en la localización de los sistemas de precipitación. Sobash and Stensrud (2015) demostraron en un sistema de asimilación de datos de mesoescala que el impacto sobre la iniciación de la convección y el pronóstico de la precipitación de corto alcance es positivo si los datos se asimilan con frecuencia (en el orden de minutos, en lugar de en el orden de horas). Maejima et al. (2019) investigaron el impacto de la asimilación con frecuencia de 1 minuto de observaciones sintéticas en un caso de precipitación intensa, encontrando que la asimilación de observaciones de radar de alta frecuencia y espacialmente densas conducen a una mejor representación de la circulación de mesoescala aunque el número de observaciones proporcionadas por las estaciones de superficie es mucho menor que el proporcionado por los radares meteorológicos. Gasperoni et al. (2018) realizaron un estudio de caso para evaluar el impacto de la asimilación de las observaciones producidas por estaciones meteorológicas privadas que no se incorporan a los análisis operativos globales. Encontraron un efecto positivo al asimilar estas observaciones sobre el inicio de la convección húmeda profunda a lo largo de una línea seca.

Se ha investigado el impacto de otros tipos de observaciones de resolución espacial y temporal relativamente alta, como observaciones de satélites, en el contexto de la asimilación de datos de mesoescala. Estas observaciones incluyen radianzas y productos derivados como perfiles de temperatura y humedad y vientos derivados de satélite, Wu et al. (2014), Cherubini et al. (2006) y Sawada et al. (2019) observaron un impacto positivo de la asimilación de viento derivado de información satelital de alta frecuencia en un estudio de caso de un ciclón tropical utilizando un sistema de asimilación de datos basado en el filtro de Kalman. Por otro lado, Gao et al. (2015) observaron un impacto positivo en pronósticos a corto plazo gracias a la asimilación de viento estimado a partir de las observaciones de satélites geoestacionarios sobre Estados Unidos. Si bien la asimilación de radianzas de manera directa es rutinario a nivel global en centros de pronóstico como el \emph{National Centers for Environmental Prediction} (NCEP) o el \emph{European Centre for Medium-Range Weather Forecasts} (ECMWF), su asimilación en escalas regionales es un problema complejo y trae desafíos tecnológicos y científicos que se describirán en la sección \ref{asim-rad}.

\hypertarget{asimilaciuxf3n-de-datos-en-sudamuxe9rica}{%
\section{Asimilación de datos en Sudamérica}\label{asimilaciuxf3n-de-datos-en-sudamuxe9rica}}

La historia de la asimilación de datos en Sudamérica y en particular en Argentina es relativamente corta. A principios de la década del 90 Vera (1992) en su tesis doctoral desarrolló un Sistema de Asimilación de Datos Intermitente que utilizaba la interpolación optima en un modelo cuasigeostrófico en la región sur de Sudamérica. Algunos años después, en 1997, el Servicio Meteorológico Nacional (SMN) se implementó un
análisis utilizando el método de Correcciones Sucesivas en el cual el campo preliminar
es corregido en sucesivas iteraciones por las observaciones, en un modelo de 10 niveles verticales (García Skabar, 1997).

Por otro lado el Centro de Pronóstico del Tiempo y Estudios Climáticos (CPTEC) de Brasil desarrolló un sistema de asimilación de datos global que utiliza el sistema Gridpoint Statistical Interpolation (GSI) en conjunto con su modelo global BAM (por nombre en inglés, Brazillian global Atmospheric Model) y posteriormente aplicaciones regionales utilizando el modelo \emph{Weather Research \& Forecasting Model} (WRF, Skamarock et al., 2008) en conjunto con el sistema de asimilación GSI. En particular, Goncalves de Goncalves et al. (2015) mostró experimentos realizados en el CPTEC usando el sistema de asimilación de datos regional para simulaciones de 12, 10 y 3 kilómetros durante un mes. Ferreira et al. (2017), Bauce Machado et al. (2017), Toshio Inouye et al. (2017) y Ferreira et al. (2020) también mostraron resultados positivos al asimilar observaciones de radar, y observaciones convencionales de superficie y altura en aplicaciones regionales sobre Brasil con resoluciones de entre 1 y 10 km.

En los últimos años, se documentaron importantes avances asociados a la asimilación de datos en Argentina. Por ejemplo Saucedo (2015) realizó un estudio teórico de asimilación de datos utilizando la metodología \emph{Local Ensemble Transform Kalman Filter} (LETKF, Hunt et al., 2007) acoplado al modelo WRF donde estudió técnicas para la representación de diferentes fuentes de incertidumbre incluyendo los errores en las condiciones de borde y los errores de modelo. Posteriormente Dillon (2017) avanzó en su tesis de doctorado en el desarrollo de un sistema de asimilación de datos reales y concluyó que la asimilación de perfiles de temperatura y humedad estimados con observaciones satelitales en la asimilación tienen un impacto positivo en los análisis y pronósticos. Por otro lado Maldonado et al. (2021) avanzó en la asimilación frecuente de observaciones de radar para casos de convección húmeda profunda observando impactos positivos tanto en el análisis como en los pronósticos a muy corto plazo generados usando estas condiciones iniciales mejoradas. Más recientemente, el SMN en conjunto con el Centro de Investigaciones del Mar y la Atmósfera (CIMA) desarrollaron y probaron el sistema de asimilación de actualización rápida LETKF-WRF de manera operativa experimental durante la campaña de campo \emph{Remote sensing of Electrification, Lightning, And Mesoscale/microscale Processes with Adaptive Ground Observations} (RELAMPAGO) (Nesbitt et al., 2021). El sistema incorporó, con una frecuencia horaria, observaciones convencionales de estaciones meteorológicas automáticas de redes privadas, productos derivados de satélites multiespectrales y viento derivado de observaciones satelitales, y observaciones de radar y generó pronósticos a 36 hs inicializados cada 3 hs. Dillon et al. (2021) mostraron que el pronóstico inicializado a partir de los análisis muestra un rendimiento general similar al de los pronósticos inicializados a partir del sistema GFS, e incluso un impacto positivo en algunos casos. Este resultado es especialmente importante para regiones con pocos datos, como el sur de Sudamérica, donde las redes operativas no son lo suficientemente densas como para captar los detalles de la mesoescala. Actualmente el SMN está desarrollando y evaluando un sistema de asimilación similar al implementado en Dillon et al. (2021) para utilizarlo en la generación de pronósticos de manera operativa.

En general, los trabajos mencionados previamente encuentran que la incorporación de nuevas fuentes de observación producen impactos positivos en la calidad de los análisis de mesoescala, sin embargo la asimilación directa de radianzas es una de las fuentes de observación menos estudiadas a nivel regional en Sudamérica. En este contexto, la asimilación directa de radianzas cobra particular importancia, ya que estas observaciones con alta resolución espacial y frecuencia temporal podrían aportar información sobre los procesos asociados a convección húmeda profunda en Sudamérica donde la red de observaciones convencionales no es muy extensa. Sin embargo y a pesar de la extensa historia de la asimilación de estas observaciones a escala global (que se describe a continuación) y otras regiones, no se conocen antecedentes de asimilación directa de radianzas en la mesoescala en esta región

\hypertarget{asim-rad}{%
\section{Asimilación de radianzas de satélites}\label{asim-rad}}

Los primeros satélites en proveer información meteorológica fueron desarrollados en las décadas de los 60 y 70. Estos estaban ubicados en órbitas polares, es decir, con cierta inclinación respecto del Ecuador, pasando cerca o sobre los polos. Incluían sensores infrarrojos y de microondas para monitorear la temperatura y humedad. Hacia finales de la década de los 70, Estados Unidos, Europa y Japón ya habían lazando los primeros satélites geoestacionarios con sensores en el rango visible e infrarrojo. Pocos años después este tipo de observaciones se incorporaban al Sistema de Observación Global (\emph{Global Observing System} en inglés).

El primer conjunto de satélites compuesto por los sensores \emph{High-resolution Infrared Radiation Sounder} (HIRS), \emph{Microwave Sounding Units} (MSU) y \emph{Stratospheric Sounder Unit} (SSU) o sistema TOVS (por su nombre en inglés, \emph{TIROS Operational Vertical Sounder}) podían cubrir el globo completo cada 12 hs. Si bien cada uno de estos sensores generaba información complementaria en la tropósfera y baja estratósfera, la resolución horizontal y vertical era limitada. En particular el primer HIRS, un sensor infrarrojo, tenía una resolución horizontal de 40 km, mientras que actualmente HIRS4 tiene una resolución horizontal de 10 km. El sensor MSU, sensible en las microondas, tenía una resolución de 110 km mientras que el sensor que lo reemplazó, \emph{Advanced Microwave Sounding Unit-A} (AMSU-A) cuenta con una resolución de 50 km. En la vertical, el espesor de la atmósfera donde cada canal es más sensible ronda entre los 5 y 10 km y aún en los casos donde la función de pesos de los canales se solapan, la resolución apenas alcanza los 3 km (Eyre et al., 2020).

Las primeras pruebas de asimilación de observaciones satelitales utilizaron datos derivados de satélites, es decir perfiles de temperatura y humedad, y fueron desarrolladas principalmente en Australia, motivadas particularmente por la escasez de observaciones en el hemisferio sur. Kelly et al. (1978) mostraron una importante mejora en los pronósticos a 24 horas de altura geopotencial entre 1000 y 200 hPa cuando se asimilaban de manera continua perfiles de temperatura derivados del satélite polar Nimbus-6 con sensores en el rango de microondas e infrarrojo. A nivel global Ohring (1979) resume los avances de la década indicando que los impactos son positivos aunque pequeños y que la mayor mejora se observa en los pronósticos en el hemisferio sur. Al mismo tiempo Ohring (1979) señala algunos de los posibles problemas asociados, por ejemplo la baja resolución vertical de los perfiles de temperatura y humedad y problemas en la generación de los mismos.

A principios de los 80 los centros de pronóstico mundiales continuaron estudiando la posibilidad de asimilar perfiles de temperatura y humedad estimados a partir de observaciones satelitales encontrando una disminución en el error de pronósticos a 6 horas principalmente en regiones donde hay poca disponibilidad de otras observaciones (Eyre et al., 2020). En particular el E\emph{CMWF Seminar on Data Assimilation Systems and Observing System Experiments} concluye en 1984 que la asimilación de estas observaciones cumple un rol importante en el análisis de sistemas meteorológicos de gran escala en latitudes medias y altas, y en particular en el Hemisferio Sur. Sin embargo, hacia finales de los 80, los modelos de pronóstico habían mejorado sustancialmente haciendo que el potencial impacto de observaciones erróneas u observaciones asimiladas de manera incorrecta degradaran sustancialmente el pronóstico particularmente en el Hemisferio Norte. Andersson et al. (1991) mostró que los incrementos en el análisis presentaba patrones con importante sesgo cuando se asimilaba perfiles de temperatura y humedad (o \emph{retrievals} en inglés) de TOVS.

Eyre et al. (2020) explica que la principal razón por la que los resultados obtenidos no fueran positivos era que se trataba a los retrievals como ``mediciones in-situ de baja calidad'' sin tener en cuenta las características particulares de estas observaciones de satélite como la resolución horizontal y vertical.

En la década de los 90, luego de que los centros de asimilación comenzaran a utilizar técnicas avanzadas de asimilación de datos como 3D-Var, se dieron las condiciones necesarias para asimilar radianzas observadas de manera directa. Esto implica transformar las variables del modelo en las variables observadas para poder hacer la comparación de manera directa. Es por esto que fue clave en la asimilación de estas observaciones el desarrollo de modelos de transferencia radiativa que pudieran transformar el campo preliminar en radianzas comparables con las observaciones en tiempos razonables para ser usados de manera operativa. Sin embargo, la correcta asimilación de estas observaciones depende de 3 factores: que las observaciones no tengan sesgo o \emph{bias}, que sus errores tengan una distribución Gaussiana y que el problema no sea afectado fuertemente por procesos no lineales (Eyre et al., 2022).

Una parte importante del desarrollo de la asimilación de datos en los últimos 20 años tiene que ver con el avance de los modelos de transferencia radiativa en relación a la emisividad del suelo y la caracterización de las propiedades dispersivas de la precipitación y las nubes, tanto en el infrarrojo como en el microondas. Inicialmente solo se asimilaron observaciones sobre agua y en condiciones de cielos despejados.

A su vez, Bauer et al. (2010) indican que más de un 75\% de las observaciones satelitales eran descartadas en los modelos globales cuando no se asimilaban observaciones contaminadas por nubes. Las observaciones nubosas presentan numerosas complicaciones desde el punto de vista de las parametrizaciones de microfísica en los modelos pero también a la hora de resolver la dispersión de manera realista en los modelos de transferencia radiativa. Estas observaciones introducen no linealidades y los errores observacionales son difíciles de estimar. Fue necesario entonces, el desarrollo de técnicas de detección de nubes que permitan filtrar las regiones afectadas por nubosidad. A su vez, mejoras en los modelos de transferencia radiativa respecto del tratamiento de los distintos tipos de superficie y la representación y tratamiento de las nubes permiten en la actualidad incorporar observaciones que que anteriormente no podían asimilarse. Finalmente, el desarrollo de métodos de corrección del bias de radianzas aplicados directamente en el proceso de asimilación fue determinante para la asimilación directa de este tipo de observaciones (Eyre et al., 2022).

Junto al desarrollo de la asimilación de radianzas, también continuó el desarrollo de nuevos sensores, como la serie AMSU-A y AMSU-B y el sistema ATOVS (\emph{Advance TOVS)} que cuenta con mayor cantidad de canales y una mayor resolución horizontal y vertical. Posteriormente el desarrollo de los sensores multiespectrales como IASI y AIRS permitieron obtener información con mayor resolución vertical al contar con más de 3000 canales en la región infrarroja del espectro electromagnético.

Mientras que la asimilación directa de radianzas en modelos globales está establecida y estudiada (Eyre et al., 2022), las aplicaciones en modelos regionales, sin embargo, siguen siendo un desafío por la escasa cobertura de las observaciones de sensores a bordo de satélites en órbitas polares, la corrección del sesgo y el tope de la atmósfera bajo usado en modelos regionales. Bao et al. (2015) estudiaron el impacto de la asimilación de datos de radianzas de microondas e infrarrojo en el pronóstico de temperatura y humedad en el oeste de EE.UU. y encontró una reducción del sesgo de la temperatura en niveles bajos y medios como resultado de las observaciones de microondas, pero un efecto opuesto cuando se asimilaban radianzas en el infrarrojo. Más recientemente, Zhu et al. (2019) estudiaron el impacto de la asimilación frecuente de radianzas de satélites para un sistema regional y mostró una mejora para todas las variables, en particular para la humedad relativa en los niveles superiores. Wang and Randriamampianina (2021) estudiaron el impacto de la asimilación de observaciones en el infrarrojo en el Reanálisis Regional Europeo Copernicus de alta resolución e informaron que las observaciones de radianza de satélite tuvieron un impacto neutro en los análisis de la altura geopotencial en la tropósfera baja, mientras que el impacto fue ligeramente negativo para la tropósfera superior y estratosfera. También observaron resultados similares para pronósticos a 3 hs inicializados a partir del análisis, pero un impacto positivo en las previsiones de mediano plazo (12 y 24 hs).

La asimilación directa de radianzas de satélites polares en aplicaciones regionales es también un área de estudio activa que cobra aún más importancia con el desarrollo y puesta en órbita de satélites geoestacionarios de última generación como GOES-16. Usando radianzas simuladas del canal 9 del sensor ABI (asociado al vapor de agua en niveles medios) Jones et al. (2013) mostraron una mejora en la representación de la humedad en niveles medios y altos de la tropósfora, resultados que mejoraban al incluir también observaciones de radar. Lee et al. (2019) mostraron que con los controles de calidad apropiados, la asimilación de observaciones de los canales asociados al vapor de agua de GOES-16 mejora los pronósticos de dos huracanes durante 2017. Por otro lado Jones et al. (2014) estudiaron el impacto de asimilar radianzas tanto de cielos despejados como nubosos y encontraron que mejoran los pronósticos de iniciación de la convección en muchos casos.

\hypertarget{hipuxf3tesis-y-objetivos}{%
\section{Hipótesis y objetivos}\label{hipuxf3tesis-y-objetivos}}

En base a los importantes avances en la asimilación de datos en general y en las aplicaciones regionales en Argentina y Sudamérica, el objetivo principal de este trabajo es contribuir a la cuantificación y comparación del impacto de diversas fuentes de datos que aportan información en escalas espacio-temporales dentro de la mesoescala alfa-beta (Orlanski, 1975). La hipótesis principal de este trabajo, es entonces, que observaciones con alta resolución espacial y/o frecuencia temporal, que no son actualmente asimilados a escala regional en Sudamérica, contribuirán a mejorar la representación del entorno previo al desarrollo de eventos de convección húmeda profunda y por lo tanto a su pronóstico en el rango de horas a 1 día.

Este trabajo estudia el impacto de la asimilación de observaciones tanto en el análisis como en pronósticos por ensambles a corto plazo en el contexto de los eventos de sistemas convectivos de mesoescala (SCM) debido a la importancia de este tipo de eventos en la región centro y norte de Argentina. Las observaciones utilizadas provienen de estaciones meteorológicas de alta resolución, observaciones de viento derivadas de satélite y radianzas satelitales polares y geoestacionarios en cielo despejado. En particular uno de los objetivos será evaluar el aporte de cada una de las fuentes de datos en una región donde la red de observación convencional es bastante escasa y donde las contribuciones potenciales de sistemas de observación como redes de estaciones automáticas y observaciones de satélite son mayores.

El estudio de la asimilación de radianzas a nivel regional cobra aún mayor importancia en Sudamérica ya que no se conocen estudios realizados previamente. Por esta razón, el segundo objetivo de este trabajo será el estudio de la asimilación de radianzas y los controles de calidad necesarios para trabajar con estas observaciones. En primer lugar se estudiará el impacto de la asimilación de observaciones de satélites polares con sensores sensibles al espectro infrarrojo y microondas. Y en segundo lugar, se estudiará la implementación de la asimilación de observaciones del satélite geoestacionario GOES-16 y el impacto de asimilar observaciones en alta resolución espacial y temporal en un contexto regional. Finalmente, como tercer objetivo, se buscará evaluar el impacto de asimilar observaciones GOES-16, disponibles con una gran resolución espacial y frecuencia temporal, en comparación con radianzas de satélites polares.

Para alcanzar los objetivos de este trabajo, se realizaron distintos experimentos de asimilación de datos de actualización frecuente y basado en ensambles aplicados a un estudio de caso de un SCM que se desarrolló sobre el sur de Sudamérica durante el 22 y 23 de noviembre de 2018 durante el período de observación intensiva de la campaña de campo RELAMPAGO. Se utilizará el sistema WRF-GSI-LETFK para la generación de los experimentos de asimilación. Mientras que el modelo WRF es uno de los más utilizados y en constante avance, con importantes antecedentes en Argentina (por ejemplo Dillon et al., 2021), el sistema de asimilación \emph{Gridpoint Statistical Interpolation system} (GSI, Shao et al., 2016) y en particular su versión de LETKF, no ha sido probado en Argentina y es otro de los aportes originales de esta tesis junto con la implementación de la asimilación de observaciones de GOES-16.

Finalmente, este trabajo de tesis ha dado lugar a una publicación en una revista con referato y una serie de presentaciones en congresos nacionales e internacionales:

\emph{Capítulo 3}
\begin{itemize}
\tightlist
\item
  Corrales P., Galligani V., Ruiz J., Sapucci L., Dillon M.E., Garcia Skabar Y., Sacco M., Schwartz C., and Nesbitt Stephen, 2022: Hourly Assimilation of Different Sources of Observations Including Satellite Radiances in a Mesoscale Convective System Case During RELAMPAGO campaign, Atmospheric Research, vol.~281, 106456, doi: \href{doi:10.1016/j.atmosres.2022.106456}{10.1016/j.atmosres.2022.106456}.
\item
  Corrales P., Galligani, V., Ruiz, J., Dillon, M. E., García Skabar, Y., Sacco, M. Asimilación horaria de observaciones convencionales y de satélite para un caso de estudio durante RELAMPAGO. Congremet XIV. 7-11 Noviembre, 2022. Buenos Aires, Argentina. Resumen y poster.
\item
  Corrales P., J. Ruiz, V.S. Galligani, M. Sacco, M. E. Dillon, Y. Garcia Skabar, L. Sapucci, and S. Nesbitt. Assimilation of conventional observations in a deep convection case during RELAMPAGO using the WRF-GSI-LETFK system. RELAMPAGO-CACTI Data Analysis Workshop. 19-22 November, 2019. Buenos Aires, Argentina. Resumen y poster.
\end{itemize}
\emph{Capítulo 4}
\begin{itemize}
\tightlist
\item
  Corrales P., Galligani V.S., Ruiz J., Sapucci L., Dillon M.E., Garcia Skabar Y., Sacco M., and Nesbitt Stephen. Assimilation of conventional and satellite observations in a deep convection case during RELAMPAGO using the WRF-GSI-LETFK system. The International EnKF Workshop 2021 (online). June, 2021. Resumen y poster.
\item
  Corrales P., J. Ruiz, V.S. Galligani. Forecast Evaluation of a Deep Convection Case During Relampago Assimilating Conventional and Satellite Observations with the WRF-GSI-LETKF System. WCRP-WWRP Symposium on Data Assimilation and Reanalysis (online). September, 2021. Resumen y poster.
\item
  Corrales P., Ruiz J., Galligani V. y otros. ``Assimilation of conventional and satellite observations in a deep convection case during RELAMPAGO using the WRF-GSI-LETFK system'' NORCE (2021). Resumen y presentación.
\end{itemize}
\emph{Capítulo 5}
\begin{itemize}
\tightlist
\item
  Corrales P., Galligani, V., Ruiz, J. Asimilación de observaciones de GOES-16 en cielo claro para aplicaciones de mesoescala utilizando el sistema GSI-LETKF-WRF. Congremet XIV. 7-11 Noviembre, 2022. Buenos Aires, Argentina. Resumen y presentación.
\item
  Corrales P., Schwartz, C., Ruiz, J. and Galligani, V. Assimilation of polar and geostationary satellite observations during RELAMPAGO using a WRF‐GSI‐LETKF system. 8th International Symposium on Data Assimilation (ISDA). Colorado, USA. June 2022. Resumen y presentación.
\item
  Corrales P., J. Ruiz, V.S. Galligani. Forecast Evaluation of a Deep Convection Case During Relampago Assimilating Conventional and Satellite Observations with the WRF-GSI-LETKF System. International TOVS Study Conferences (ITSC) XXIII (online). June, 2021. Resumen y poster.
\end{itemize}
\hypertarget{metodologuxeda-y-datos}{%
\chapter{Metodología y datos}\label{metodologuxeda-y-datos}}

\hypertarget{caso-de-estudio}{%
\section{Caso de estudio}\label{caso-de-estudio}}

El caso de estudio utilizado en este trabajo corresponde a un SCM que se inicio y alcanzo su madurez durante el 22 de noviembre de 2018 en el centro y norte de Argentina. Previo al desarrollo de este SCM, el centro y norte de Argentina se encontraba inmerso en una masa de aire cálido y húmedo con altos valores de energía potencial disponible convectiva (CAPE), como lo muestra ERA 5 (Hersbach et al., 2018) en la Figura \ref{fig:caso}a. En las Figuras \ref{fig:caso}a-c se puede apreciar una zona baroclínica en el campo de espesores 1000-500 hPa (contornos rojos discontinuos) y una vaguada en el campo de presión en superficie (contornos negros) que da cuenta de un frente frío que cruzó el centro de Argentina el 22 de noviembre de 2018. Este frente frío desencadenó el desarrollo de células convectivas aisladas que rápidamente crecieron hasta convertirse en un SCM excepcionalmente grande que puede observarse en las imágenes de temperatura de brillo del canal 13 de GOES-16 (Figura \ref{fig:caso}d,e). Durante ese día, varias estaciones de superficie observaron actividad eléctrica, fuertes ráfagas de viento y lluvias intensas. Al norte de la región, el entorno cálido y húmedo contribuyó al desarrollo de convección aislada que finalmente creció y se fusionó con el SCM (Figura \ref{fig:caso}f). El SCM recorrió aproximadamente 2500 km de sur a norte, disipándose sobre Paraguay y el sur de Brasil después de 42 horas desde el inicio de su desarrollo. Este caso de estudio es de particular interés por desarrollarse durante la campaña RELAMPAGO donde se generaron observaciones extraordinarias que son utilizadas en la verificación de los experimentos desarrollados.


\begin{figure}

{\centering \includegraphics{thesis_files/figure-latex/caso-1} 

}

\caption{Presión a nivel del mar (hPa, contornos negros), espesor 1000-500 hPa (contornos rojos discontinuos) y energía potencial convectiva disponible (sombreada) según ERA5 y temperatura de brillo del canal 13 de GOES-16 para a,d) 00 y b,e) 12 UTC, 22 de Noviembre y c,f) 00 UTC, 23 de Noviembre.}\label{fig:caso}
\end{figure}
\hypertarget{observaciones}{%
\section{Observaciones}\label{observaciones}}

Durante el periodo de estudio se realizaron diferentes experimentos para evaluar el impacto al asimilar diferentes fuentes de observaciones, entre los que se incluyen datos convencionales oficiales de superficie y altura, redes no oficiales de estaciones meteorológicas automáticas de superficie, vientos derivados de satélite, y radianzas en cielo despejado. También se utilizaron otras observaciones como radiosondeos para cuantificar la calidad de los análisis y pronósticos.

\hypertarget{conjuntos-de-observaciones-asimiladas}{%
\subsection{Conjuntos de observaciones asimiladas}\label{conjuntos-de-observaciones-asimiladas}}

\hypertarget{convencionales}{%
\subsubsection{Convencionales}\label{convencionales}}

Las observaciones convencionales (OMC) utilizadas forman parte del flujo de datos del Sistema Global de Asimilación de Datos (GDAS). Se asimilarán las observaciones convencionales incluidas en los archivos \emph{Binary Universal Form for Representation of Meteorological Data} (PREPBUFR) generados por NCEP. Los archivos PREPBUFR incluyen observaciones de superficie procedentes de 117 estaciones meteorológicas de superficie (EMC), barcos, y observaciones de altura procedentes de 13 sitios de lanzamiento de radiosondeos y aviones dentro del área en estudio. Los triángulos naranjas de la Figura \ref{fig:dominio}a indican la ubicación de las estaciones de superficie incluidas en este experimento. La frecuencia de estas observaciones varia entre 1 hora para las estaciones de superficie y 12/24 horas para los radiosondeos. Las observaciones del viento en superficie sobre los océanos (ASCATW) proceden de los dispersómetros y también se incluyen en los archivos PREPBUFR.

La tabla \ref{tab:tabla-obs} enumera todos los tipos de observación (presión en superficie, temperatura, humedad específica y viento) disponibles para cada fuente de observación, junto con el error asociado para cada una definidos de acuerdo a la configuración por defecto del sistema de asimilación usado en este trabajo. En algunos casos, el error varía con la altura y depende de la plataforma específica (avión y viento derivado del satélite). En cuanto al control de calidad aplicado a las observaciones convencionales, el sistema de asimilación realiza una primera comparación entre la innovación (la diferencia entre la observación y la observación simulada por el modelo para el campo preliminar) y un umbral predefinido que depende del error de observación (también incluido en la Tabla \ref{tab:tabla-obs}).


\begin{figure}

{\centering \includegraphics[width=0.8\linewidth,]{thesis_files/figure-latex/dominio-1} 

}

\caption{a) Dominio utilizado para las simulaciones (recuadro negro), dominio interior utilizado para la comparación entre experimentos (recuadro rojo), la región mostrada en b) (recuadro azul claro), y la ubicación de las Estaciones Meteorológicas Automáticas (EMA, cuadrados verdes) y las Estaciones Meteorológicas Convencionales (EMC, triángulos naranjas). b) Ubicación de los lanzamientos de radiosondeos durante RELAMPAGO. Los puntos verdes corresponden a los radiosondeos lanzados durante el IOP 7, los triángulos naranjas son radiosondeos lanzados durante el IOP 8, y los cuadrados morados son radiosondeos lanzados fuera de los periodos de medición intensiva. También se muestra la topografía en metros (sombreada).}\label{fig:dominio}
\end{figure}
\hypertarget{ema}{%
\subsubsection{Red de Estaciones Meteorológicas Automáticas}\label{ema}}

También se asimilaron observaciones de 866 Estaciones Meteorológicas Automáticas (EMA) que forman parte de 17 redes de superficie públicas y privadas en la región. El conjunto de datos utilizado en este estudio fue obtenido del repositorio de datos de RELAMPAGO (Garcia et al., 2019). Estas estaciones se indican como cuadrados verdes en la Figura \ref{fig:dominio}a. Tienen mayor cobertura espacial que las EMC y una frecuencia de muestreo de 10 minutos en la mayoría de los casos. Todas las estaciones miden temperatura, pero sólo 395 estaciones proporcionan humedad, 422 la presión y 605 el viento.
Los errores de observación utilizados para asimilar estas observaciones son los mismos que para las observaciones de EMC (véase la tabla \ref{tab:tabla-obs}).

\hypertarget{vientos-estimados-por-satuxe9lite}{%
\subsubsection{Vientos estimados por satélite}\label{vientos-estimados-por-satuxe9lite}}

Las observaciones de viento derivadas de los satélites también se incluyen en los archivos PREPBUFR disponibles cada 6 hs, y consisten en estimaciones de GOES-16 (utilizando los canales visible, infrarrojo y de vapor de agua) y METEOSAT 8 y 11 (utilizando los canales visible y de vapor de agua). Estos satélites son geoestacionarios y tienen una frecuencia de escaneo de 15 minutos durante el evento analizado, por lo que, ante la presencia de nubosidad para estimar el viento, generarán estimaciones que estarán disponibles en cada ciclo de asimilación (ver Capítulo \ref{cap-3-analisis}, Figura \ref{fig:cycle}). Las observaciones de viento derivadas de los satélites METEOSAT tienen una resolución de 3 km mientras que las de GOES-16 tienen una resolución de 10 km. Debido al dominio elegido y la cobertura de estos satélites, GOES-16 es la principal fuente de observaciones de vientos derivados de los satélites (99 \% de las observaciones). Los errores de observación utilizados para asimilar estas observaciones siguen la configuración por defecto del GSI y se indican en la Tabla \ref{tab:tabla-obs} bajo el nombre SATWND.
\begin{table}

\caption{\label{tab:tabla-obs}Características de las observaciones asimiladas: El código de cada tipo de observación y su fuente, las variables disponibles, el error de observación y los umbrales de control de calidad utilizados.}
\centering
\fontsize{9}{11}\selectfont
\begin{tabular}[t]{>{\raggedright\arraybackslash}p{4.5em}>{\raggedright\arraybackslash}p{5.5em}>{\raggedright\arraybackslash}p{6em}>{\raggedright\arraybackslash}p{8em}>{\raggedright\arraybackslash}p{8em}}
\toprule
Código & Plataforma & Variable & Error & Umbral de error\\
\midrule
 &  & Presión & 1-1.6 $hPa^*$ & 3.6 $hPa$\\

 &  & Temperatura & 1.5 $K$ & 7 $K$\\

 &  & Humedad específica & 20 \% & 8 $gkg^{-1}$\\

\multirow{-4}{4.5em}{\raggedright\arraybackslash EMC   EMA} & \multirow{-4}{5.5em}{\raggedright\arraybackslash Estaciones meteorológicas de superficie} & Viento & 2.2 $ms^{-1}$ & 6 $ms^{-1}$\\
\cmidrule{1-5}
 &  & Presión & 1.1-1.2 $hPa^{**}$ & 4 $hPa$\\

 &  & Temperatura & 0.8-1.5 $K^*$ & 8 $K$\\

 &  & Humedad específica & 20 \% & 8 $gkg^{-1}$\\

\multirow{-4}{4.5em}{\raggedright\arraybackslash ADPUPA} & \multirow{-4}{5.5em}{\raggedright\arraybackslash Radiosondeos} & Viento & 1.4-3 $ms^{-1}$* & 8 $ms^{-1}$\\
\cmidrule{1-5}
 &  & Temperatura & 1.47-2.5 $K^+$ & 7 $K$\\

\multirow{-2}{4.5em}{\raggedright\arraybackslash AIRCFT} & \multirow{-2}{5.5em}{\raggedright\arraybackslash Aviones} & Viento & 2.4-3.6 $ms^{-1+}$ & 6.5-7.5 $ms^{-1+}$\\
\cmidrule{1-5}
ASCATW & Dispersómetros & Viento & 1.5 $ms^{-1}$ & 5 $ms^{-1}$\\
\cmidrule{1-5}
 &  & Presión & 1.3 $hPa$ & 4 $hPa$\\

 &  & Temperatura & 2.5 $K$ & 7 $K$\\

 &  & Humedad específica & 20 \% & 8 $gkg^{-1}$\\

\multirow{-4}{4.5em}{\raggedright\arraybackslash SFCSHP} & \multirow{-4}{5.5em}{\raggedright\arraybackslash Barcos y boyas} & Viento & 2.5 $ms^{-1}$ & 5 $ms^{-1}$\\
\cmidrule{1-5}
SATWND & Viento derivado de satélites & Viento & 3.8-8 $ms^{-1*+}$ & 1.3-2.5 $ms^{-1+}$\\
\bottomrule
\multicolumn{5}{l}{\rule{0pt}{1em}\textsuperscript{*} El error de la observación varía con la altura.}\\
\multicolumn{5}{l}{\rule{0pt}{1em}\textsuperscript{**} Observationes por encima de 600 hPa son rechazadas.}\\
\multicolumn{5}{l}{\rule{0pt}{1em}\textsuperscript{+} El error de la observación depende del tipo de reporte.}\\
\end{tabular}
\end{table}
\hypertarget{radianzas-de-satuxe9lite}{%
\subsubsection{Radianzas de satélite}\label{radianzas-de-satuxe9lite}}

\hypertarget{satuxe9lites-polares}{%
\paragraph{Satélites polares}\label{satuxe9lites-polares}}

En este trabajo se utilizaron radianzas de satélites disponibles a través del flujo de datos del GDAS, que incluye observaciones en el espectro infrarrojo y microondas. Dado que el dominio regional que se utilizará en este trabajo se encuentra en latitudes medias y que la mayoría de los satélites de interés están en órbitas polares, cada sensor escanea la zona sólo dos veces al día con una cobertura espacial que depende de la franja de cobertura del satélite. Por esta razón, el número de observaciones de los satélites polares varía significativamente en cada ciclo de asimilación. En particular, los sensores multiespectrales proporcionaron entre 10.000 y 100.000 observaciones en el dominio de interés por cada escaneo cada 12 horas, contribuyendo al 88 \% de la cantidad total de radianzas asimiladas (Tabla \ref{tab:table-rad}) en los experimentos descriptos en la Sección \ref{config}.

Los sensores microondas disponibles para ser asimilados en el periodo de interés son el \emph{Advanced Microwave Sounding Unit - A} (AMSU-A) a bordo de las plataformas NOAA-15, NOAA-18 y METOP-A, y el \emph{Microwave Humidity Sounder} (MHS) a bordo de las plataformas NOAA-19, METOP-A y METOP-B.

AMSU-A (Robel and Graumann, 2014) es un radiómetro de microondas que cuenta con 15 canales de observación diferentes entre 23.8 - 89 GHz y ha demostrado ser un importante sensor para el sondeo de la temperatura atmosférica (English et al., 2000). Cada canal detecta la radiación de microondas procedente de distintas capas de la atmósfera terrestre, cuya localización se describe mediante la función de peso, que corresponde a la derivada de la transmitancia con respecto a la altura. Estas funciones de peso son necesarias para definir la ubicación de las observaciones y por lo tanto en que espesor de la atmósfera generaran un impacto al momento de la asimilación. El AMSU-A es conocido como un ``sensor de temperatura'' ya que sus canales fueron seleccionados para detectar la radiación emitida por capas sucesivas de la atmósfera. La Figura \ref{fig:wf}a muestra las funciones de peso en cielo claro para aquellos canales del AMSU-A que tienen sensibilidad en la tropósfera. No se incluyen en la asimilación de observaciones los dos primeros canales ubicados en 23.8 y 31.4 GHz ya que estos son sensibles a la fase líquida de las nubes y a la emisividad de la superficie, aunque si aportan información para el control de calidad de las observaciones de otros canales.

Por ejemplo, la función de peso del canal 7 (54.94 GHz) muestra que la contribución máxima a la radiación de microondas detectada por el instrumento AMSU-A proviene de aproximadamente una altura de 10 km por encima de la superficie. Como indica la Tabla \ref{tab:table-rad} solo entre 2 y 4 canales del sensor AMSU-A se asimilaron en este trabajo. Entre los canales asimilados se encuentran los canales 5 a 8 (53.596, 54.4, 54.94 y 55.5 GHz respectivamente).
\begin{table}

\caption{\label{tab:table-rad}Lista de los sensores disponibles para cada plataforma satelital, el número de canales aceptados para su asimilación y el porcentaje de observaciones asimiladas calculado sobre todas las observaciones de radianzas y todos los ciclos de asimilación correspondientes al experimento RAD (ver Capítulo 3).}
\centering
\fontsize{9}{11}\selectfont
\begin{tabu} to \linewidth {>{\raggedright}X>{\raggedright}X>{\raggedleft}X>{\raggedright}X}
\toprule
Sensor & Plataforma & Canales asimilados & Porcentaje sobre el total\\
\midrule
AIRS & AQUA & 52 & 31.63 \%\\
\cmidrule{1-4}
 & NOAA15 & 2 & 3.31 \%\\
\cmidrule{2-4}
 & NOAA18 & 2 & 4.45 \%\\
\cmidrule{2-4}
\multirow[t]{-3}{*}{\raggedright\arraybackslash AMSUA} & METOP-A & 2 & 2.08 \%\\
\cmidrule{1-4}
 & METOP-A & 66 & 52.72 \%\\
\cmidrule{2-4}
\multirow[t]{-2}{*}{\raggedright\arraybackslash IASI} & METOP-B & 68 & 3.47 \%\\
\cmidrule{1-4}
 & NOAA19 & 2 & 0.68 \%\\
\cmidrule{2-4}
 & METOP-A & 3 & 0.8 \%\\
\cmidrule{2-4}
\multirow[t]{-3}{*}{\raggedright\arraybackslash MHS} & METOP-B & 3 & 0.85 \%\\
\bottomrule
\end{tabu}
\end{table}
De manera similar, el sensor microondas MHS es conocido por ser un ``sensor de humedad''. Este sensor tiene dos canales sensibles a la superficie y la presencia de nubes (canales 1 y 2: 89 y 157 GHz), y tres canales mayoritariamente sensibles al vapor de agua (canales 3, 4 y 5: 183.3 \(\pm\) 1.0, 183.3 \(\pm\) 3.0 y 190 GHz). Como indican las funciones de peso (Figura \ref{fig:wf}b), todos los canales del MHS se encuentran por debajo del tope del modelo, sin embargo, como se puede ver en la Tabla \ref{tab:table-rad}, entre 2 y 3 canales se asimilan en los experimentos llevados a cabo en esta tesis. El canal 5 no es utilizado en algunas situaciones ya que se ve afectado por la presencia de nubes y las observaciones son rechazadas durante el control de calidad. Los canales 1 y 2 no son asimilados debido a que la información que aportan proviene principalmente de la superficie.

En cuanto a los sensores multiespectrales, se asimilaron dos: el \emph{Atmospheric Infrared Sounder} (AIRS) y el \emph{Infrared Atmospheric Sounding Interferometer} (IASI) ubicados en distintas plataformas (ver Tabla \ref{tab:table-rad}). IASI mide la radiación emitida por la Tierra en 8461 canales que cubren el intervalo espectral del infrarrojo térmico entre 645 y 2760 \(cm^{-1}\) (3.62 - 15.5 \(\mu m\)). De manera similar, AIRS tiene canales en 3 rangos del espectro electromagnético: 645 y 1136 \(cm^{-1}\), 1265 y 1629 \(cm^{-1}\), y 2169 y 2674 \(cm^{-1}\) donde la transmisividad de la atmósfera varía rápidamente con la frecuencia y por lo tanto cada canal aporta información en distintos niveles de la atmósfera. Asimilar la totalidad de estos canales no es recomendable ya que tanto las observaciones como sus errores se encuentran muy correlacionados entre si, es por ello que numerosos estudios que se han enfocado en definir un conjunto de canales adecuado en el contexto de los pronósticos numéricos globales para su asimilación (por ejemplo, Collard, 2007; Rabier et al., 2002). Realizar un estudio de sensibilidad de los canales para la región de interés esta fuera del alcance de los objetivos de este trabajo, por lo que se decidió utilizar la configuración regional del sistema GSI en la cual se asimilan entre 52 y 68 canales distintos (ver Tabla \ref{tab:table-rad}). Las diferencias en la cantidad de canales y/o cantidad de observaciones asimiladas para un mismo sensor a bordo de distintas plataformas puede deberse a la presencia de errores sistemáticos o a fallas conocidas en los canales de sensores específicos que son rechazados durante la asimilación.


\begin{figure}

{\centering \includegraphics{thesis_files/figure-latex/wf-1} 

}

\caption{Funciones de peso calculada sobre cielos despejados asumiendo una atmósfera estándar para distintos sensores y canales, a) AMSU-A, canal 3 - 50.3 GHz, canal 4 - 52.8 GHz, canal 5 - 53.596 \(\pm\) 0.115 GHz, canal 6 - 54.4 GHz, canal 7 - 54.94 GHz; y b) MHS, canal 1 - 89.0 GHz, canal 2 - 157.0 GHz, canal 3 - 183.311 \(\pm\) 1.00 GHz, canal 4 - 183.311 \(\pm\) 3.00 GHz, canal 5 - 190.311 GHz.}\label{fig:wf}
\end{figure}
\hypertarget{goes-16}{%
\paragraph{GOES-16}\label{goes-16}}

En el trabajo también se evaluó la utilización del satélite geoestacionario GOES-16. En particular se utilizaron observaciones del sensor \emph{Advanced Baseline Imager} (ABI) que tienen una frecuencia temporal de 15 minutos y una resolución espacial de 2 kilómetros aportando más observaciones que el conjunto de satélites polares utilizados.

Teniendo en cuenta los objetivos propuestos, se asimilaron los canales sensibles al vapor de agua en distintos niveles, información que puede ser clave a la hora de pronosticar el desarrollo de convección húmeda profunda:
\begin{itemize}
\tightlist
\item
  Canal 8 (6.2 \(\mu m\)): vapor de agua en niveles altos
\item
  Canal 9 (6.9 \(\mu m\)): vapor de agua en niveles medios
\item
  Canal 10 (7.3 \(\mu m\)): vapor de agua en niveles bajos
\end{itemize}
La Figura \ref{fig:wf-goes} muestra las funciones de peso calculadas para una atmósfera estándar en latitudes medias para cada uno de estos canales. Si bien hay un solapamiento en las funciones de peso, para una atmósfera estándar, el canal 8 es más sensible al vapor de agua alrededor de los 350 hPa, el canal 9 alrededor de 450 hPa y el canal 10 alrededor de los 600 hPa aproximadamente. Además, la contribución de estos canales por encima de 50 hPa es mínima por lo que su asimilación en aplicación regionales donde el tope de la atmósfera suele ser bajo, no presenta grandes desafíos. En la Sección \ref{canales} se analizará que combinación de canales da mejores resultados en el contexto de este caso de estudio.


\begin{figure}

{\centering \includegraphics{thesis_files/figure-latex/wf-goes-1} 

}

\caption{Funciones de peso calculada sobre cielo claro para ABI, canal 8 - 6.2 \(\mu m\), canal 9 - 6.9 \(\mu m\), canal 10 - 7.3 \(\mu m\).}\label{fig:wf-goes}
\end{figure}
\hypertarget{conjuntos-de-observaciones-para-verificaciuxf3n}{%
\subsection{Conjuntos de observaciones para verificación}\label{conjuntos-de-observaciones-para-verificaciuxf3n}}

Para evaluar el desempeño del sistema de asimilación de datos presentado en esta tesis, utilizamos los siguientes conjuntos de observaciones:
\begin{itemize}
\item
  \textbf{Datos horarios en niveles de presión de ERA5 (Hersbach et al., 2018):} Las variables de interés (temperatura del aire, humedad y viento) fueron interpoladas desde la retícula del reanálisis que tiene una resolución de 0.25\(^{\circ}\) a la retícula del modelo para compararlas con el análisis de cada experimento.
\item
  \textbf{Multi-Network Composite Highest Resolution Radiosonde Data (Earth Observing Laboratory, 2020):} radiosondeos en alta resolución lanzados desde varias ubicaciones durante el periodo de la campaña de campo RELAMPAGO en conjunto con las radiosondeos operativos del SMN. Sólo se utilizaron para la validación los sondeos que no ingresaron en el sistema de asimilación y que corresponden a los sitios de observación de la campaña. El periodo del experimento abarca las misiones \emph{Intensive Observation Period} (IOP) 7 (21/11/2018) y 8 (22/11/2018), durante las cuales se lanzaron 74 radiosondeos en una pequeña zona cercana al centro del dominio experimental (Figura \ref{fig:dominio}b). Estas observaciones son de particular importancia ya que si bien corresponden a una región acotada del dominio, aportan información de alta resolución vertical y temporal en comparación con los radiosondeos operacionales que se lanzan una o dos veces al día. Esto además permite observar la capa límite a lo largo de un periodo de tiempo extendido donde normalmente no se cuenta con observaciones convencionales.
\item
  \textbf{IMERG Final Run (Huffman et al., 2018):} estimación de la precipitación a partir de datos de la constelación de satélites GPM (por su nombre en inglés Global Precipitation Measurement) con una resolución espacial de 0,01\(^{\circ}\) y una resolución temporal de 30 minutos para validar la habilidad de los pronósticos de 1 hora para representar la precipitación sobre el dominio. La estimación de la precipitación se realiza combinando observaciones de sensores de microondas activos y pasivos que se complementa con estimaciones provenientes de sensores infrarrojos a bordo de satélites polares. Los datos de IMERG son ampliamente utilizados en la región y han sido evaluados previamente con buenos resultados (Hobouchian et al., 2018).
\item
  \textbf{Datos del Sistema Nacional de Radares Meteorológicos (SINARAME):} Se utilizaron observaciones de radar para realizar una evaluación cualitativa y visual de la ubicación e intensidad de la actividad convectiva. Los datos provienen de 9 radares ubicados en el dominio y son provistos por la red de radares Doppler de doble polarización en banda C (de Elía et al., 2017) con una frecuencia temporal de 10 minutos. Para este trabajo se utilizó únicamente la máxima reflectividad de la columna (COLMAX) más cercana al momento de análisis que incluye el control de calidad de los datos (Arruti et al., 2021).
\item
  \textbf{Observaciones de las redes de estaciones meteorológicas automáticas:} Las observaciones de EMA descriptas en la Sección \ref{ema} fueron utilizadas en la verificación de pronósticos inicializados a partir de los análisis generados. Las observaciones utilizadas para validar los pronósticos son posteriores a las utilizadas para generar las condiciones iniciales de dichos pronósticos.
\end{itemize}
\hypertarget{la-asimilaciuxf3n-de-datos}{%
\section{La asimilación de datos}\label{la-asimilaciuxf3n-de-datos}}

La asimilación de datos combina de manera optima un pronóstico numérico o campo preliminar en un periodo de tiempo \(t\) con las observaciones disponibles para ese mismo tiempo, generando un análisis (Carrassi et al., 2018). Esta combinación optima toma en cuenta el error asociado a la predicción numérica (que incluye los errores debido a la imperfección de las condiciones iniciales y aquellos asociados a la imperfección del modelo) y el error de las observaciones (que resulta de los errores instrumentales y los errores de representatividad). Por esta razón el análisis es considerado desde un punto de vista estadístico y bajo ciertas hipótesis, como la \emph{mejor aproximación} disponible del estado real de la atmósfera.


\begin{figure}

{\centering \includegraphics[width=1\linewidth,]{/home/paola.corrales/tesis_doctorado/figure/ciclo_asimilacion_teorico} 

}

\caption{Esquema de un ciclo de asimilación típico.}\label{fig:ciclo-asimilacion-teorico}
\end{figure}
Un ciclo de asimilación de datos típico se muestra en la Figura \ref{fig:ciclo-asimilacion-teorico}. En un periodo de tiempo dado se comparan las observaciones disponibles con el campo preliminar para ese mismo periodo, generando así el análisis que se utilizará como condición inicial para un futuro pronóstico o campo preliminar. En el caso de modelos globales, típicamente cada ciclo de asimilación cada 6 horas utiliza el campo preliminar previo, es decir el pronóstico a 6 horas inicializado a partir del análisis anterior y las observaciones disponibles en la ventana de asimilación que corresponde a las 6 horas previas o en un periodo similar centrado en la hora del análisis. Muchos sistemas de asimilación utilizan, además, se utilizan estrategias 4D donde las observaciones dentro de la ventana de asimilación se comparan con el campo preliminar más cercano a esas observaciones.

En este sentido tanto el modelo como las observaciones cumplen un rol fundamental en la asimilación. Por un lado las observaciones permiten incorporar información continua para corregir los pronósticos. Por el otro, el modelo permite \emph{transportar} información de regiones donde existe mucha información disponible (por ejemplo, los continentes) a regiones donde las observaciones son escasas (zonas oceánicas) manteniendo los balances físicos que rigen los procesos atmosféricos.

Existen distintas técnicas de asimilación de datos, cada una con sus ventajas y desventajas que fueron descriptas por Dillon (2017) en su tesis doctoral. En este trabajo, al igual que en trabajos previos en Argentina, se utilizará la técnica \emph{Local Ensemble Transform Kalman filter} (LETKF, Hunt et al., 2007) que forma parte de un conjunto de técnicas basadas en Filtros de Kalman por ensambles (Evensen, 2009). Los Filtros de Kalman utilizan el ensamble, un conjunto de pronósticos ligeramente diferentes que se resuelven simultáneamente, para representar posibles estados futuros de la atmósfera cuya ocurrencia se considera equiprobable dado la incertidumbre presente en las condiciones iniciales. El ensamble además provee información dependiente de la dinámica durante la ventana de asimilación y permite estimar el error del modelo a lo largo del tiempo.

A fin de simplificar la siguiente discusión, asumimos que la comparación entre el modelo y las observaciones se realiza en un tiempo \(t\) y no en un periodo de tiempo. En el contexto de los Filtros de Kalman, el análisis o \(x_a\) estará representado por un vector columna de dimensión \(n_x\) x \(1\), donde \(n_x\) es la dimensión del estado simulado por el modelo. A su vez, \(x_a\) será el estado más probable de la atmósfera teniendo en cuenta las \(n_o\) observaciones \(y_o\) (vector de dimensión \(n_o\) x \(1\)) disponibles en un tiempo \(t\). Para asimilar las observaciones el primer paso es comparar el valor de la variable observada con el valor de la variable simulada por el modelo. En casos simple, por ejemplo para la temperatura o humedad en superficie, esto solo requiere la interpolación de la variable del modelo a la ubicación de las observaciones. En otros casos, por ejemplo cuando se trabaja con observaciones de satélite o radar, será necesario transformar las variables del modelo (ej. temperatura y humedad) para obtener las variables observadas (ej. temperatura de brillo). En la siguiente ecuación \(H\) es el operador de las observaciones que se encarga de las interpolaciones y transformaciones necesarias sobre el campo preliminar \(x_f\), representado como un vector de dimensión \(n_x\). Teniendo en cuenta las transformaciones que pueda realizar \(H\), este operador podría ser no lineal.
\begin{equation}
x_a = x_f + K[y_o - H(x_b )]
\label{eq:eq1}
\end{equation}
La diferencia entre las observaciones \(y_o\) y el campo preliminar se denomina innovación. El análisis \(x_a\) se obtiene aplicando las innovaciones al campo preliminar pesadas por una matriz \(K\) o Ganancia de Kalman. Esta matriz de dimensión \(n_x\) x \(n_o\) que incluye información sobre los errores del pronóstico y de las observaciones como se observa en la ecuación \eqref{eq:eq3},
\begin{equation}
K = PH^T (HPH^T + R^{-1})^{-1}
\label{eq:eq3}
\end{equation}
donde \(P\) es la matriz de covarianza de los errores del pronóstico. \(R\) es la matriz de covarianza de los errores de las observaciones que describe la magnitud del error de las observaciones en la diagonal y la correlación entre estos errores en los elementos fuera de la diagonal. Sin embargo, se asume que los errores de las observaciones son independientes y por lo tanto la matriz \(R\) es una matriz diagonal. La implementación de los los métodos de asimilación de datos suelen ser computacionalmente muy costosos, en parte debido a la estimación de la matriz \(P\). Los métodos de Filtro de Kalman por Ensambles tienen la ventaja de estimar la matriz \(P\) a partir del ensamble y actualizarla en cada ciclo de asimilación para incluir los errores dependientes de la dinámica del sistema siguiendo la ecuación:
\begin{equation}
P \approx \frac{1}{m-1} \sum_{k=1}^{m}(x_{f}^{k}-\overline{x}_f)(x_{f}^{k}-\overline{x}_f)^T
\label{eq:eq5}
\end{equation}
donde \(x_{f}^{k}\) es el pronóstico del miembro \emph{k-ésimo} del ensamble. Esta estimación será buena si el ensamble logra capturar los posibles estados futuros o en otras palabras si la dispersión es suficiente para estimar la incertidumbre de los pronósticos y capturar los cambios a lo largo de los ciclos de asimilación. El ensamble debe además tener un tamaño suficiente para representar las direcciones de máximo crecimiento de los errores que se asocian principalmente a la cantidad de modos inestables presentes en el sistema (Bousquet et al., 2008). Si esto ocurre la estimación de \(P\) será buena.

El método LETKF es una técnica eficiente ya que resuelve la ecuación \eqref{eq:eq1} en un espacio de dimensión reducida definido por los miembros del ensamble. Por lo tanto la matriz \(K\) no se resuelve explicitamente, eliminando la necesidad de computar la inversa de matrices de gran tamaño. Cómo en el resto de los métodos que usan filtro de Kalman, se localiza o restringe el área de influencia de las observaciones a un determinado radio de localización reduciendo el costo computacional necesario. La localización tiene además otra ventajas. Por un lado, y desde el punto de vista dinámico las inestabilidades en la atmósfera son de naturaleza local y por lo tanto el tamaño del ensamble necesario para representar estas inestabilidades será menor que si miramos el problema globalmente (Patil et al., 2001). Por otro lado la localización reduce el impacto de las covarianzas espurias entre entre variables del estado y observaciones que están alejadas entre si y cuyos errores son independientes reduciendo el ruido estadístico en la estimación del análisis. Además, este método calcula el análisis para cada punto de retícula individualmente, incorporando todas las observaciones que puedan tener influencia en ese punto al mismo tiempo permitiendo paralelizar los cálculos. De esta manera este método es hasta un orden de magnitud más rápido comparado con otros métodos desarrollados previamente (Whitaker et al., 2008).

\hypertarget{el-sistema-de-asimilaciuxf3n-gsi}{%
\subsection{El sistema de asimilación GSI}\label{el-sistema-de-asimilaciuxf3n-gsi}}

GSI por su nombre en inglés Gridpoint Statistical Interpolation, es un sistema de asimilación de datos de última generación, desarrollado inicialmente por el Environmental Modeling Center del NCEP. Se diseñó como un sistema 3DVAR tradicional aplicado en el espacio de puntos de retícula de los modelos para facilitar la implementación de covarianzas anisotrópicas no homogéneas (Wu et al., 2002; Purser et al., 2003a, 2003b).
Está diseñado para funcionar en varias plataformas computacionales, crear análisis para diferentes modelos numéricos de pronóstico, y seguir siendo lo suficientemente flexible como para poder manejar futuros desarrollos científicos, como el uso de nuevos tipos de observación, una mejor selección de datos y nuevas variables de estado (Kleist et al., 2009).

Este sistema 3DVAR sustituyó al sistema de análisis operativo regional de punto de retícula del NCEP por el Sistema de Predicción de Mesoescala de América del Norte (NAM) en 2006 y al sistema de análisis global \emph{Spectral Statistical Interpolation} (SSI) usado para generar las condiciones iniciales del \emph{Global Forecast System} (GFS) en 2007 (Kleist et al., 2009).
En los últimos años, GSI ha evolucionado para incluir varias técnicas de asimilación de datos para múltiples aplicaciones operativas, incluyendo 2DVAR (por ejemplo, el sistema \emph{Real-Time Mesoscale Analysis} (RTMA); Pondeca et al., 2011), la técnica híbrida EnVar (por ejemplo los sistemas de asimilación de datos para el GFS, el \emph{Rapid Refresh system} (RAP), el NAM, el HWRF, etc.), y 4DVAR (por ejemplo, el sistema de asimilación de datos para el Sistema Goddard de Observación de la Tierra, versión 5 (GEOS-5) de la NASA; Zhu and Gelaro, 2008).
GSI también incluye un enfoque híbrido 4D-EnVar que actualmente se utiliza para la generación del GFS.

Además del desarrollo de técnicas híbridas, GSI permite el uso de métodos de asimilación por ensambles. Para lograr esto, utiliza el mismo operador de las observaciones que los métodos variacionales para comparar el campo preliminar con las observaciones.
De esta manera los exhaustivos controles de calidad desarrollados para los métodos variacionales también son aplicados en la asimilación por ensambles.
El código EnKF fue desarrollado por el Earth System Research Lab (ESRL) de la National Oceanic and Atmospheric Administration (NOAA) en colaboración con la comunidad científica.
Contiene dos algoritmos distintos para calcular el incremento del analisis, el serial Ensemble Square Root Filter (EnSRF, Whitaker and Hamill, 2002) y el LETKF (Hunt et al., 2007) aportado por Yoichiro Ota de la Agencia Meteorológica Japonesa (JMA).

Para reducir el impacto de covarianzas espurias en el incremento que se aplica al análisis, los sistemas por ensamble aplican una localización a la matriz de covarianza de los errores de las observaciones \(R\) tanto en la dirección horizontal como vertical.
GSI usa un polinomio de orden 5 para reducir el impacto de cada observación de manera gradual hasta llegar a una distancia límite a partir de la cual el impacto es cero. La escala de localización vertical se define en términos del logaritmo de la presión y la escala horizontal usualmente se define en kilómetros.
Estos parámetros son importantes a la hora de obtener un buen análisis y dependen de factores como el tamaño del ensamble y la resolución del modelo. La configuración utilizada para los experimentos de esta tesis se describe en la Sección \ref{configmodelo}.

A su vez utiliza el Community Radiative Transfer Model (CRTM, Liu et al., 2008) como operador de las observaciones de radianzas que calcula la temperatura de brillo simulada por el modelo para poder compararlo con las observaciones de sensores satelitales.
GSI, además implementa un algoritmo de corrección de bias de las observaciones de radianzas de satélites.
La estimación hecha por el campo preliminar obtenida con el CRMT se compara con las observaciones de radianza para obtener la innovación.
Esta innovación a su vez se utiliza para actualizar los coeficientes que permiten estimar el bias de las observaciones que posteriormente se aplica para obtener una innovación corregida. Este proceso puede repetirse varias veces hasta que la innovación y los coeficientes de corrección de bias converjan.
Este algoritmo se describe en mayor detalle en la Sección \ref{sat}.

\hypertarget{el-modelo-de-transferencia-radiativa-crtm}{%
\subsubsection{El modelo de transferencia radiativa CRTM}\label{el-modelo-de-transferencia-radiativa-crtm}}

El CRTM es un modelo de transferencia radiativa rápido que fue desarrollado conjuntamente por el \emph{NOAA Center for Satellite Applications and Research} y el \emph{Joint Center for Satellite Data Assimilation} (JCSDA). Es un modelo utilizado ampliamente por la comunidad de sensoramiento remoto ya que es de código abierto y se encuentra disponible para su uso públicamente. Además, se utiliza para la calibración de instrumentos satelitales (Weng et al., 2013; Iacovazzi et al., 2020; Crews et al., 2021), y a su vez para generar retrievals a partir de observaciones satélites (Boukabara et al., 2011; Hu et al., 2019; Hu and Han, 2021). De especial importancia para este trabajo, es su uso como operador de observaciones en la asimilación de radianzas satelitales (Tong et al., 2020; Barton et al., 2021).

El CRTM es capaz de simular radianzas de microondas, infrarrojas y visibles, en condiciones de cielos despejados y nubosos, utilizando perfiles atmosféricos de presión, temperatura, humedad y otras especies como el ozono. Recientemente Cutraro et al. (2021) evaluaron su despeño en la región con buenos resultados en la simulación de observaciones de GOES-16.

El CRTM es un modelo de transferencia radiativa orientado a sensores, es decir que contiene parameterizaciones y tablas de coeficientes precalculadas específicamente para los sensores operativos. Incluye módulos que calculan la radiación térmica a partir de la absorción gaseosa, la absorción y dispersión de la radiación por aerosoles y nubes, y la emisión y reflexión de la radiación por la superficie terrestre. La entrada al CRTM incluye variables de estado atmosférico, por ejemplo, temperatura, vapor de agua, presión y concentración de ozono en capas definidas por el usuario, y variables y parámetros de estado de la superficie, incluyendo la emisividad, la temperatura de la superficie y el viento.

El CRTM permite simular observaciones satelitales de radianzas a partir del estado de la atmósfera. Esto es necesario para la asimilación de radianzas pero también es utilizado para verificar la precisión y los errores de estas observaciones.

El cálculo de las observaciones simuladas tiene un costo computacional muy alto ya que requiere transponer un matriz de grandes dimensiones y la minimización de una función de costo. Esta matriz \(K^{*}\) se construye a partir de las derivadas parciales de las radianzas con respecto a parámetros geofísicos. CRTM permite hacer estos cálculos de manera rápida para que pueda usarse en contextos operacionales.

Para obtener resultados rápidos, CRTM aplica ciertas simplificaciones y aproximaciones a la hora de resolver la ecuación de transferencia radiativa. En primer lugar asume que la atmósfera terrestre está formada por capas plano-paralelas y homogéneas en equilibrio termodinámico y donde los efectos tridimensionales y de polarización pueden ser ignorados.

En contextos de cielos despejados, además se asume que no existe dispersión y solo se considera la absorción de los gases en la atmósfera. En cielos nubosos, la dispersión generada por las nubes si es incluida. En este último caso, la ecuación de transferencia radiativa no se puede resolver analíticamente y se recurre a modelos numéricos.

\hypertarget{sat}{%
\subsubsection{Asimilación de radianzas en GSI}\label{sat}}

El preprocesamiento y control de calidad de los datos es un paso esencial en la asimilación de radianzas y depende de cada sensor y canal. Esto incluye principalmente un \emph{thinning} espacial, la corrección del bias, y en particular en este estudio la detección de observaciones de cielos nubosos. Primero se aplicó un thinning espacial. Durante el proceso de thinning las observaciones que se van a asimilar se eligen en función de su distancia a los puntos de la retícula del modelo, la calidad de la observación (basada en la información disponible sobre la calidad de los datos) y el número de canales disponibles (para el mismo píxel y sensor). El algoritmo de thinning determina la calidad de los datos en este punto según la información disponible sobre los canales que son conocidos por tener errores, la superficie por debajo de cada píxel (prefiere observaciones sobre el mar a las de la tierra o nieve) y predictores sobre la calidad de las observaciones (Hu et al., 2018). El objetivo al aplicar thinning es evitar incorporar información de procesos de escalas menores a las que puede representar el modelo y reducir la correlación en los errores de las observaciones que vienen de un mismo sensor.

Luego del thinning, se aplica la corrección de bias. La metodología de corrección de bias implementada en GSI tiene una componente dependiente de características termodinámicas del aire y otro dependiente del ángulo de escaneo (Zhu et al., 2014) y se calcula como una polinomio lineal de N predictores \(p_i(x)\), con coeficientes asociados \(\beta_i\). Por lo tanto, la temperatura de brillo corregida por bias (\(BT_{cb}\)) puede obtenerse como
\begin{equation}
\mathrm{\mathit{BT_{cb}} =\mathit{ BT} + \sum_{i = 0}^{N} \beta_i p_i (x)}
\label{eq:eq12}
\end{equation}
GSI tiene un término de corrección de bias constante (\(p_0 = 1\)) mientras que los términos restantes y sus predictores son el contenido de agua líquida de las nubes (CLW), la tasa de cambio de temperatura con la presión, el cuadrado de la tasa de cambio de temperatura con la presión y la sensibilidad de la emisividad de la superficie para tener en cuenta la diferencia entre la tierra y el mar. El bias dependiente del ángulo de escaneo se modela como un polinomio de 4\(^\circ\) orden (Zhu et al., 2014).

En el sistema GSI, los coeficientes \(\beta_i\) se entrenan utilizando un método de estimación variacional que genera los \(\beta_i\) que proporciona el mejor ajuste entre la simulación y las observaciones.

La metodología de detección de pixeles nubosos depende de la longitud de onda de las observaciones. Para las radianzas de microondas, las observaciones potencialmente contaminadas por nubes se detectan utilizando los índices de dispersión y del Liquid Water Path (LWP) calculados a partir de diferencias entre distintos canales de cada sensor (Zhu et al., 2016; Weston et al., 2019). Para los canales infrarrojos, las observaciones contaminadas por nubes se detectan utilizando el perfil de transmitancia calculado por el modelo CRTM. Además, GSI comprueba la diferencia entre las observaciones y la temperatura de brillo simulada para detectar los píxeles nublados. Un caso particular son las observaciones de ABI ya que se utiliza la mascara de nubes que se genera como producto de nivel 2 y que está disponible con la misma resolución que las observaciones. Esta máscara de nubes se genera combinando información de 8 canales del sensor ABI desde el punto de vista espacial y temporal.

Por otro lado, el control de calidad de GSI filtra aquellas observaciones de canales cercanos al rango visible sobre superficies de agua con un ángulo cenital mayor a 60\(^{\circ}\) para rechazar aquellas observaciones que pudieran estar contaminadas por reflexión. Para las observaciones en el infrarrojos y microondas también realiza una chequeo de la emisividad para detectar observaciones contaminadas por efecto de la superficie. Finalmente se aplica un \emph{gross check}, es decir, se compara la diferencia entre la observación y la observación simulada por el modelo y un umbral predefinido que depende del error de observación para rechazar observaciones erróneas.

La ubicación vertical de cada observación de radianza se estimó como el nivel del modelo en el que se maximizaba su función de peso calculada por el CRTM. La función de peso de cada canal corresponde al cambio en la transmitancia con la altura y su máximo describe la capa de la atmósfera desde donde la radiación captada por el canal fue emitida. Los sensores multiespectrales tienen una buena cobertura vertical y son capaces de captar desde la baja troposfera hasta la baja estratosfera. Los canales elegidos para la asimilación y sus errores asociados se definieron teniendo en cuenta la configuración que GSI utiliza para generar los análisis y pronósticos de GFS, el tope del modelo elegido en este trabajo (50 hPa) y la posible influencia de la superficie (Tabla \ref{tab:table-rad}).

\hypertarget{configmodelo}{%
\section{Configuración del sistema de asimilación}\label{configmodelo}}

Las simulaciones numéricas que conforman los experimentos que se discuten en este trabajo se realizan utilizando la versión 3.9.1 de modelo WRF (Skamarock et al., 2008).
Se utilizó una resolución horizontal de 10 km y 37 niveles en la vertical con el tope del modelo en 50 hPa.
Las condiciones iniciales y de contorno surgen del análisis del Global Forecast System con una resolución horizontal de 0,25\(^{\circ}\) y frecuencia temporal de 6 horas (GFS, National Centers for Environmental Prediction, National Weather Service, NOAA, U.S. Department of Commerce, 2015).
El dominio de 150 x 200 puntos de retícula cubre la zona indicada en la Figura \ref{fig:dominio} para capturar el desarrollo del SCM durante el periodo simulado.

Los análisis se generaron utilizando la implementación LETKF (V1.3, Hunt et al., 2007) que forma parte del sistema de asimilación Gridpoint Statistical Interpolation (GSI V3.8; Shao et al., 2016).
Se utilizó un enfoque de actualización rápida con análisis horarios y una ventana de asimilación centrada, lo que significa que se asimilaron todas las observaciones dentro de \(\pm\) 30 minutos del tiempo de análisis.
Además, las observaciones se asimilaron usando un enfoque 4D, es decir, comparándolas con el campo preliminar más cercano disponible con una frecuencia de 10 minutos.
Para las observaciones de satélite, se utilizó el CRTM como operador de observaciones para calcular las temperaturas de brillo simuladas por el modelo.

Se generó un ensamble de 60 miembros, cuya media al principio del ciclo de asimilación de datos se inicializa utilizando el análisis determístico del GFS al que se le suman perturbaciones aleatorias para generar el ensamble inicial. Las perturbaciones se generaron como diferencias escaladas entre dos estados atmosféricos aleatorios obtenidos a partir de los datos del \emph{Climate Forecast System Reanalysis} (CFSR) con una resolución horizontal de 0,5\(^{\circ}\) que tiene una evolución temporal suave (Necker et al., 2020; Maldonado et al., 2021). De este modo, preservamos el equilibrio hidrostatico y casi geostrófico de las escalas mayores. Este método ayuda a evitar una subestimación del spread del ensamble (Ouaraini et al., 2015). Las perturbaciones también se aplicaron en los bordes laterales para mantener niveles adecuados de spread dentro del dominio del ensamble.

Además de las perturbaciones aleatorias en los bordes laterales, se utilizó un esquema de multifísicas para representar mejor el aporte de los errores de modelo a la incertidumbre del pronóstico dentro del sistema de asimilación de datos. Utilizamos 9 configuraciones diferentes que consisten en la combinación de 3 esquemas de convección húmeda (Kain-Fritsch, Kain, 2004; Grell-Freitas, Grell and Freitas, 2013; y Betts-Miller-Janjic, Janjić, 1994) y 3 esquemas de capa límite planetaria (esquema de la Universidad de Yonsei, Hong, Noh, et al., 2006; Esquema Mellor-Yamada-Janjic, Janjić, 1994; y Mellor-Yamada Nakanishi Niino, Nakanishi and Niino, 2009). La distribución de estas parametrizaciones entre los 60 miembros del ensamble se muestra en la Tabla \ref{tab:miembros-desc}. Todos los miembros del ensamble utilizan el modelo de superficie terrestre (Noah-MP, Chen and Dudhia, 2001) y parametrizaciones de microfísica (esquema de un solo momento de 6 clases del WRF, Hong, Kim, et al., 2006) y de procesos radiativos (esquema de onda corta y onda larga del RRTMG, Iacono et al., 2008).
\begin{table}

\caption{\label{tab:miembros-desc}Generación de los 60 miembros del ensamble multifísica como combinación de parametrizaciones de Cumulus y PBL}
\centering
\fontsize{9}{11}\selectfont
\begin{tabular}[t]{c>{\centering\arraybackslash}p{8em}>{\centering\arraybackslash}p{8em}>{\centering\arraybackslash}p{8em}}
\toprule
\multicolumn{1}{c}{ } & \multicolumn{3}{c}{PBL} \\
\cmidrule(l{3pt}r{3pt}){2-4}
Cumulus & MYJ & MYNN2 & YSU\\
\midrule
BMJ & 5, 14, 23, 32, 41, 50, 59 & 8, 17, 26, 35, 44, 53 & 2, 11, 20, 29, 38, 47, 56\\
GF & 6, 15, 24, 33, 42, 51, 60 & 9, 18, 27, 36, 45, 54 & 3, 12, 21, 30, 39, 48, 57\\
KF & 4, 13, 22, 31, 40, 49, 58 & 7, 16, 25, 34, 43, 52 & 1, 10, 19, 28, 37, 46, 55\\
\bottomrule
\end{tabular}
\end{table}
Para reducir el efecto de las correlaciones espurias en la estimación de las covarianzas de los errores de los pronósticos, utilizamos un radio de localización horizontal de 180 km y un radio de localización vertical de 0,4 (en coordenadas del logaritmo de la presión) como en Dillon et al. (2021) para todos los tipos de observaciones.
Se aplicó con un parámetro de inflación \(\alpha=0,9\) que incrementa el spread del análisis para mitigar el impacto de los errores de muestreo y para considerar los errores del modelo que no se tienen en cuenta en el enfoque multifísica del ensamble (Whitaker and Hamill, 2012).

Los errores de las observaciones utilizados fueron definidos de acuerdo a las tablas de errores disponibles como parte del sistema GSI. Para las observaciones de radianzas polares se aplicó un thinning con una retícula de 60 km siguiendo Singh et al. (2016), Jones et al. (2013) y Lin et al. (2017) ya que utilizan configuraciones de modelo similares a la usada en este trabajo. Para las observaciones de GOES-16 se realizaron pruebas de sensibilidad para determinar la resolución de thinning más apropiada para este tipo de observaciones que se describen en la Sección \ref{thinning}.

Los coeficientes de corrección de bias para las observaciones de satélites polares se inicializaron a las 18 UTC del 18 de noviembre de 2018 a partir los coeficientes generados para la misma hora por el sistema GFS. El sistema de asimilación se configuró para utilizar una varianza de error de los coeficientes constante de 0,01 para evitar grandes ajustes en los coeficientes estimados en cada momento. Por razones que se describen en la Sección \ref{asim-abi} se decidió no aplicar corrección de bias a las observaciones de GOES-16.

La Figura \ref{fig:flujo-asimilacion} muestra el proceso de asimilación de datos de manera resumida. En primer lugar se procesan las observaciones en formato bufr y se genera el campo prelimilar utilizando el modelo WRF. El sistema de asimilación se encarga, de la comparación entre las observaciones y el campo preliminar. En los casos necesarios, por ejemplo durante la asimilación de radianzas, intervienen operadores de las observaciones complejos como el CRTM. En este paso, además, las observaciones pasan por el control de calidad correspondiente. Finalmente, el sistema de asimilación aplica la innovación al campo preliminar para generar el análisis que se usará como condición inicial para futuros pronósticos o subsiguientes ciclos de asimilación.


\begin{figure}
\includegraphics[width=1\linewidth,]{figure/flujo} \caption{Diagrama del proceso de asimilación de datos.}\label{fig:flujo-asimilacion}
\end{figure}
\hypertarget{muxe9todos-de-verificaciuxf3n}{%
\section{Métodos de verificación}\label{muxe9todos-de-verificaciuxf3n}}

Se seleccionaron un conjunto de métricas para evaluar diferentes aspectos del análisis obtenido para cada experimento y los pronósticos inicializados a partir de ellos. Estas métricas incluyen una validación de cómo se cuantifica la incertidumbre en el campo preliminar y en el análisis, y cómo los diferentes experimentos se ajustan a un conjunto independiente de observaciones que no se asimilan.

Para evaluar la consistencia de la estimación de la incertidumbre en el campo preliminar y en el análisis utilizamos la Reduced Centered Random Variable o RCRV (Candille et al., 2007) que se define como:
\begin{equation}
\mathit{RCRV} = \frac{m - x_o}{\sqrt{\sigma_o^2 + \sigma^2}}
\label{eq:eq6}
\end{equation}
donde \(x_o\) es la observación asimilada y la desviación estándar de su error \(\sigma_o\), \(m\) la media del ensamble del análisis en el espacio de las observaciones, y \(\sigma\) la desviación estándar del ensamble.
La media de \(RCRV\) calculada sobre todas las realizaciones representa el sesgo de la media del conjunto con respecto a las observaciones normalizado por la desviación estándar estimada de la distancia entre dicha media y las observaciones:
\begin{equation}
\mathit{mean RCRV} = E[RCRV]
\label{eq:eq7}
\end{equation}
La desviación estándar de la \(RCRV\) o \(sd RCRV\) mide la concordancia de la dispersión del ensamble y el error de observación con respecto a la distancia entre la media del ensamble y las observaciones, y por lo tanto, la sobre o subdispersión sistemática del ensamble:
\begin{equation}
\mathit{sd RCRV} = \sqrt{\frac{1}{N -1}\sum_{i=1}^{N}(RCRV_i - \mathit{mean RCRV})^2}
\label{eq:eq8}
\end{equation}
donde \(N\) es la cantidad de observaciones utilizadas. Suponiendo que el error de observación fue estimado con precisión, un \(sd RCRV > 1\) indica que el ensamble es subdispersivo (es decir, la distancia entre las observaciones y los pronósticos es mayor de lo esperado), y un \(sd RCRV < 1\) indica que el conjunto es sobredispersivo (es decir, la distancia entre las observaciones y los pronósticos es menor de lo esperado). Un sistema consistente no tendrá sesgo (\(media RCRV = 0\)) y tendrá una desviación estándar igual a 1 (\(sd RCRV = 1\)).

Para analizar el ajuste del campo preliminar y el análisis a un conjunto de observaciones independientes se calculó la raíz del error cuadrático medio (RMSE) y el BIAS:
\begin{equation}
\mathit{RMSE} = \sqrt{\frac{1}{N}\sum_{i = 1}^{N} (H(x_i) - y_{oi})^{2}}
\label{eq:eq9}
\end{equation}
\begin{equation}
\mathit{BIAS} = \frac{1}{N}\sum_{i = 1}^{N} (X_i - y_{oi})
\label{eq:eq10}
\end{equation}
donde \(H(x)\) representa el modelo interpolado al espacio de la observaciones, \(y_{o}\) las observaciones independientes, y N es el tamaño de la muestra.

Para aquellas observaciones disponibles en punto de grilla tales cómo la precipitación las estimaciones de precipitación de IMERG y la temperatura de brillo de GOES-16, calculamos el Fractions Skill Score (FSS, Roberts, 2008) para diferentes radios de influencia y umbrales de precipitación o temperatura de brillo con el objetivo de hacer una verificación espacial:
\begin{equation}
\mathit{FSS} = 1-\frac{\sum_{i=1}^{N} ({P_x}_i-{P_o}_i)^{2}}{\sum_{i=1}^{N} ({P_x}_i)^{2}+\sum_{i=1}^{N} ({P_o}_i)^{2}}
\label{eq:eq11}
\end{equation}
donde \(P_{oi}\) es la proporción de puntos de retícula en el subdominio \(i-ésimo\), definido por el radio de influencia, donde la precipitación acumulada observada es mayor que un umbral especificado. Siguiendo a Roberts et al. (2020), \(P_{xi}\) se calcula sobre el ensamble completo y cuantifica la probabilidad de que la precipitación sea mayor al mismo umbral en cada punto de retícula, que luego es promediando sobre el subdominio \(i-ésimo\).

Esta métrica, a diferencia de otras como el RMSE que se calcula punto a punto, permite permite comparar el pronóstico con observaciones en distintas escalas espaciales. Esto es particularmente importante para variables como la precipitación o la temperatura de brillo que muchas veces es representada correctamente por el modelo en forma y magnitud pero en posiciones ligeramente distintas a lo observado.

Para los pronósticos por ensambles se calculó la probabilidad de precipitación por encima de determinados umbrales en cada punto de retícula como la proporción de miembros del ensamble que pronosticaron precipitación por encima de cada umbral. Además se generaron gráficos de confiabilidad (Wilks, 2011) con intervalos de probabilidades pronosticadas de 10\% para analizar la calibración de los pronósticos inicializados a partir de los análisis de cada experimento.

Finalmente se calculó el índice de Brier (Brier, 1950) para evaluar el error cuadrático medio de las probabilidades de precipitación calculadas de acuerdo a la siguiente ecuación:
\begin{equation}
\mathit{BS} = \frac{1}{N}\sum_{i=1}^{N} ({P_i}-{O_i})^{2}
\label{eq:eq13}
\end{equation}
donde \(P_i\) es la probabilidad pronosticada de la ocurrencia de un evento en el punto \(i-ésimo\) y \(O_i\) será 1 si se observó efectivamente la ocurrencia del evento y 0 si no.

\hypertarget{recursos-computacionales}{%
\section{Recursos computacionales}\label{recursos-computacionales}}

Todos los experimentos corrieron en la supercomputadora Cheyenne del \emph{National Center for Atmospheric Research} (NCAR) (Computational and Information Systems Laboratory, 2019) que cuenta con 4032 nodos con 36 núcleos cada uno y 313 Tb de memoria de trabajo. El sistema de asimilación GSI y el modelo numérico WRF están programados en su totalidad en Fortran 90. Fue necesario incluir y modificar rutinas escritas en este lenguaje para ampliar las funcionalidades del sistema de asimilación, necesarias para algunos experimentos.

El postprocesamiento y análisis se realizó usando los recursos computacionales del Centro de Investigaciones del Mar y la Atmósfera. El análisis de datos se generó utilizando el lenguaje de programación R (R Core Team, 2020), utilizando los paquetes data.table (Dowle and Srinivasan, 2020) para trabajar con grandes volúmenes de datos y metR (Campitelli, 2020) para la lectura de archivos NetCDF y la visualización de datos grillados, entre otros.
Todos los gráficos se han realizado con ggplot2 (Wickham, 2009) y la versión final de la tesis se generó con knitr, rmarkdown (Xie, 2015; Allaire et al., 2019) y thesisdown (Ismay and Solomon, 2022).

\hypertarget{disponibilidad-de-los-datos}{%
\subsection{Disponibilidad de los datos}\label{disponibilidad-de-los-datos}}

Debido a que cada experimento de asimilación, que incluye la media y miembros del análisis, media del campo preliminar, archivos diagnósticos y con información de las observaciones durante 66 ciclos de asimilación, requiere aproximadamente 1 Tb de memoria en disco, no es posible almacenar estos datos de manera abierta. Sin embargo los datos que surgen del postprocesamiento para generar los resultados de la tesis se encuentran disponibles en Zenodo (\url{doi:10.5281/zenodo.7968629}, versión 0.9).

\hypertarget{cap-3-analisis}{%
\chapter{Asimilación de observaciones de estaciones meteorológicas automáticas, vientos derivados de satélite y radianzas de satelites polares}\label{cap-3-analisis}}

Este capítulo busca contribuir a la cuantificación y comparación del impacto de asimilar observaciones de las estaciones meteorológicas automáticas de alta resolución, viento estimando por satélite y radianzas de cielo despejado de satélites polares a la hora de aplicar la asimilación de datos para mejorar los pronósticos numéricos de eventos severos sobre Sudamérica, donde la red de observación convencional es bastante escasa y otras fuentes de información podrían potencialmente cubrir las áreas menos observadas y aportar información en las escalas más pequeñas.

\hypertarget{metodologuxeda}{%
\section{Metodología}\label{metodologuxeda}}

\hypertarget{config}{%
\subsection{Configuración de los experimentos}\label{config}}

Para investigar el impacto de las diferentes observaciones en el análisis, se realizaron cuatro experimentos de asimilación de datos utilizando diferentes conjuntos de observaciones (Tabla \ref{tab:table-exp}). El experimento CONV utiliza únicamente las observaciones convencionales de PREPBUFR. En un segundo experimento, denominado AWS, se asimilan todas las observaciones incluidas en CONV más las observaciones de EMA con 10 minutos de frecuencia. En el tercer experimento, denominado SATWND, se asimilan las observaciones del experimento AWS junto con los vientos estimados por satélite. Por último, un cuarto experimento, denominado RAD, asimila todas las observaciones mencionadas previamente más las radianzas en cielo despejado procedentes de los sensores a bordo de los satélites de órbita polar, como se describe en la Sección \ref{sat}.
\begin{table}

\caption{\label{tab:table-exp}Tipos de observaciones asimiladas en cada experimento.}
\centering
\begin{tabu} to \linewidth {>{\raggedright\arraybackslash}p{8em}>{\centering\arraybackslash}m{2.5em}>{\centering\arraybackslash}m{2.5em}>{\centering\arraybackslash}m{3.3em}>{\centering\arraybackslash}m{3.3em}}
\toprule
Obs type & CONV & AWS & SATWND & RAD\\
\midrule
Convencional (PREPBUFR) & x & x & x & x\\
Convencional (EMA) &  & x & x & x\\
Viento estimado por satélite &  &  & x & x\\
Radiancias &  &  &  & x\\
\bottomrule
\end{tabu}
\end{table}
La distribución horizontal del promedio de observaciones asimiladas en cada ciclo de asimilación de cada experimento se muestra en la Figura \ref{fig:obs-horizontal}. El mayor número de observaciones asimiladas sobre el centro y el este del dominio corresponde a las observaciones de EMA. En la Figura \ref{fig:obs-cycle}a se muestra el número de observaciones asimiladas a lo largo del tiempo. Los máximos locales a las 12 y a las 00 UTC encontrados principalmente en CONV corresponden a las observaciones obtenidas con los radiosondeo operativos. La fuerte variabilidad en el número de observaciones de radianzas por ciclo es también notable y depende de la cobertura de cada satélite. Los máximos a las 13-14 y 01-02 UTC en RAD corresponden a la contribución de los sensores multiespectrales. La distribución vertical del número medio de observaciones por ciclo (Figura \ref{fig:obs-cycle}b) muestra un máximo en niveles bajos debido a las observaciones de EMA. Los vientos estimados por satélite, disponibles para la asimilación, tienen un máximo en la troposfera alta (entre 500-250 hPa). Por encima de 850 hPa, la mayoría de las observaciones corresponden a radianzas.


\begin{figure}
\includegraphics{thesis_files/figure-latex/obs-horizontal-1} \caption{Distribución espacial horizontal media de las observaciones disponibles en cada ciclo de asimilación para los experimentos a) CONV, b) AWS, c) SATWND y d) RAD calculados sobre cajas de 2,5\(^{\circ}\).}\label{fig:obs-horizontal}
\end{figure}

\begin{figure}
\includegraphics{thesis_files/figure-latex/obs-cycle-1} \caption{a) Número de observaciones asimiladas por ciclo y b) número medio de observaciones asimiladas por ciclo dividido en capas verticales de 50 hPa de espesor para los experimentos CONV (cuadrados azules), AWS (puntos celestes), SATWND (triángulos naranjas) y RAD (diamantes rojos).}\label{fig:obs-cycle}
\end{figure}
Todos los experimentos de asimilación comienzan a las 18 UTC del 20 de noviembre de 2018 y continúan hasta las 12 UTC del 23 de noviembre (un total de 67 horas/ciclos de asimilación). El ensamble inicial de 60 miembros se genera según se explica en la Sección \ref{configmodelo} y todos los experimentos se inician con un periodo de spin up sin asimilación de observaciones entre las 12 UTC y las 18 UTC del 20 de Noviembre (Figura \ref{fig:cycle}).


\begin{figure}
\includegraphics[width=1\linewidth,]{/home/paola.corrales/tesis_doctorado/figure/analisis} \caption{Diagrama de los ciclos de asimilación entre las 18 UTC del 20 de noviembre y las 12 UTC del 23 de noviembre más el periodo de spin up de 6 horas. La sección ampliada muestra la asimilación horaria que se realiza dentro de una ventana centrada de una hora y la incorporación de condiciones de borde del GFS cada 6 horas. Se muestran las dos misiones IOP de la campaña RELAMPAGO.}\label{fig:cycle}
\end{figure}
\hypertarget{resultados}{%
\section{Resultados}\label{resultados}}

\hypertarget{consistencia-del-ensamble}{%
\subsection{Consistencia del ensamble}\label{consistencia-del-ensamble}}

Para investigar la capacidad de la media del ensamble del campo preliminar para ajustarse a las observaciones teniendo en cuenta las incertidumbres del pronóstico y de las observaciones, se calcularon el \(meanRCRV\) y el \(sdRCRV\) para el experimento RAD. Como este experimento asimila todos los tipos de observaciones utilizados en este trabajo, es posible analizar la coherencia del ensamble comparándolo con cada tipo de observación. La figura \ref{fig:rcrv-sfc} muestra el \(sdRCRV\) para las observaciones de superficie promediadas en una retícula de 2,5°. El \(sdRCRV\) para las observaciones del viento (Figura \ref{fig:rcrv-sfc}a) es cercano a 1, lo que sugiere una buena concordancia entre la dispersión del ensamble, el error de pronóstico y el error de observación. Para la temperatura (Figura \ref{fig:rcrv-sfc}b), los resultados son similares, salvo que para algunas zonas del oeste del dominio donde el \(sdRCRV\) puede llegar a ser de 4,5. Estos valores más altos de \(sdRCRV\) pueden estar asociados a errores sistemáticos derivados de las diferencias entre la topografía del modelo y al altura a la que están ubicadas las observaciones. La circulación de pequeña escala asociada al terreno complejo podría no estar bien resuelta por el modelo contribuyendo a aumentar la distancia entre el pronóstico y las observaciones. Estos aspectos no suelen ser captados por la dispersión del ensamble, a menos que se utilice un esquema de inflación dependiente del espacio bien ajustado, lo que conduciría a valores de \(sdRCRV\) mas cercanos a 1.

La figura \ref{fig:rcrv-profile} muestra la media y el desvío estándar del RCRV para las observaciones de altura. Las figuras \ref{fig:rcrv-profile}a-b muestran las estadísticas de RCRV para los radiosondessondeo (ADPUPA) y aviones (AIRCAR y AIRCFT). Tanto ADPUPA como AIRCFT muestran, en general, un buen acuerdo entre la dispersión del ensamble y el error de observación. Como las observaciones de sondeo y sus errores asociados son considerados de buena calidad, este resultado indica que el ensamble tiene una dispersión adecuada. AIRCAR presenta un perfil irregular con valores \(sdRCRV\) que sugieren que el error de este tipo de observación está sobrestimado. ADPUPA y AIRCAR presentan un perfil de \(meanRCRV\) cercano a cero en los niveles medios y altos. En niveles bajos, el perfil \(meanRCRV\) es positivo, mostrando un bias cálido presente en el modelo, característica ya estudiada en Ruiz et al. (2010) y Dillon et al. (2021).

Las observaciones de vientos estimadas por satélites varían en número dependiendo del satélite y del nivel. En la Figura \ref{fig:rcrv-profile}c sólo se incluye el \(RCRV\) calculado con al menos 100 observaciones disponibles para cada sensor y nivel. En niveles bajos, donde no hay muchas observaciones disponibles, los perfiles de \(meanRCRV\) y \(sdRCRV\) muestran una mayor desviación del comportamiento esperado con un bias negativo, y una posible sobreestimación del error de observación. Las estimaciones del viento derivadas de los canales de vapor de agua son abundantes por encima de 500 hPa, donde su bias es cercano a cero. La única excepción son las observaciones de EUMETSAT que contribuyen muy poco en la región.

Los perfiles de \(meanRCRV\) calculados a partir de las observaciones de radianzas (Figura \ref{fig:rcrv-profile}d) no muestran casi ningún bias y lo mismo ocurre si se calcula el \(mean RCRV\) sobre cada canal de cada sensor (no incluido en el trabajo). Esto indica que el algoritmo de corrección del bias funciona como se esperaba. Los valores de \(sd RCRV\) son inferiores a 1 para todos los sensores, posiblemente debido a una sobreestimación de los errores de observación para reducir la influencia de las observaciones potencialmente erróneas.

En general, estos resultados indican que la dispersión del ensamble es coherente con el error de pronóstico a corto plazo y que los errores sistemáticos son relativamente pequeños para la mayoría de los tipos de observación utilizados en este trabajo. Además, estos resultados sugieren que el parámetro de inflación \(\alpha = 0,9\) es adecuado para el sistema.


\begin{figure}
\includegraphics{thesis_files/figure-latex/rcrv-sfc-1} \caption{\(sd RCRV\) calculado para observaciones de superficie (de PREPBUFR y EMA) de a) viento, y b) temperatura promediados en cajas de 2.5º para el experimento RAD. Se usaron las observaciones agregadas de cada ciclo de asimilación horario para todo el periodo del experimento.}\label{fig:rcrv-sfc}
\end{figure}

\begin{figure}
\includegraphics{thesis_files/figure-latex/rcrv-profile-1} \caption{Perfiles verticales de \(mean RCRV\) (línea punteada) y \(sd RCRV\) (línea sólida) para observaciones de a) temperatura y b) viento de radiosondeo y aviones, c) viento estimado por satélites, y d) temperatura de brillo para el experimento RAD. Se usaron las observaciones agregadas de cada ciclo de asimilación horario para todo el periodo del experimento.}\label{fig:rcrv-profile}
\end{figure}
\hypertarget{impacto-analisis}{%
\subsection{Impacto de la asimilación de las distintas observaciones}\label{impacto-analisis}}

Esta sección presenta el impacto de la asimilación de diferentes tipos de observación sobre algunas variables que son particularmente relevantes para el desarrollo de convección húmeda profunda. El análisis se realiza sobre un dominio más pequeño (recuadro rojo en la Figura \ref{fig:dominio}a) que corresponde a la región directamente afectada por el SCM. Las figuras \ref{fig:TQ-diff}a-c muestran la diferencia entre experimentos del perfil vertical de temperatura de la media del análisis promediado espacialmente. Al promediar las diferencias entre dos experimentos se puede aislar el impacto sistemático producido por los diferentes sistemas de observación sobre la situación analizada. Durante el primer día, la asimilación de las observaciones de EMA da como resultado el desarrollo de una PBL más fría. Este efecto de enfriamiento tiene un claro ciclo diurno, siendo más fuerte durante la noche (Figura \ref{fig:TQ-diff}a). Durante el segundo día del experimento, el impacto de las observaciones de EMA se extiende a la troposfera media y alta, coincidiendo con la fase madura del SCM. La diferencia positiva que se observa en AWS-CONV entre 500 y 200 hPa se produce por el desarrollo de convección más intensa en AWS en comparación con CONV. Este es un buen ejemplo de cómo la información en niveles bajos de la atmósfera proporcionada por las estaciones meteorológicas de superficie puede extenderse rápidamente a la troposfera en presencia de convección húmeda profunda. Aunque la circulación media y alta puede tener un impacto importante en la organización y evolución del SCM sobre la región, los vientos estimados por satélite no tuvieron un impacto apreciable en la temperatura y humedad media (Figura \ref{fig:TQ-diff}b-e), posiblemente debido al alto valor del error de estas observaciones utilizados para la asimilación.
Durante el primer día del experimento, la asimilación de las radianzas produce un efecto de calentamiento en la PBL que compensa parcialmente el efecto de enfriamiento de las observaciones de EMA (Figura \ref{fig:TQ-diff}c). No se encuentra un impacto sistemático claro por encima de la PBL durante este periodo para estas variables. Durante el segundo día, el impacto de las observaciones de radianza se observa en toda la troposfera con una distribución que es similar al impacto encontrado en el experimento AWS pero con signo opuesto.

Comparando la representación de la humedad específica en los experimentos (Figuras \ref{fig:TQ-diff}d-f), el impacto de la asimilación de EMA, que tienen una resolución espacial y temporal mayor, es muy importante en niveles bajos (Figura \ref{fig:TQ-diff}d). La PBL en el experimento AWS es sistemáticamente más húmeda que en el experimento CONV, especialmente durante la noche. El aumento de la humedad en niveles bajos gracias a la asimilación de observaciones de una red de superficie más densa corrige el bias seco reportado previamente en el modelo WRF sobre la región (Ruiz et al., 2010; Matsudo et al., 2021; Casaretto et al., 2022). El humedecimiento de la PBL es impulsado principalmente por la covarianza entre la temperatura y la humedad específica dentro de la PBL. En el experimento y sobre el centro del dominio, esta covarianza se mantiene negativa, generando un aumento de la humedad en niveles bajos a medida que las observaciones introducen correcciones negativas de temperatura. Como en el caso de la temperatura, el impacto sistemático de los vientos estimados por satélite sobre la humedad es pequeño (Figura \ref{fig:TQ-diff}e). La figura \ref{fig:TQ-diff}f muestra que las radianzas reducen la humedad en niveles medios y bajos durante el primer día del experimento. El efecto de secamiento se extiende a niveles medios y bajos durante el segundo día del experimento, coincidiendo con el desarrollo del SCM entre las 00 y las 12 UTC del 22 de noviembre.


\begin{figure}

\includegraphics{thesis_files/figure-latex/TQ-diff-1} \hfill{}

\caption{Diferencia entre la media del ensamble de los análisis a) y d) AWS-CONV, b) y e) SATWND-AWS, y c) y f) RAD-SATWND para los perfiles verticales espacialmente promediados de la temperatura (a, b y c, en \(K\)) y la humedad específica (d, e y f en \(gkg^{-1}\)) calculados sobre el dominio interior (recuadro rojo en la Figura \ref{fig:dominio}a) para cada ciclo de análisis.}\label{fig:TQ-diff}
\end{figure}
De las Figuras \ref{fig:TQ-diff} surge que el impacto el impacto que genera la asimilación de las observaciones de EMA es muy importante particularmente en niveles bajos. Para entender mejor el origen de estos impactos, las Figuras \ref{fig:ana-guess} muestran perfiles verticales medios a lo largo del tiempo para la diferencia entre el análisis y el campo preliminar o \emph{actualización} de los experimentos CONV y AWS. Las observaciones convencionales y en particular los radiosondeos operativos generan un calentamiento de la PBL en CONV, particularmente previo al desarrollo de la convección (Figura \ref{fig:ana-guess}a). Por otro lado las observaciones de EMA disminuyen la temperatura en niveles bajos durante la noche y la aumentan durante el día, lo que podría indicar que el modelo genera una PBL con menos variabilidad de temperatura. Luego del desarrollo de la convección, el impacto de las observaciones de EMA está asociado a un enfriamiento de la capa límite lo que sugiere que el ingreso de aire frió simulado por el modelo es más débil que lo observado (Figura \ref{fig:ana-guess}b). En las Figuras \ref{fig:ana-guess}c-d se observa nuevamente un aumento de la humedad en niveles bajos gracias a la asimilación de observaciones de EMA particularmente durante la noche disminuyendo el bias seco presente en el modelo.


\begin{figure}

{\centering \includegraphics{thesis_files/figure-latex/ana-guess-1} 

}

\caption{Perfiles verticales medios de la diferencia entre el análisis y el campo preliminar para a) y d) CONV, b) y b) y d) AWS para los perfiles verticales espacialmente promediados de la temperatura (a y b en \(K\)) y la humedad específica (d y c en \(gkg^{-1}\)) calculados sobre el dominio interior (recuadro rojo en la Figura \ref{fig:dominio}a) para cada ciclo de análisis.}\label{fig:ana-guess}
\end{figure}
El impacto en las componentes del viento se muestran en la Figura \ref{fig:UV-diff}, junto con la correspondiente componente del viento promediada en el experimento con el mayor número de observaciones asimiladas (por ejemplo, la Figura \ref{fig:UV-diff}a muestra la diferencia de viento zonal entre AWS y CONV y el viento zonal para AWS). La asimilación de EMA produce un viento más del este y menos del norte en niveles bajos durante los dos primeros días de análisis (Figuras \ref{fig:UV-diff}a,b). Existe un ciclo diurno asociado al impacto de las estaciones meteorológicas automáticas sobre el viento meridional (Figura \ref{fig:UV-diff}d) con una mayor reducción del viento del norte durante las horas nocturnas. Esto indica que las observaciones en superficie están reduciendo la intensidad del jet de capas bajas presente en el entorno preconvectivo. Después de las 18 UTC del 22 de noviembre, se observa el efecto contrario cuando el SCM se desplaza por el dominio hacia el noreste. Tras el inicio de las celdas convectivas, el impacto sistemático en el campo de viento es mayor en los niveles medios y altos (Figuras \ref{fig:UV-diff}d, f). Durante los días 22 y 23 de noviembre el impacto de asimilar EMA produce un aumento del viento del norte en los niveles superiores de la tropósfera. Esto podría ser una consecuencia de un SCM más intenso que produce un aumento del flujo de salida del lado polar del SCM. Aunque las observaciones de viento estimada por satélites producen el mayor impacto en niveles medios y altos, donde el número de observaciones es mayor, el impacto sistemático es en general menor que el producido por la asimilación de datos de EMA (Figuras \ref{fig:UV-diff}b, e). La razón del pequeño impacto observado en SATWND podría estar asociada al valor del error de observación utilizado para la asimilación de estas observaciones.

La asimilación de las radianzas produce una reducción del viento del oeste con respecto a SATWND en niveles bajos y altos (Figura \ref{fig:UV-diff}c). Para el viento meridional, estas observaciones producen un aumento de 1 \(ms^{-1}\) en el flujo del norte en niveles bajos, opuesto al generado por la asimilación de las observaciones de EMA durante la noche, entre las 03 y las 12 UTC, previas al desarrollo del SCM (Figura \ref{fig:UV-diff}f). En niveles superiores y durante los días 22 y 23 de noviembre, el impacto medio de la asimilación de radianzas es una disminución de la velocidad del viento. El campo de viento meridional a 200 hPa en diferentes momentos muestra que el flujo de salida del SCM es aún más intenso que en los otros experimentos, mientras que el viento sur por delante del SCM también aumenta produciendo una reducción media del viento del norte (Figura \ref{fig:UV-diff}f).


\begin{figure}

\includegraphics{thesis_files/figure-latex/UV-diff-1} \hfill{}

\caption{Diferencia entre la media del ensamble de los análisis a) y d) AWS-CONV, b) y e) SATWND-AWS, y c) y f) RAD-SATWND para los perfiles verticales espacialmente promediados del viento zonal (a, b y c, en K) y viento meridional (d, e y f en \(gkg^{-1}\)) calculados sobre el dominio interior (recuadro rojo en la Figura \ref{fig:dominio}a) para cada ciclo de análisis. Los contornos negros corresponden al viento zonal y meridional para (a,d) AWS, (b,e) SATWND, y (c,f) RAD ya que son los experimentos tienen más observaciones asimiladas en cada panel.}\label{fig:UV-diff}
\end{figure}
La diferencia entre ERA5 (Hersbach et al., 2018) y la media del ensamble del análisis para cada experimento se compara en la Figura \ref{fig:era5}, apoyando los resultados observados en las Figuras \ref{fig:TQ-diff} y \ref{fig:UV-diff}. En concreto, la Figura \ref{fig:era5}a muestra un bias cálido en los niveles bajos (es decir, CONV es más cálido que ERA5) que disminuye en la Figura \ref{fig:era5}b cuando se asimilan las observaciones de EMA. En la misma dirección, la Figura \ref{fig:TQ-diff}a muestra una diferencia negativa entre AWS y CONV, lo que significa que las observaciones de EMA están enfriando los niveles bajos de la atmósfera. Comparando ERA5-RAD (Figura \ref{fig:era5}d), hay un pequeño aumento del bias cálido, asociado al calentamiento producido por la asimilación de las observaciones de radianza, como se muestra en la Figura \ref{fig:TQ-diff}c.~Un efecto similar puede observarse en la humedad específica, las observaciones de EMA corrigen parcialmente el bias seco presente en la Figura \ref{fig:era5}e y la asimilación de las observaciones de radianza reduce el impacto positivo de EMA. El impacto sobre las componentes del viento es menor, por lo que sólo se incluye el viento meridional en las figuras \ref{fig:era5}i-l, que muestran que las observaciones de radianza son las principales responsables del impacto positivo observado en el análisis al reducir la distancia ERA5-RAD, en particular durante la fase madura del MCS. En general, los ajustes asociados a la asimilación de las observaciones de radianza y de EMA conducen a un análisis más cercano a los reanálisis ERA5.


\begin{figure}

\includegraphics{thesis_files/figure-latex/era5-1} \hfill{}

\caption{Diferencia entre la media del ensamble del análisis de cada experimento y el ERA5 para los perfiles verticales espacialmente promediados de la temperatura del aire (K, a--d), la humedad específica (\(gKg^{-1}\), e--h) y el viento meridional (\(ms^{-1}\), i--l) calculados sobre el dominio interior (recuadro rojo en la Figura \ref{fig:dominio}a) para cada ciclo de análisis.}\label{fig:era5}
\end{figure}
Para investigar cómo los cambios en la PBL pueden modificar el entorno preconvectivo, se compara la distribución horizontal media de análisis del viento meridional del norte (para los primeros 7 niveles sigma, desde la superficie hasta aproximadamente 800 hPa), el agua precipitable, la temperatura en niveles bajos y el MCAPE. A las 00 UTC del 22 de noviembre (después de 30 ciclos de asimilación) las primeras celdas convectivas se estaban desarrollando sobre la región sur del dominio a lo largo del frente frío. La figura \ref{fig:summary-fields}a muestra el agua precipitable (sombreada) y la componente de viento meridional en niveles bajos promediada verticalmente (contornos). La figura muestra que la región húmeda que se extiende en la zona norte del dominio se ve reforzada por la asimilación de las observaciones de superficie de EMA. El aumento de la humedad es particularmente intenso en el extremo sur de esta región, justo por delante del frente frío donde se produce la iniciación de la convección. Los experimentos AWS y SATWND son muy similares, con valores de agua precipitable superiores a 55 \(kgm^{-2}\) al norte de 30\(^{\circ}\)S y una distribución vertical similar de la humedad específica (no mostrada). RAD tiene un contenido de agua precipitable menor que AWS y SATWND, pero mayor que CONV. La distribución de la humedad en niveles bajos en RAD parece ser el resultado de la combinación del efecto de humedecimiento de la asimilación de EMA -compensado parcialmente por la asimilación de las observaciones de radianza- y un menor transporte meridional de humedad debido al flujo más débil del norte sobre el centro del dominio en comparación con CONV.

La distribución de la temperatura y la humedad en la PBL (Figura \ref{fig:summary-fields}b) se asemeja a las características observadas en los perfiles de temperatura (Figura \ref{fig:TQ-diff}a-c) donde AWS produce una PBL más fría que CONV mientras que la PBL en RAD es más cálida que en SATWND. En promedio, la PBL en AWS y SATWND es más fría que en CONV, mientras que RAD muestra una PBL más cálida que AWS debido a la asimilación de radianzas. Una PBL más cálida aumenta la inestabilidad convectiva y contribuye a reducir la energía inhibitoria para la convección. La figura \ref{fig:summary-fields}c muestra la energía potencial convectiva disponible para el nivel (o parcela) más inestable (MCAPE, sombreada) y la cortante del viento entre 0 a 6 \(km\). Los valores de MCAPE en CONV no superan los 2000 \(Jkg^{-1}\) mientras que en el resto de los experimentos el MCAPE máximo supera los 4000 \(Jkg^{-1}\). El MCAPE en el experimento RAD es menor en comparación con AWS o SATWND. Esto es consistente con una menor humedad en la PBL con respecto a estos experimentos pero puede ser parcialmente compensado por una PBL ligeramente más cálida en el experimento RAD. La cortante del viento es más intensa en AWS, SATWND y RAD, alcanzando valores superiores a 15 \(ms^{-1}\) en el extremo sur de la región con valores positivos de MCAPE. Además, en esta misma región, estos experimentos muestran valores de MCAPE mayores que CONV. Cortantes de viento superior a 15 \(ms^{-1}\) están asociadas al desarrollo de SCMs más intensos y organizados (Chen et al., 2015) y también a condiciones favorables para el desarrollo de superceldas (Markowski and Richardson, 2010).


\begin{figure}
\includegraphics[width=1\linewidth,]{thesis_files/figure-latex/summary-fields-1} \caption{a) Agua precipitable (sombreada, \(kgm^{-2}\)) y componte meridional del viento promediada sobre los primeros los primeros 7 niveles sigma (desde la superficie hasta aproximadamente 800 hPa, contornos, \(ms^{-1}\)), b) Temperatura potencial media para la PBL (primeros 10 niveles sigma), y c) CAPE máximo y cortante del viento en \textasciitilde0-6 km para 15 y 30 \(ms^{-1}\) en cada experimento. Todos los campos corresponden a la media del ensamble del análisis para las 00 UTC del 22 de noviembre. Los contornos rellenos de color gris corresponden a la topografía de más de 1500 metros sobre el nivel del mar.}\label{fig:summary-fields}
\end{figure}
\hypertarget{val-analisis}{%
\subsection{Validación con observaciones independientes}\label{val-analisis}}

En primer lugar, se analizó el impacto de la asimilación de diferentes tipos de observación en la representación del SCM y su precipitación asociada. La figura \ref{fig:pp-hov}a muestra la precipitación acumulada horaria estimada por IMERG, y la media calibrada mediante una técnica de ajuste de la función de distribución de probabilidad (PM, Clark, 2017) para la precipitación acumulada horaria del campo preliminar promediada entre 67\(^{\circ}\)W y 54,5\(^{\circ}\)W en función del tiempo y la latitud en los diferentes experimentos. Las precipitaciones más intensas comienzan durante la tarde del 22 de noviembre y continúan durante el 23 de noviembre posterior al periodo simulado (Figura \ref{fig:pp-hov}a). En todos los experimentos, la precipitación acumulada en los pronósticos de corto plazo es subestimada. Esto es particularmente evidente en CONV (Figura \ref{fig:pp-hov}b), donde el inicio de la convección se retrasa y ocurre más al norte con respecto al inicio observado. AWS, SATWND y RAD captan mejor el momento y la ubicación de la iniciación de la convección (Figuras \ref{fig:pp-hov}c-e). AWS y RAD muestran una distribución más fragmentada en comparación con SATWND, posiblemente debido al desarrollo de una convección menos organizada durante el 22 de noviembre. Después de las 18 UTC del 22 de noviembre, RAD muestra mejoras en la tasa de precipitación y su distribución en comparación con los otros experimentos como resultado de un desarrollo de la convección más intenso.


\begin{figure}

{\centering \includegraphics{thesis_files/figure-latex/pp-hov-1} 

}

\caption{Diagrama de Hövmoller de la media ajustada a la probabilidad de la precipitación acumulada horaria para cada banda de latitud estimada por IMERG (a) y simulada (b-e) por cada experimento, promediada en un rango de longitudes entre 67\(^{\circ}\)W y 54,5\(^{\circ}\)W. Los contornos se dibujan cada 0,5 \(mm^{-1}\), comenzando en 0,5 \(mm^{-1}\).}\label{fig:pp-hov}
\end{figure}
El FSS se calcula para cuantificar la coincidencia espacial en distintas escalas entre la precipitación observada y la precipitación acumulada horaria simulada por el campo preliminar de los diferentes experimentos (Figura \ref{fig:fss}). Para cada umbral y escala espacial, se aplica la ecuación \eqref{eq:eq11} en ventanas móviles de 6 horas a lo largo del periodo del experimento. Todos los experimentos muestran valores similares de FSS durante el inicio de la convección antes de las 06 UTC del 22 de noviembre, excepto RAD, que obtiene mejores resultados que el resto de los experimentos durante este periodo. Esto indica que las observaciones de radianza tienen un impacto positivo en el análisis. El FSS de CONV es el más bajo en comparación con el resto de los experimentos y las diferencias son mayores durante la fase madura del SCM. AWS y SATWND muestran FSSs similares, lo que indica que la asimilación del viento estimado por satélites tiene poco impacto en la representación de la precipitación para este caso de estudio. La asimilación de radianzas conduce a una mejora general de la representación de la precipitación pronosticada a una hora, sobre todo para el umbral de 25 mm durante el periodo de mayor precipitación durante 22 de noviembre (Figura \ref{fig:fss}b,d). La mejora también es importante en la fase de desarrollo del SCM (entre las 00 y las 12 UTC del 22 de noviembre y también para escalas espaciales superiores a 500 km, no mostradas).


\begin{figure}

{\centering \includegraphics{thesis_files/figure-latex/fss-1} 

}

\caption{FSS calculado sobre la precipitación acumulada a 1 hora en una ventana móvil de 6 horas para umbrales de 1 mm (a y c) y 25 mm (b y d), en escalas de 10 km (a y b) y 100 km (c y d), para el campo preliminar de los experimentos CONV (línea azul), AWS (línea celeste), SATWND (línea naranja) y RAD (línea roja).}\label{fig:fss}
\end{figure}
Para complementar el análisis, la Figura \ref{fig:dbz-mean} muestra la reflectividad máxima observada en la columna vertical (COLMAX) y el COLMAX de la media del ensamble para los experimentos CONV y RAD en diferentes momentos entre las 10 y las 19 UTC del 22 de noviembre. Estos experimentos fueron elegidos porque representan el análisis con el mínimo (CONV) y el máximo (RAD) número de observaciones asimiladas. Además, son los experimentos con peor (CONV) y mejor (RAD) desempeño en cuanto a la habilidad para pronosticar la precipitación a una hora (Figura \ref{fig:fss}). En general, ninguno de los pronósticos de corto alcance capta los detalles finos de la distribución de la reflectividad. Esto es esperable si se tiene en cuenta la resolución horizontal del modelo (10 km), que no es suficiente para representar adecuadamente la intensidad de la banda convectiva asociada al SCM. RAD representa mejor las características observadas del sistema mostrando un SCM más fuerte y organizado que CONV, sobre el centro del dominio a las 10 y 13 UTC (primera y segunda columnas en la Figura \ref{fig:dbz-mean}). Las celdas convectivas que se inician después de las 16 UTC a lo largo del frente cálido en la parte noreste del dominio son bien captadas por ambos experimentos, pero están mejor representadas en términos de intensidad en RAD. Además, CONV capta la ubicación del SCM, pero la convección parece estar menos organizada y ser mucho más débil que en RAD. Antes y después de los tiempos mostrados en la Figura \ref{fig:dbz-mean}, la concordancia entre la localización de las celdas convectivas observadas y las simuladas en los experimentos es bastante buena en las regiones donde se dispone de datos de radar, especialmente para RAD.

Finalmente, la Figura \ref{fig:soundings} muestra el RMSE y el bias calculados comparando los experimentos con los datos de radiosondeos durante las misiones de observación intensiva del experimento RELAMPAGO, IOP 7 del 15 al 21 UTC de noviembre (incluyendo 30 radiosondeos), e IOP 8 del 14 al 20 UTC de noviembre (incluyendo 22 radiosondeos).

El IOP 7 (Figuras \ref{fig:soundings}a-d) proporciona una buena caracterización del entorno preconvectivo durante el primer día de nuestros experimentos. La zona donde se realizaron las observaciones se caracterizó por cielos mayoritariamente despejados y un flujo de norte en niveles bajos asociado a una advección cálida y húmeda. En general, los experimentos muestran un RMSE y un bias similares para todas las variables. Las observaciones de EMA lograron reducir el RMSE para la temperatura y la temperatura del punto de rocío y un bias seco en la PBL. Sin embargo, en esta región (Figura \ref{fig:dominio}b) y para este periodo, los incrementos generados por observaciones de EMA en el análisis (Figura \ref{fig:UV-diff}d) degradan el viento zonal entre 7 y 12 km aumentando el bias y el RMSE (Figura \ref{fig:soundings}c).


\begin{figure}
\includegraphics[width=1\linewidth,]{thesis_files/figure-latex/dbz-mean-1} \caption{Reflectividad máxima en la columna (COLMAX en \(dBZ\)), observada (fila superior) y pronosticada a 1 hora por los experimentos CONV (segunda fila) y RAD (tercera fila) a las 10 UTC (primera columna), 13 UTC (segunda columna), 16 UTC (tercera columna) y 19 UTC (cuarta columna) del 22 de noviembre de 2018. Los círculos negros en la primera fila muestran el rango de observación de cada radar.}\label{fig:dbz-mean}
\end{figure}
Durante el IOP 8 (Figuras \ref{fig:soundings}e-h), la zona densamente observada se encontraba detrás del SCM, lo suficientemente alejada del sistema como para no estar directamente afectada por su circulación de mesoescala. Esta zona también estaba detrás del frente frío y se veía afectada por advección fría en niveles bajos. La asimilación de observaciones en AWS, SATWND y RAD reduce el bias frío y el RMSE para la temperatura entre 5 y 12 km y el RMSE en la PBL en comparación con CONV (Figura \ref{fig:soundings}e). La reducción del bias y del RMSE también es importante para la temperatura del punto de rocío (Figura \ref{fig:soundings}f), siendo SATWND el que muestra el mayor impacto, seguido de AWS y RAD. El viento zonal está sobrestimado en los análisis y sólo RAD muestra una mejora con respecto a CONV en la troposfera alta (Figura \ref{fig:soundings}g). En niveles bajos el viento meridional (Figura \ref{fig:soundings}g) presenta un bias negativo, indicando una subestimación del viento del sur detrás del frente frío principalmente en AWS, SATWND, y RAD. De hecho, los bias en niveles bajos en estos experimentos son mayores que en el experimento CONV, lo que indica un efecto negativo de las observaciones asimiladas (posiblemente asociado al efecto de EMA).


\begin{figure}

{\centering \includegraphics{thesis_files/figure-latex/soundings-1} 

}

\caption{RMSE (línea sólida) y BIAS (línea discontinua) de a) la temperatura (\(K\)), b) la temperatura del punto de rocío (\(K\)), c) el viento zonal (\(m\ s^{-1}\)) y d) el viento meridional (\(m\ s^{-1}\)) calculados comparando el análisis de cada experimento con los radiosondeo de RELAMPAGO durante el IOP 7 y el IOP 8. La línea azul corresponde a CONV, la línea celeste a AWS, SATWND se representa con una línea naranja y RAD con una línea roja.}\label{fig:soundings}
\end{figure}
\hypertarget{conclusiones}{%
\section{Conclusiones}\label{conclusiones}}

En este capítulo se investiga el impacto de los diferentes sistemas de observación de interés en el desempeño de un sistema de asimilación de datos regional de mesoescala basado en ensambles. Este caso de estudio corresponde a un SCM de gran extensión que se desarrolló sobre el sur de Sudamérica el 22 de noviembre de 2018 durante la campaña de campo de RELAMPAGO. En particular, se explora el impacto en la calidad del análisis de la asimilación de observaciones superficiales frecuentes y espacialmente densas provenientes de estaciones automáticas, los vientos estimados por satélite y las radianzas de cielo despejado a sensores a bordo de satélites polares.

En primer lugar, se evaluó la consistencia del ensamble para analizar la concordancia entre la dispersión del ensamble y los errores de observación con respecto a la distancia entre la media del ensamble y las observaciones. Mientras que las desviaciones de las observaciones convencionales son coherentes con la dispersión del ensamble y los errores de observación asumidos, las desviaciones de las observaciones de viento estimado por satélite y radianzas son menores de lo esperado. Esto último podría ser el resultado de una sobreestimación de los errores de observación que se suele introducir para evitar el impacto perjudicial en el análisis de las observaciones de baja calidad.

En general todos los tipos de observación considerados (es decir, las estaciones meteorológicas automáticas, los vientos estimados por satélites y las radianzas de cielo despejado procedentes de los satélites de órbita polar) mejoran la calidad del análisis y del pronóstico a corto plazo de precipitación cuando se comparan los resultados obtenidos al asimilar solo las observaciones de la red de observación convencional. En cuanto al análisis, las observaciones de las estaciones meteorológicas automáticas, que tienen una alta resolución espacial y temporal, produjeron impactos principalmente dentro de la PBL, pero que ocasionalmente se extienden por toda la troposfera durante los períodos en los que la convección húmeda es más intensa dentro del dominio. Estas observaciones también ayudaron a reducir el bias cálido y seco presente en el modelo, produciendo un análisis más cercano a ERA5. Durante el periodo preconvectivo, la asimilación de la temperatura de superficie, la temperatura del punto de rocío y el viento meridional mejoró el análisis en niveles bajos en comparación con los radiosondeo de la campaña RELAMPAGO. En particular, cuando se asimilan estas observaciones, el contenido de agua precipitable y la circulación meridional en niveles bajos condujeron a un aumento de la convección profunda y de la precipitación intensa acercando el análisis a lo observado.

También se encontraron resultados positivos al asimilar las observaciones de radianza, que produjeron un mejor desarrollo de la convección y de su circulación de salida asociada, principalmente durante la fase madura del SCM, lo que condujo a un aumento de la precipitación acumulada en comparación con el caso en el que no se asimilan estas observaciones. Sin embargo, estas observaciones debilitaron el impacto de las observaciones de las estaciones meteorológicas automáticas dentro de la PBL, aumentando ligeramente el bias cálido y seco del modelo. Aunque esto debe estudiarse más a fondo, podría estar relacionado con un desarrollo diferencial de la nubosidad producido por la incorporación de estas observaciones o la asimilación de canales afectados por la superficie o con una corrección de bias de las observaciones subóptima. Comparando el experimento con radiosondeo independientes, la asimilación de radianzas mejoró el viento en niveles medios y altos.

La asimilación del viento estimado por satélite no produjo un impacto notable en el análisis. Esto se debe posiblemente al número relativamente pequeño de observaciones por debajo de 600 hPa disponibles para este caso de estudio y gran error de observación asignado durante la asimilación. Sin embargo, se observan mejoras en la distribución de la precipitación acumulada en el pronóstico a 1 hora. Es necesario realizar un análisis más exhaustivo para comprender los mecanismos que subyacen al impacto de estas observaciones en los pronósticos de mayor alcance, que será tema de análisis en el Capítulo \ref{cap-4-pronosticos}.

\hypertarget{cap-4-pronosticos}{%
\chapter{Impacto de la asimilación de diferentes fuentes en los pronosticos a corto plazo}\label{cap-4-pronosticos}}

En este capítulo se evalúa el impacto de la asimilación de distintas fuentes de observación en pronósticos por ensable a corto plazo utilizando como condiciones iniciales los análisis de los distintos experimentos descriptos en la Sección \ref{config}. Estos pronósticos abarcan el periodo de iniciación, madurez y desarrollo del SCM observado durante el 22 y 23 de noviembre de 2022. Otros autores estudiaron los pronósticos generados a partir de sistemas de asimilación regionales, Jones et al. (2014) por ejemplo, evaluaron pronósticos de eventos de precipitación utilizando condiciones iniciales generadas a partir de la asimilación de observaciones de radar y radianzas infrarrojas sintéticas. Encontraron que la combinación de estas fuentes de observación generaban el mayor impacto en comparación con la asimilación de la observaciones por separado y en particular, que las observaciones del canal 6.95 \(\mu m\) mejoraban la representación de la humedad en niveles medios. Sin embargo el impacto en los pronósticos no perduraba más allá de las 3 horas. Por otro lado, Bao et al. (2015) estudiaron el impacto de la asimilación de observaciones convencionales y radianzas de satélites polares con sensores de microondas e infrarrojo sobre la temperatura y humedad pronosticadas para distintas capas de la atmósfera sobre un dominio regional en Estados Unidos. Observaron una disminución del RMSE de la temperatura en la tropósfera alta y la estratósfera baja gracias a observaciones de microondas pero un impacto negativo asociados a las observaciones en el espectro infrarrojo. El bias asociado a la humedad disminuyó con la asimilación de observaciones satelitales, en particular de infrarrojo en toda la tropósfera.

En este capítulo además se utilizó el ensamble multifísica, que permite mejorar la representación del aporte de los errores de modelo a la incertidumbre del pronóstico dentro del sistema de asimilación de datos, para estudiar el desempeño de las parametrizaciones utilizadas. En muchos trabajos donde se utiliza un ensamble multifísica, no se hace un análisis por parametrizaciones ya que los resultados pueden depender fuertemente del caso de estudio, de la región o de los fenómenos involucrados. Un ejemplo local es Dillon et al. (2021) donde aplicaron la misma combinación de parametrizaciones que usan los experimentos de este trabajo y encontraron que las combinaciones de parametrizaciones que mejor representan las variables en superficie como temperatura y humedad a 2 metros eran distintas de las combinaciones que lograban el mejor pronóstico de precipitación.

En este contexto, en el capítulo se analizará el impacto de la asimilación de observaciones convencionales, de
estaciones automáticas, vientos derivados de satélite y radianzas de satélites polares en pronósticos a corto plazo en un contexto de convección húmeda profunda y se evaluará si las mejoras observadas en los análisis al representar este caso de estudio se trasladan a los pronósticos. En particular se hará foco en la representación de la temperatura y humedad en la tropósfera para evaluar el entorno en el cual se desarrolla la convección y en el viento meridional para analizar el desplazamiento del frente frío asociado al SCM. Para evaluar la intensidad del SCM estudiaremos la precipitación generada por los pronósticos evaluando tanto la ubicación como su intensidad. Finalmente, un análisis por conjuntos de parametrizaciones nos permitirá establecer cuales tienen un rol más importante en la representación de estos fenómenos.

\hypertarget{metodologuxeda-1}{%
\section{Metodología}\label{metodologuxeda-1}}

\hypertarget{configuraciuxf3n-de-los-experimentos}{%
\subsection{Configuración de los experimentos}\label{configuraciuxf3n-de-los-experimentos}}

Para estudiar el impacto de la asimilación de observaciones en pronósticos independientes, se generaron pronósticos por ensambles inicializados a partir de los análisis de los experimentos previamente descriptos. Estos pronósticos toman el nombre del análisis que utilizan como condición inicial, por ejemplo el pronóstico CONV será aquel que fue incializado a partir del análisis del experimento CONV. Todos los pronósticos utilizan la misma configuración de dominio y de ensamble multifísica que los análisis. Las condiciones de borde de los miembros del ensamble se generaron añadiendo perturbaciones aleatorias al pronóstico determinístico del GFS (GFSF, 0.25\(^{\circ}\) de resolución horizontal y resolución temporal de 6 horas; National Centers for Environmental Prediction, National Weather Service, NOAA, U.S. Department of Commerce, 2015) como se muestra en la Figura \ref{fig:cycle-fcst}. Estas perturbaciones son las mismas que se utilizaron para la generación de las condiciones de borde de los análisis.

Los pronósticos se inicializaron a las 00 y 06 UTC del 22 de noviembre para capturar la inicialización y desarrollo del SCM y además evaluar las posibles diferencias entre los experimentos debido a la asimilación de nuevas observaciones entre las 00 y las 06 UTC. Todos los pronósticos finalizan a las 12 UTC del 23 de noviembre cuando el SCM se encuentra en el borde noroeste del dominio (Figura \ref{fig:caso}f).


\begin{figure}

{\centering \includegraphics[width=0.8\linewidth,]{figure/forecast_diag} 

}

\caption{Diagrama de los pronósticos por ensambles inicializados a las 00 y 06 UTC del 22 de noviembre y que se extienden hasta las 12 UTC del 23 de noviembre. Las condiciones de borde corresponden al pronóstico de GFS y se incorporan cada 6 horas.}\label{fig:cycle-fcst}
\end{figure}
\hypertarget{resultados-1}{%
\section{Resultados}\label{resultados-1}}

\hypertarget{prono-impacto}{%
\subsection{Impacto de las observaciones en el pronostico}\label{prono-impacto}}

Al igual que en la Sección \ref{impacto-analisis}, se compararon los pronósticos con ERA5 sobre la región definida en la Figura \ref{fig:dominio}. La diferencia entre los perfiles de temperatura de la media del ensamble promediados espacialmente y ERA5 a lo largo del tiempo se muestran en la Figura \ref{fig:era5-fcst-1}. En lineas generales la temperatura por encima de 750 hPa es subestimada por los pronósticos y esto ocurre independientemente de las condiciones iniciales que se utilicen. Sin embargo, y al igual que en los análisis, se observa una mejor representación de la temperatura por encima de 750 hPa cuando las condiciones iniciales incluyen observaciones de EMA. En niveles bajos se observa el mismo mismo patrón que en los análisis, los pronósticos inicializados a partir de AWS y SATWND muestran un bias cálido mucho menor que CONV gracias a las condiciones iniciales que incluyen la asimilación de observaciones de EMA.




\begin{figure}[ht]

{\centering \includegraphics{thesis_files/figure-latex/era5-fcst-1} 

}

\caption{Diferencia entre la media del ensamble del pronóstico inicializado a partir de cada experimento y ERA5 de los perfiles verticales promediados espacialmente de la temperatura del aire (\(K\)) calculado sobre el dominio de validación (cuadro rojo en la Figura \ref{fig:dominio}a) para cada tiempo de pronóstico.}\label{fig:era5-fcst-1}
\end{figure}
\begin{figure}[ht]

{\centering \includegraphics{thesis_files/figure-latex/era5-fcst-2} 

}

\caption{Como en la Figura \ref{fig:era5-fcst-1} pero para la humedad específica (\(gkg^{-1}\)).}\label{fig:era5-fcst-2}
\end{figure}
En RAD el bias cálido vuelve a aumentar ligeramente manteniendo el patrón observado en los análisis de RAD (Figuras \ref{fig:era5-fcst-1}d y h). Es importante notar que el impacto que genera la asimilación de las distintas fuentes de observación persiste en los pronósticos, aunque la persistencia del bias cálido en la capa límite podría indicar que los errores en el modelo contribuyen a este error. En la capa límite sí se ven diferencias entre las dos inicializaciones, en particular se observan mejoras para la inicialización de las 06 UTC en los casos de AWS y SATWND, indicando que las observaciones asimiladas entre las 00 y las 06 UTC tienen un impacto positivo en la representación de la temperatura cerca de superficie que se traslada a los pronósticos.

Las diferencias entre los pronósticos y ERA5 para los perfiles promediados de humedad específica (Figuras \ref{fig:era5-fcst-2}) muestran nuevamente el bias seco presente en el modelo y que es más importante en horas nocturnas. Sin embargo es notaria la mejora en los pronósticos inicializados a partir de AWS y SATWND y RAD, en comparación con CONV. Los pronósticos inicializados a partir de CONV mantienen un bias seco muy marcado a lo largo de todo el periodo y que nuevamente puede estar asociado con errores sistemáticos en el modelo (Figuras \ref{fig:era5-fcst-2}a y e). Las dos inicialización, en este caso, no muestran diferencias apreciables.




\begin{figure}[ht]

{\centering \includegraphics{thesis_files/figure-latex/era5-fcst-uv-1} 

}

\caption{Como en la Figura \ref{fig:era5-fcst-1} pero para el viento meridional (\(ms^{-1}\)).}\label{fig:era5-fcst-uv-1}
\end{figure}
\begin{figure}[ht]

{\centering \includegraphics{thesis_files/figure-latex/era5-fcst-uv-2} 

}

\caption{Como en la Figura \ref{fig:era5-fcst-1} pero para el viento zonal (\(ms^{-1}\)).}\label{fig:era5-fcst-uv-2}
\end{figure}
Es interesante notar que la asimilación de radianzas que aportan información en niveles medios y altos, también produce un impacto en niveles bajos de la atmósfera. Es posible que el desarrollo nuboso que se ve influenciado por cambios de temperatura y humedad genere también cambios en la temperatura en superficie. En este caso las mayores temperaturas observadas en los pronósticos de RAD se pueden explicar por un aumento de la radiación de onda larga saliente, posiblemente asociado a una disminución en el desarrollo de nubes (no se muestra).

La mayor diferencia en los perfiles promediados para el viento meridional se observa en niveles altos y está asociado al periodo de mayor actividad convectiva sobre el norte del dominio (Figuras \ref{fig:era5-fcst-uv-1}). En particular SATWND es el pronóstico que mejor representa el viento meridional, particularmente en el pronóstico inicializado a las 06 UTC (Figura \ref{fig:era5-fcst-uv-1}g). Esto es particularmente interesante ya que el impacto de las observaciones de viento derivadas de satélite asimiladas en el experimento SATWND era marginal comparado al impacto de otra fuentes de observaciones. Sin embargo los pronósticos inicializados con condiciones iniciales que incluye información de viento derivado de satélite representan mejor el viento meridional aún en los periodos de mayor convección. Si se analizan los experimentos individualmente alrededor de las 00 UTC del 23 de noviembre, se observa que SATWND presenta mayor convergencia en niveles bajos y mayor divergencia en niveles altos cuando se lo compara con RAD (no se muestra). Esto podría indica que la convección en SATWND es más intensa.

Algo similar se observa en la diferencia entre perfiles verticales del viento zonal en las Figuras \ref{fig:era5-fcst-uv-2}. Los pronósticos inicializados a partir de los experimentos AWS, SATWND y RAD muestran una mejor representación del viento zonal en todos los niveles. Sin embargo las observaciones de satélites polares parecen generar un impacto negativo en las condiciones iniciales de RAD que se traslada al pronóstico aumentando la diferencia respecto de ERA5. En todos los casos los pronósticos subestiman el viento zonal cerca de superficie luego del paso del frente frío, esto podría deberse a un retraso en el avance del frente o una disminución en la intensidad del viento postfrontal. Al comparar la ubicación del frente en los pronósticos y en ERA5, aproximado por la isolínea de viento meridional igual a 0 \(ms^{-1}\) en las Figuras \ref{fig:frente}, se observa que la ubicación del frente es similar en todos los casos y solo se retrasa en los pronósticos respecto de ERA5 hacia el final del periodo y por encima de 900 hPa (Figuras \ref{fig:frente}c y f). Por lo anterior, las diferencias entre los pronósticos y ERA5 para el viento zonal parecen deberse principalmente a una disminución en el flujo postfrontal.


\begin{figure}
\includegraphics{thesis_files/figure-latex/frente-1} \caption{Viento meridional igual a 0 \(ms^{-1}\) en 800 (a-c), 900 (d-f) y 1000 hPa (g-i) en los pronósticos que verifican a las 12 (a,d,g) del 22 de noviembre y las 00 (b,e,h) y 12 UTC (c,f,i) del 23 de noviembre. Se muestra ERA5 (línea negra) y la media del ensamble de los experimentos CONV (línea azul), AWS (línea celeste), SATWND (línea naranja) y RAD (línea roja).}\label{fig:frente}
\end{figure}
Finalmente, para evaluar si las condiciones iniciales de los pronósticos perduran en el tiempo se compararon los experimentos CONV y RAD por ser los que tienen la menor y mayor cantidad de observaciones asimiladas respectivamente. Para realizar la comparación se calculó la raíz de la diferencia cuadrática media (RDCM) entre RAD y CONV a lo largo del periodo pronosticado para temperatura y humedad específica. La Figura \ref{fig:ci} muestra campos de esta diferencia en distintos tiempos de pronóstico. Para ambas variables los campos de RDCM iniciales muestran los valores más altos y las diferencias se encuentran distribuidas en todo el dominio (Figura \ref{fig:ci}a y e). Esto muestra que las condiciones iniciales de RAD y CONV tienen características muy distintas, y particularmente difieren en el contenido de humedad debido a la asimilación previa de observaciones de EMA. Aún más interesante es que estas diferencias iniciales perduran a lo largo del todo el pronóstico hasta las 12 UTC del 23 de noviembre. Si bien estas diferencias disminuyen en magnitud, muestran que el impacto positivo de la asimilación de observaciones de EMA, vientos derivados de satélite y radianzas se traslada a los pronósticos y se mantienen en el tiempo. Por la ubicación de las diferencias a lo largo del tiempo, es posible que los procesos de mesoescala y escala convectiva tengan una contribución importante. Será importante estudiar en el futuro las posibles causas que llevan al decaimiento de las diferencias en otras regiones dominio, particularmente en variables como la humedad.


\begin{figure}

\includegraphics{thesis_files/figure-latex/ci-1} \hfill{}

\caption{Raíz de la diferencia cuadrática media (RDCM) para la humedad específica (\(gkg^{-1}\), a-d) y temperatura (K, e-h) calculada sobre la diferencia entre RAD y CONV sobre todos los niveles del modelo para distintos tiempos a lo largo del pronóstico inicializado a las 00 UTC del 22 de noviembre.}\label{fig:ci}
\end{figure}
\hypertarget{verificacion-de-los-pronuxf3sticos-de-precipitaciuxf3n}{%
\subsection{Verificacion de los pronósticos de precipitación}\label{verificacion-de-los-pronuxf3sticos-de-precipitaciuxf3n}}

Para cuantificar la habilidad de los pronósticos en distintos plazos al representar la precipitación en la Figura \ref{fig:fssfcst} se calculó el FSS a partir de los pronósticos por ensambles en ventanas móviles de 6 horas para los mismos umbrales y escalas espaciales usados para la precipitación acumulada horaria del campo preliminar en la Sección \ref{val-analisis}. Los pronósticos de CONV generan valores muy bajos de FSS en comparación con los experimentos que incluyen otras fuentes de observaciones, lo que habla de una subestimación de la precipitación en todas las escalas. AWS, SATWND y RAD muestran mejoras en los valores de FSS, especialmente para el umbral de 25 mm (Figura \ref{fig:fssfcst}b, d). Además, la inicialización de las 06 UTC muestra mejores resultados que los pronósticos inicializados a las 00 UTC para todos los experimentos, lo que pone de relieve el impacto positivo de las observaciones asimiladas entre las 00 y las 06 UTC en la representación de la precipitación, como así también la incorporación de condiciones de borde provenientes del GFSF.

A diferencia de lo que se observó al comparar el pronóstico a 1 hora con observaciones independientes en la verificación de los análisis, las observaciones de viento derivadas de satélite muestran un impacto claramente positivo en los pronósticos. Esto evidencia el desarrollo de convección más intensa y es coherente con lo analizado en la Sección \ref{prono-impacto}. Por el contrario, las radianzas tuvieron un impacto neutro a ligeramente negativo en los pronósticos inicializados a las 00 y 06 UTC particularmente en los umbrales altos de precipitación. Esto puede deberse en parte a un menor MCAPE presente en las condiciones iniciales de RAD al compararlo con SATWND. Además, SATWND muestra un MCAPE más intenso que RAD durante las primeras 12 horas de pronóstico, lo que muestra que las condiciones iniciales de los distintos experimentos se mantienen a lo largo del periodo pronosticado (no se muestra).

Existen muchos posibles motivos por los que los pronósticos inicializados a partir de RAD se degradan con el tiempo y en consecuencia generan menor precipitación. Por ejemplo, Lim et al. (2014) observó un impacto limitado al asimilar las observaciones de AIRS y atribuye este resultado al uso de canales superficiales en los que las incertidumbres asociadas a la emisividad de la superficie son grandes. Teniendo en cuenta la cantidad de radianzas asimiladas cerca de superficie (Figura \ref{fig:obs-cycle}b), es posible que la asimilación de observaciones asociadas a canales afectados por la superficie contribuya a la degradación de la capa límite en el análisis y, posteriormente, en los pronósticos. Sin embargo es necesario realizar nuevos experimentos para evaluar esto en profundidad y definir que canales deberían ser asimilados.


\begin{figure}
\includegraphics{thesis_files/figure-latex/fssfcst-1} \caption{FSS calculado sobre la precipitación acumulada a 1 hora en una ventana móvil de 6 horas para umbrales de 1 mm (a y c) y 25 mm (b y d), en escalas de 10 km (a y b) y 100 km (c y d), para los pronósticos inicializados a partir de los experimentos CONV (línea azul), AWS (línea celeste), SATWND (línea naranja) y RAD (línea roja) a las 00 UTC (línea sólida) y 06 UTC (linea punteada) del 22 de noviembre.}\label{fig:fssfcst}
\end{figure}
El porcentaje de área cubierta por precipitación superior a determinados umbrales se muestra en la Figura \ref{fig:area} y permite analizar la distribución espacial de la lluvia generada por los pronósticos. Nuevamente, los experimentos muestran una importante subestimación de la precipitación media, en particular CONV. El área sombreada que muestra el área máxima estimada sobre todos los miembros del ensamble se acerca al área estimada por IMERG, sin embargo todos los experimentos y sobre todo en el umbral de 1 \(mmh^{-1}\), la precipitación se adelantan en el tiempo. En este sentido, RAD es el experimento que mejor representa la precipitación respecto de su inicialización. Los pronósticos inicializados a partir de CONV subestiman el área cubierta por precipitación durante el periodo más activo y luego la sobreestiman durante el 23 de noviembre cuando ya no se observa precipitación intensa dentro del dominio de estudio.


\begin{figure}
\includegraphics{thesis_files/figure-latex/area-1} \caption{Porcentaje de área cubierta por precipitación superior a 1 \(mmh^{-1}\) (a y c) y 5 \(mmh^{-1}\) (b y d) a lo largo del tiempo para los pronósticos inicializados a partir de los experimentos CONV (línea azul), AWS (línea celeste), SATWND (línea naranja) y RAD (línea roja) a las 00 UTC (línea sólida) y 06 UTC (linea punteada) del 22 de noviembre. En sombreado y para cada experimento se muestra el área máxima y mínima estimada por el ensamble de pronósticos. En línea continua negra se muestra la estimación de IMERG.}\label{fig:area}
\end{figure}
Los pronósticos por ensambles permiten obtener información probabilística al calcular, por ejemplo, la proporción de miembros del ensamble que pronostican precipitación por encima de 1 mm en un determinado lugar. Para evaluar el pronóstico de precipitación desde el punto de vista probabilístico, se generaron diagramas de confiabilidad usando probabilidades pronosticadas de 0, 20, 40, 60, 80 y 100\% calculadas sobre el ensamble para los pronósticos inicializados a las 00 UTC (Figuras \ref{fig:reliability-1}) y 06 UTC (Figuras \ref{fig:reliability-2}) utilizando umbrales de precipitación de 1, 5 y 10 mm. En un diagrama de confiabilidad, la línea diagonal representa un pronóstico totalmente confiable. Por el contrario, pronósticos por debajo de la diagonal sobreestiman la probabilidad ocurrencia del evento observado y pronósticos por encima de la diagonal, subestiman la probabilidad ocurrencia del evento. Debido a que los pronósticos generados subestiman fuertemente la precipitación, se incluye además la frecuencia relativa de eventos asociados a cada intervalo de probabilidad donde se observa este bias negativo.




\begin{figure}

{\centering \includegraphics{thesis_files/figure-latex/reliability-1} 

}

\caption{Diagrama de confiabilidad calculado sobre el pronóstico inicializado a las 00 UTC para probabilidades de 0, 20, 40, 60, 80 y 100\% de ocurrencia de precipitación acumulada en 3 horas mayor al umbral a) 1 mm, b) 5 mm y c) 10. Los gráficos secundarios dentro de cada diagrama muestran la frecuencia relativa de los diferentes valores de probabilidad pronosticada (x 1e5) asociada a cada intervalo de probabilidad para cada percentil.}\label{fig:reliability-1}
\end{figure}
\begin{figure}

{\centering \includegraphics{thesis_files/figure-latex/reliability-2} 

}

\caption{Como en la Figura \ref{fig:reliability-1} pero para el pronóstico inicializado a las 06 UTC}\label{fig:reliability-2}
\end{figure}
Para el umbral más bajo de 1 mm, los dos pronósticos muestran curvas de confiabilidad con tendencia positiva, por lo que son considerados confiables (Figuras \ref{fig:reliability-1}a y \ref{fig:reliability-2}a). Sin embargo, para probabilidades de precipitación hasta 40\%, los pronósticos subestiman la ocurrencia de precipitación. Esta subestimación aumenta a medida que aumenta el umbral, particularmente para 10 mm (Figuras \ref{fig:reliability-2}b), lo que indica nuevamente que los pronósticos subestiman la ocurrencia de precipitación muy intensa. En probabilidades del ensamble superiores al 40\%, los pronósticos sobrestiman la ocurrencia de precipitación en todos los umbrales pero, en particular las curvas de los umbrales de 5 y 10 mm dejan de tener una tendencia positiva perdiendo confiabilidad.

Los pronósticos inicializados a las 06 UTC (Figuras \ref{fig:reliability-2}) muestran tendencias similares a los pronósticos de las 00 UTC, sin embargo se observan ligeras mejoras, es decir curvas de confiabilidad están más cerca a la diagonal. Es posible que las condiciones iniciales que incluyen nuevas observaciones en conjunto con condiciones de borde actualizadas de GFSF favorezcan un aumento de la precipitación.

En líneas generales los experimentos AWS, SATWND y RAD muestran un comportamiento similar con pequeñas diferencias que aumentan al aumentar el umbral. Por otro lado CONV muestra los peores resultados con una perdida de confiabilidad para pronósticos con probabilidades por encima del 40\% en el caso del umbral de 5 mm y 20\% para el umbral de 10 mm. Esto es consistente con resultados previos donde se ve que subestima el área cubierta por precipitación (Figura \ref{fig:area}) y tiene los valores más bajos de FSS (Figura \ref{fig:fssfcst}).

Para complementar el análisis probabilístico las Figuras \ref{fig:brier} muestran en índice de Brier (BS por su nombre en inglés) calculado para los mismo umbrales y las 2 inicializaciones de pronóstico. Este índice cuantifica el error medio asociado a las probabilidad pronosticadas y por lo tanto valores cercanos a cero indican que el pronóstico será mejor desde este punto de vista. En este caso los resultados son consistentes con lo analizado previamente, CONV es el experimento con mayor BS y solo es mejor para el umbral de 1 mm al comienzo de los pronósticos. Para el resto de los umbrales AWS y SATWND muestra un BS menores que RAD y estas diferencias aumentan con el paso del tiempo.


\begin{figure}

\includegraphics{thesis_files/figure-latex/brier-1} \hfill{}

\caption{Índice de Brier calculado sobre el ensamble de pronósticos inicializados a las 00 UTC y 06 UTC y cada experimento utilizando los mismo umbrales seleccionados en las Figuras \ref{fig:reliability-1} y \ref{fig:reliability-2}.}\label{fig:brier}
\end{figure}
Finalmente las Figuras \ref{fig:prob} muestran la probabilidad de ocurrencia de precipitación acumulada en 36 hs para los pronósticos inicializados a las 00 UTC y 30 hs para los pronósticos inicializados a las 06 UTC; para distintos umbrales de precipitación.

Los campos de probabilidad de precipitación mayor a 10 mm (Figuras \ref{fig:prob}a-h), muestra que la precipitación se genera más hacia el norte y oeste de lo estimado por IMERG (línea negra). Al igual que en los análisis (Figura \ref{fig:pp-hov}), los pronósticos no son capaces de representar correctamente la precipitación en la región sur del dominio. Esto también se observa en la Figura \ref{fig:fssfcst} donde se ve que los valores de FSS van aumentando en las primeras horas de pronóstico, es decir, la precipitación se acerca a lo observado tanto en magnitud como en localización. Sin embargo, es notoria la mejora que se produce en los experimentos que se generaron a partir de condiciones iniciales con asimilación de EMA y vientos derivados de satélite respecto de CONV. Los pronósticos RAD si bien son mejores que CONV, subestiman la intensidad de la precipitación en mayor medida que SATWND.

Las Figuras \ref{fig:prob}i-p muestran los campos de probabilidad de precipitación superior a 50 mm donde se puede apreciar una vez más como los pronósticos subestiman la precipitación, en particular para los umbrales más altos. Al igual que para el umbral de 10 mm, es interesante notar la mejora que se produce entre las inicializaciones. La inicializacion de las 06 UTC logra representar mucho mejor la distribución de precipitación en el dominio que los pronósticos de las 00 UTC. Por ejemplo, el pronóstico SATWND de las 06 UTC (Figura \ref{fig:prob}o) captura casi la totalidad del área de precipitación observada, aunque solo con un 10\% probabilidad.


\begin{figure}

\includegraphics{thesis_files/figure-latex/prob-1} \hfill{}

\caption{Probabilidad de ocurrencia de precipitación acumulada en 36 y 30 hs superior a 10 mm (a-h) y 50 mm (i-p) para los pronósticos inicializados a las 00 y 06 UTC respectivamente. Los contornos negros representan la estimación de IMERG para cada umbral de precipitación.}\label{fig:prob}
\end{figure}
\hypertarget{verificacion-a-partir-de-observaciones-independientes}{%
\subsection{Verificacion a partir de observaciones independientes}\label{verificacion-a-partir-de-observaciones-independientes}}

En esta sección evaluamos los pronósticos de manera cuantitativa comparándolos con radiosondeos de la campaña RELAMPAGO y observaciones de EMA, que si bien son asimiladas en los experimentos de análisis, son observaciones independientes para los pronósticos.

En primer lugar se calculó el RMSE y BIAS respecto de las observaciones de EMA a lo largo del tiempo para cada pronóstico e inicialización. El BIAS y RMSE de la humedad específica (Figuras \ref{fig:rmse-bias-aut}a y e) al comienzo de los pronósticos es coherente con las condiciones iniciales. Por ejemplo el BIAS seco asociado a CONV es mucho más importante que el de los pronósticos inicializados a partir de AWS. También se observan mejoras en la inicialización de las 06 UTC, algo que no era evidente en la comparación con ERA5 (Figura \ref{fig:era5-fcst-2}). Los errores asociados a la temperatura (Figuras \ref{fig:rmse-bias-aut}b y f) son muy dependientes del momento del día, con un aumento del RMSE y del BIAS cálido durante las horas nocturnas. Los pronósticos CONV y RAD para la inicialización de las 06 UTC tienen un mejor desempeño en las primeras horas de pronóstico pero luego se degradan más rápidamente que AWS o SATWND.


\begin{figure}
\includegraphics{thesis_files/figure-latex/rmse-bias-aut-1} \caption{Evolución del RMSE y BIAS de a) humedad específica (\(gkg{^-1}\)), b) temperatura (\(K\)), c) el viento zonal (\(ms^{-1}\)) y d) el viento meridional (\(ms^{-1}\)) calculados comparando los pronósticos de cada experimento con las observaciones de EMA. La línea azul corresponde a CONV, la línea celeste a AWS, SATWND se representa con una línea naranja y RAD con una línea roja. El pronóstico inicializado a las 00 UTC se muestra en línea sólida y el pronóstico de las 06 UTC en línea discontinua.}\label{fig:rmse-bias-aut}
\end{figure}
Mientras que no hay características sobresalientes en los errores del viento zonal, en el viento meridional (Figuras \ref{fig:rmse-bias-aut}d y h), es interesante notar que el RMSE de los pronósticos de las 06 UTC es más bajo que para los pronósticos de las 00 UTC y esta mejora se mantiene por al menos 6 horas. Esto también ocurre en el BIAS aunque en menor medida.

La Figura \ref{fig:soundings-fcst} muestra el RMSE y BIAS calculado comparando los pronósticos con los radiosondeos disponibles durante el periodo pronosticado. En líneas generales no se observan grandes diferencias entre las dos inicializaciones. Si bien los pronósticos de las 06 UTC utilizan condiciones iniciales que incluyen información actualizada del entorno, la mayoría de los radiosondeos utilizados en esta verificación fueron lanzados en un área muy pequeña en el centro del dominio (Figura \ref{fig:dominio}b) que podría no haber sido particularmente influenciada por la asimilación de nuevas observaciones y por lo tanto el RMSE y BIAS asociado a cada inicialización no es muy diferente.

Es interesante comparar en este punto los errores asociados a los pronósticos con los errores observados en el análisis (Figura \ref{fig:soundings}e-h). En general, el rango de valores que toma el RMSE y BIAS para cada variable es similar tanto en los pronósticos como en los análisis, lo que indica que los pronósticos logran representar adecuadamente la situación en esa región del dominio. Además el RMSE y BIAS son, en general, menores en los experimentos inicializados con condiciones iniciales que incluyen observaciones de EMA si se los compara con CONV, lo que habla de una persistencia del impacto de la asimilación en estos pronósticos.

En esta comparación, la característica sobresaliente es la disminución del RMSE y BIAS en niveles bajos en casi todas las variables pero particularmente el viento meridional. El aumento de los errores en los análisis puede atribuirse a la asimilación de observaciones de EMA, que si bien en general producen impactos muy positivos en la representación de la capa límite (comparando la Figura \ref{fig:soundings-fcst} con la Figura \ref{fig:era5}), en esta región limitada parece generar el efecto contrario. En los pronósticos a corto plazo los errores en niveles bajos disminuyen a valores cercanos a los observados en los análisis de CONV.


\begin{figure}

{\centering \includegraphics{thesis_files/figure-latex/soundings-fcst-1} 

}

\caption{RMSE (línea sólida) y BIAS (línea discontinua) de a) la temperatura (\(K\)), b) la temperatura del punto de rocío (\(K\)), c) el viento zonal (\(m\ s^{-1}\)) y d) el viento meridional (\(m\ s^{-1}\)) calculados comparando los pronósticos de cada experimento con los radiosondeos operativos y de RELAMPAGO. La línea azul corresponde a CONV, la línea celeste a AWS, SATWND se representa con una línea naranja y RAD con una línea roja.}\label{fig:soundings-fcst}
\end{figure}
\hypertarget{anuxe1lisis-y-verificaciuxf3n-por-parametrizaciones}{%
\subsection{Análisis y verificación por parametrizaciones}\label{anuxe1lisis-y-verificaciuxf3n-por-parametrizaciones}}

En esta sección analizamos el desempeño de las distintas parametrizaciones de capa límite y convección utilizadas en el ensamble multifísica, comparando los miembros del ensamble con los radiosondeos de RELAMPAGO y las observaciones de EMA.

La Figura \ref{fig:aut-all} muestra el RMSE y BIAS calculado comparando con las observaciones de EMA sobre los miembros del ensamble que utilizan cada una de las parametrizaciones de capa límite. En este caso no se muestran las dos inicializaciones por separado ya que ambos pronósticos presentan patrones muy similares. Al analizar los resultados generados por los miembros que comparten las parametrizaciones de capa límite hay que tener en cuenta que todos usan, además, distintas parametrizaciones de convección, por lo que parte de la sensibilidad a las parametrizaciones observada podría estar parcialmente influenciada por esto. Sin embargo, el análisis de sensibilidad sobre las parametrizaciones de convección sobre variables cercanas a superficies mostró que estas tienen poco impacto en la representación del viento, humedad y temperatura en la capa límite.

Si bien sería deseable encontrar una parametrización de capa límite que represente correctamente todas las variables de interés, estudios previos en la región (por ejemplo Ruiz et al., 2010; Dillon et al., 2021) mostraron que la combinación de parametrizaciones que produzca mejores resultados, puede depender de la variable que se considere para la validación de los pronósticos. Para esta situación, la parametrización MYJ presenta un menor RMSE y BIAS para las variables termodinámicas (humedad relativa y temperatura) mientras que MYNN2 representa mejor el viento. Sin embargo, a excepción del RMSE del viento, los errores tienen una variabilidad pequeña por lo que la elección de una parametrización u otra podría no tener un impacto importante en la representación de la capa límite. Sin embargo, que las parametrizaciones tengan errores asociados similares no necesariamente quiere decir que representen los procesos de pequeña escala de la misma manera. Esto refuerza la necesidad de generar un ensamble multifísica como el utilizado en este trabajo para poder capturar los posibles errores asociados al modelo y la representación de los procesos parametrizados.


\begin{figure}

{\centering \includegraphics{thesis_files/figure-latex/aut-all-1} 

}

\caption{RMSE y BIAS calculados comparando los pronósticos con observaciones de EMA y promediados sobre los miembros del ensamble que utilizan las parametrizaciones de capa límite MYJ (círculo), MYNN2 (triángulo) e YSU (cuadrado) para todos los pronósticos en conjunto.}\label{fig:aut-all}
\end{figure}
Si se analizan los errores asociados a las parametrizaciones de capa límite a lo largo del periodo pronosticado (Figura \ref{fig:aut-rad}), en este caso solo para el experimento RAD, el patrón observado previamente se mantiene. Más importante aún, los errores de ninguna de las parametrizaciones aumenta por sobre el resto, indicando que tienen un comportamiento consistente a lo largo de las 30 y 36 hs de pronóstico.


\begin{figure}

{\centering \includegraphics{thesis_files/figure-latex/aut-rad-1} 

}

\caption{RMSE y BIAS calculados comparando los pronósticos con observaciones de EMA y promediados sobre los miembros del ensamble que utilizan las parametrizaciones de capa límite MYJ (línea llena), MYNN2 (línea a rayas) e YSU (línea punteada) para todos los pronósticos en conjunto a lo largo del tiempo de pronóstico.}\label{fig:aut-rad}
\end{figure}
La Figura \ref{fig:soudings-all} muestra el RMSE y BIAS calculado comparando con los radiosondeos de RELAMPAGO sobre los miembros del ensamble que utilizan cada una de las parametrizaciones de convección. En este caso hacemos la verificación de estas parametrizaciones utilizando observaciones en altura donde estas parametrizaciones juegan un rol importante. Sin embargo es importante notar que las características de la capa límite también juegan un rol importante en el desarrollo de la convección y ambos conjuntos de parametrizaciones están fuertemente relacionados. Para estas comparaciones también se muestran las dos inicializaciones en conjunto ya que ambos pronósticos presentan patrones muy similares. Si bien el comportamiento de las parametrizaciones de convección no es tan homogeneo como si sucedía con las de capa límite, es posible identificar algunos patrones. La parametrización BMJ tiene el menor RMSE tanto para la temperatura como para la temperatura de punto de rocío, mientras que la parametrización con menor RMSE en el viento meridional es KF.


\begin{figure}

{\centering \includegraphics{thesis_files/figure-latex/soudings-all-1} 

}

\caption{RMSE y BIAS calculados comparando los pronósticos con radiosondeos de RELAMPAGO y promediados sobre los miembros del ensamble que utilizan las parametrizaciones de convección BMJ (círculo), GF (triángulo) e KF (cuadrado) para todos los pronósticos en conjunto.}\label{fig:soudings-all}
\end{figure}

\begin{figure}

{\centering \includegraphics{thesis_files/figure-latex/soudings-pbl-1} 

}

\caption{RMSE y BIAS calculados comparando los pronósticos con radiosondeos de RELAMPAGO y promediados sobre los miembros del ensamble que utilizan las parametrizaciones de capa límite MYJ (círculo), MYNN2 (triángulo) e YSU (cuadrado) para todos los pronósticos en conjunto.}\label{fig:soudings-pbl}
\end{figure}
Si analizamos el RMSE y BIAS para cada parametrización de capa límite calculados solo sobre los radiosondeos (Figura \ref{fig:soudings-pbl}), vemos que en general todas las parametrizaciones tienen valores de RMSE similares en todas las variables con una menor variabilidad que entre las parametrizaciones de convección. Algo similar ocurre con el BIAS y a diferencia de lo que se observaba al analizar las variables de superficie, en este caso ninguna parametrización parece ser la mejor al representar todas las variables en altura o en todos los experimentos.

Si bien no es posible elegir solo una parametrización de convección que mejor represente todas las variables, es interesante observar la buena performance de BMJ al representar la temperatura y humedad en altura. Sin embargo, Dillon et al. (2021), utilizando una configuración del ensamble multifísica similar a este trabajo, encontraron que esta parametrización era la que peor representaba la precipitación cuantificada utilizando el FSS. Un análisis similar sobre los experimentos discutidos en este capítulo mostraron el mismo patrón encontrado por Dillon et al. (2021), por ejemplo la Figura \ref{fig:fss-param} muestra el FSS calculado sobre el ensamble completo de RAD y los miembros que usan cada parametrización de campa límite o convección. En esta Figura se observa que BMJ es la parametrización de convección que peor representa la precipitación desde el punto de vista del FSS mientras que los miembros que utilizan KF y GF muestran los mejores resultados de todo el conjunto de parametrizaciones. Teniendo en cuenta este resultado y ante la poca variabilidad en los valores de los errores asociados a temperatura, humedad y viento, es posible que el reemplazo de BMJ por otra parametrización de convección de mejores resultados en términos de la representación de la precipitación sin producir mayor impacto en la representación de otras variables. También es interesante notar que MYNN2 es la parametrización de capa límite que junto con BMJ tiene un FSS sistemáticamente menor que el resto de las parametrizaciones y el ensamble completo. Teniendo en cuenta los buenos resultados de MYNN2 al representar el viento meridional, sería importante realizar nuevos experimentos de sensibilidad a esta parametrización para evaluar si debería seguir usándose dentro del ensamble multifísica.




\begin{figure}

\includegraphics{thesis_files/figure-latex/fss-param-1} \hfill{}

\caption{FSS calculado sobre la precipitación acumulada a 1 hora en una ventana móvil de 6 horas para el umbral de 25 mm y 10 km de escala sobre el pronóstico RAD inicializado a las 00 UTC. La línea roja (RAD) corresponde al ensamble completo, las líneas punteadas se calculan sobre los miembros del ensamble que usan cada parametrización de capa límite o convección.}\label{fig:fss-param}
\end{figure}
\hypertarget{conclusiones-1}{%
\section{Conclusiones}\label{conclusiones-1}}

Este capítulo evalúa el impacto de la asimilación de los distintos sistemas de observación considerados en pronósticos por ensambles a corto plazo inicializados usando los análisis estudiados en el Capítulo \ref{cap-3-analisis}. Se evalúa el impacto en las distintas variables meteorológicas y las diferencias entre las 2 inicializaciones (00 y 06 UTC del 22 de noviembre) generadas por la asimilación continua y la actualización de las condiciones de borde. Los pronósticos abarcan la iniciación, desarrollo y madurez del SCM en estudio para estudiar la capacidad de los pronóstico de representar este sistema convectivo y la precipitación que genera. Finalmente se pone el foco en el ensamble y se evalúa el desempeño de las distintas parametrizaciones de capa límite y convección al representar tanto la temperatura, humedad y viento como la representación de la precipitación.

En general, los pronósticos inicializados usando condiciones iniciales con asimilación de estaciones meteorológicas automáticas, los vientos estimados por satélites y las radianzas de cielo despejado son de mejor calidad que el pronóstico inicializado utilizando solo la red de observación convencional. La disminución del bias cálido y seco del modelo gracias a la asimilación de observaciones de estaciones meteorológicas automáticas también se observa en los pronósticos. Los pronósticos inicializados con condiciones iniciales que incluye información de viento derivado de satélite representan mejor el viento meridional aún cuando estas observaciones habían producido un impacto marginal en el análisis. Los pronósticos inicializados con asimilación de radianzas mantienen los patrones que se observaban en el análisis, por ejemplo se observa un aumento en la temperatura en niveles bajos y por lo tanto un ligero aumento del bias cálido. Lo más interesante en este punto es notar que las mejoras introducidas por las nuevas observaciones persisten a lo largo de todo el periodo pronosticado. Las inicializaciones de los pronósticos producen diferencias aunque solo en algunas variables, en particular se observan mejoras para la inicialización de las 06 UTC lo que indica que las observaciones de estaciones meteorológicas automáticas asimiladas entre las 00 y las 06 UTC tienen un impacto positivo en la representación de la temperatura cerca de superficie que luego se traslada a los pronósticos.

Los pronósticos por ensambles, si bien logran representar la precipitación en cuanto la ubicación espacial y temporal, la subestiman fuertemente respecto de lo observado. Esto es particularmente notorio cuando las condiciones iniciales solo incluyen observaciones convencionales. Desde este punto de vista si se observan mejoras importantes entre los pronósticos inicializados a las 00 y a las 06 UTC y estas mejoras en la representación de la precipitación perduran hasta el final del periodo pronosticado. Desde el punto de vista de la precipitación, los pronósticos con condiciones iniciales que incluyen la asimilación de vientos derivados de satélite son los que mejor la representan particularmente cuando analizamos la probabilidad de precipitación para umbrales altos. A diferencia de lo que pasaba en el análisis (Capítulo \ref{cap-3-analisis}), los pronósticos inicializados a partir de análisis con asimilación de radianzas no generan suficiente precipitación. Es posible que estos pronósticos generen una atmósfera más estable, y por lo tanto un menor desarrollo de la convección, debido al menor contenido de humedad presente en la capa límite.

Finalmente, del análisis por parametrizaciones surge en primer lugar poca o ausencia de variabilidad entre las distintas inicializaciones. En todos los casos MYJ representa mejor las variables termodinámicas en la capa límite mientras que MYNN2 representa mejor el viento. Entre las parametrizaciones de convección no se observa un patrón muy definido pero se destaca BMJ en la representación de la temperatura y humedad en altura y KF en la representación del viento meridional cuando se compara los pronósticos con observaciones independientes. Sin embargo, desde el punto de vista de la representación de la precipitación surge que tanto BMJ como MYNN2 son las parametrizaciones de convección y capa límite respectivamente que peor simulan la precipitación observada. Teniendo en cuenta estos resultados es posible que estas parametrizaciones no sean las más adecuadas para representar los procesos asociados en esta región. Además, esto refuerza la necesidad de generar un ensamble multifísica como el utilizado en este trabajo para poder capturar los posibles errores asociados al modelo y la representación de los procesos parametrizados.

\hypertarget{cap-5-goes}{%
\chapter{Asimilacion de radianzas del satelite geoestacionario GOES-16 en cielos despejados}\label{cap-5-goes}}

En este capítulo se evalúa el impacto que genera la asimilación de observaciones en cielos despejados del sensor ABI a bordo del satélite GOES-16 tanto en el análisis como en pronósticos determinísticos de alta resolución para el caso de estudio del 22 de noviembre de 2018. Estas observaciones son de particular interés en la región de Sudamérica por su alta resolución espacial y frecuencia temporal. Usaremos las observaciones de los canales sensibles al vapor de agua que brindan información en 3 niveles de la atmósfera, algo que no era posible con los sensores previos a bordo de los satélites GOES. Esperamos que la asimilación de estas observaciones contribuya a mejorar la representación de los procesos convectivos y de mesoescala en pronósticos a corto plazo. En particular, los pronósticos en alta resolución permitiran evaluar el desarrollo de la convección que produce el modelo de manera explícita.

Por ser un satélite y sensor nuevos, la última versión de GSI (V3.7) que fue publicada en julio de 2018 no incluye la posibilidad de asimilar estas observaciones. Fue necesario, entonces, modificar el código fuente del sistema de asimilación para incluir las rutinas necesarias utilizando como base los trabajos previos de Lee et al. (2019) y Liu et al. (2019). También por esta razón, el capítulo incluirá el análisis de experimentos para definir la mejor configuración posible y evaluar técnicamente la asimilación de observaciones de ABI.

Trabajos previos evaluaron el impacto de asimilar observaciones de los canales asociados al vapor de agua de ABI en escala regional. En líneas general estos trabajos (Lee et al. (2019), Jones et al. (2020), entre otros) utilizan las observaciones de los 3 canales en conjunto. Sin embargo Honda et al. (2018), utilizaron observaciones del sensor Advanced Himawari Imager (AHI), de iguales características que ABI, y encontraron mejores resultados al asimilar los canales 8 (6.2 \(\mu m\)) y 10 (7.3 \(\mu m\)). También observaron que el canal 9 (6.9 \(\mu m\)) está muy correlacionado con los dos primeros y por lo tanto no aporta información independiente en la asimilación. Por otro lado Lee et al. (2019) observaron que el impacto era mayor al asimilar observaciones de los 3 canales sobre el continente y el mar al mismo tiempo, incorporando así más observaciones en el análisis. En el mismo sentido, observaron que el incremento del análisis disminuía considerablemente al remover el canal 10 y en algunos casos también el canal 9, lo que indica que estos canales producen un impacto importante.

En la asimilación de datos y en particular de radianzas de satélites, muchas observaciones con alta resolución espacial deben ser descartadas porque podrían tener un impacto negativo en el análisis al no ser independientes entre si (Lazarus et al., 2010; Janjić et al., 2018). En los esquemas de asimilación actuales se asume que la matriz de covarianza de los errores de las observaciones \(R\) es diagonal, es decir, no hay correlación entre los errores de las distintas observaciones y cada observación es independiente. Sin embargo, las observaciones de satélites pueden estar muy correlacionadas debido a su resolución espectral y espacial. En este contexto se utilizan estrategias como el thinning para disminuir la resolución de las observaciones asumiendo que la correlación decae con la distancia y conservando la información que aportan. Por ejemplo Jones et al. (2020) aplican la técnica de thinning para llevar las observaciones con resolución de 2 km a una resolución de 15 km y así evitar el impacto de la correlación espacial entre las observaciones.

\hypertarget{metodologuxeda-2}{%
\section{Metodología}\label{metodologuxeda-2}}

\hypertarget{asim-abi}{%
\subsection{Asimilación de observaciones de GOES-16}\label{asim-abi}}

La asimilación directa de radianzas de ABI requiere las mismas consideraciones descriptas previamente en la Sección \ref{asim-rad} en cuanto a control de calidad, corrección de bias, thinning e identificación de pixeles nubosos. En esta Sección se describen las decisiones metodológicas y técnicas que se tomaron a la hora de asimilar las observaciones, en base a las pruebas preliminares realizadas.

La asimilación de radianzas en \emph{all sky}, es decir tanto para cielos claros como nubosos es todavía un importante desafío ya que estos dos tipos de observaciones requieren un control de calidad y estimaciones de errores específicos. En esta primera aproximación a la asimilación de observaciones de ABI solo se utilizaron observaciones en cielos despejados. Para detectar y filtrar los pixeles nubosos utilizamos la mascara de nubes ACM (por \emph{ABI Cloud Mask,} Heidinger and Straka III, 2013) disponible con la misma frecuencia que las observaciones. Esta mascara de nubes da información binaria (cielo despejado o cielo nuboso) y tienen una exactitud del 87\% (Heidinger and Straka III, 2013). La clasificación es realizada utilizando información de 10 canales de ABI e incluye criterios espectrales, espaciales y temporales. En las Figuras \ref{fig:cld-msk} se presentan 2 ejemplos de máscara de nubes a las 00 UTC del 20 de noviembre y a las 18 UTC del 22 de noviembre.


\begin{figure}
\includegraphics{thesis_files/figure-latex/cld-msk-1} \caption{Ubicación de píxeles nubosos (gris) según la máscara de nubes generada para GOES-16, durante el a) 06 UTC del 20 de noviembre y b) 18 UTC del 22 de noviembre.}\label{fig:cld-msk}
\end{figure}
Las radianzas de satélite pueden tener errores sistemáticos asociados a las condiciones de la atmósfera, a la geometría de la observación y condiciones específicas de los sensores que deben ser corregidos previo a la asimilación. Por este motivo se generó un primer experimento de asimilación para evaluar la magnitud de estos errores. Esta primera prueba utiliza la misma configuración de modelo y ensamble que los experimentos discutidos previamente e incluye la asimilación de los 3 canales de ABI sensibles al vapor de agua en conjunto.

La Figura \ref{fig:oma-dist} muestra la distribución de la diferencia entre las observaciones y el campo preliminar (OmF) y entre las observaciones y el análisis (OmA) para todo el periodo. Si las observaciones no tienen bias la distribución OmF estará centrada en cero. En este caso solo el canal 9 presenta un bias positivo pequeño del orden de 0.3 K (Tabla \ref{tab:oma-tabla}). Por esta razón y siguiendo trabajos previos (por ejemplo Honda et al., 2018) en los experimentos discutidos en este capítulo no se aplica una corrección del bias a estas observaciones.

La Figura \ref{fig:oma-dist} y la Tabla \ref{tab:oma-tabla} muestran además otro resultado importante. Si la asimilación de las observaciones se realizó correctamente es esperable que la diferencia OmA esté más cerca de cero que OmF y que su distribución tenga menos varianza. Esto se observa para los 3 canales asimilados y confirma que la implementación técnica dentro del sistema funciona correctamente. Notar que los cambios abruptos y simétricos en las curvas de distribución se deben al chequeo preliminar que se realiza como parte de los controles de calidad y que elimina aquellas observaciones que tienen un diferencia mayor a 2.2 K con respecto del campo preliminar.


\begin{figure}

{\centering \includegraphics{thesis_files/figure-latex/oma-dist-1} 

}

\caption{Distribución de la diferencia entre las observaciones y el campo preliminar (OmF) y la observación y el análisis (OmA) para los 3 canales de ABI y a lo largo de todo el periodo de prueba.}\label{fig:oma-dist}
\end{figure}
\begin{table}

\caption{\label{tab:oma-tabla}Observación menos análisis o campo preliminar medio calculado sobre todo el periodo en K.}
\centering
\begin{tabu} to \linewidth {>{\raggedright}X>{\raggedleft}X>{\raggedleft}X>{\raggedleft}X}
\toprule
 & Canal 8 & Canal 9 & Canal 10\\
\midrule
OmF & 0.0355 & 0.3212 & -0.0158\\
OmA & 0.0032 & 0.3140 & -0.0498\\
\bottomrule
\end{tabu}
\end{table}
Debido al desarrollo de actividad convectiva en el dominio durante el periodo simulado, la cantidad de observaciones en cielos despejados varía considerablemente (Figuras \ref{fig:cld-msk}a y b). La Figura \ref{fig:obs-rad}a muestra el número de radianzas asimiladas en cada ciclo de asimilación donde se aprecia una disminución del número de observaciones en las últimas 24 hs del periodo debido a la presencia de nubes. Sin embargo la cantidad de observaciones que aporta ABI es casi siempre entre 1 y 2 órdenes de magnitud mayor que el número de observaciones de satélites polares. En este caso ABI utiliza un thinning de 10 km (ver Sección \ref{thinning}) pero aún cuando se utiliza un thinning de 30 km, el número de observaciones de ABI asimiladas es siempre mayor a lo que pueden aportar los satélites polares. Las observaciones de los canales 8, 9 y 10 de ABI, además, se distribuyen de manera homogénea en el dominio y cubren gran parte de los niveles verticales de la atmósfera. La Figura \ref{fig:obs-rad}c muestra la distribución promedio de observaciones de cada canal con la altura. Si bien el nivel donde se observa el máximo de observaciones para cada canal es similar a lo esperado según sus funciones de peso (Figura \ref{fig:wf-goes}), las observaciones se encuentran distribuidas en un rango amplio de alturas, ya que la posición del máximo de la función de peso en cada caso depende de las condiciones particulares de la atmósfera en cada punto que difiere de la atmósfera estándar utilizada para el cálculo de las funciones de peso teóricas.


\begin{figure}
\includegraphics{thesis_files/figure-latex/obs-rad-1} \caption{a) Número de radianzas asimiladas en cada ciclo, b) número medio de radianzas asimiladas por ciclo en capas verticales de 50 hPa de espesor para los experimentos RAD (radianzas de satélites polares, círculos rojos) y ABI (radianzas de del sensor ABI, cuadrados turquesas) y c) número medio de radianzas asimiladas por ciclo en capaz verticales de 50 hPa de espesor para los 3 canales de ABI.}\label{fig:obs-rad}
\end{figure}
\hypertarget{configuracion-de-los-experimentos}{%
\subsection{Configuracion de los experimentos}\label{configuracion-de-los-experimentos}}

En este capítulo se analizan una serie de experimentos, primero para evaluar la sensibilidad a distintas combinaciones de canales y a la resolución del thinning y finalmente para comparar el impacto de la asimilación de observaciones de ABI en conjunto o no con observaciones de satélites polares. Los experimentos SATWND y RAD fueron analizados previamente en el Capítulo \ref{cap-3-analisis}. SATWND asimila observaciones convencionales, EMA y vientos derivados de satélite y RAD suma a lo anterior radianzas en aire claro de sensores a bordo de satélites polares. RAD+ABI incluye además observaciones de los 3 canales de vapor de agua de ABI con un thinning de 10 km. Finalmente el conjunto de experimentos ABI, descriptos en la Tabla \ref{tab:exp-rad}, incluyen radianzas de ABI pero no de satélites polares, y usan distintas combinaciones de canales y resolución de thinning para evaluar la sensibilidad de la asimilación de distintos canales y la resolución del thinning.
\begin{table}

\caption{\label{tab:exp-rad}Experimentos realizados. Todos los experimentos incluyen observaciones convencionales, EMA y vientos derivados de satélite. }
\centering
\begin{tabu} to \linewidth {>{\raggedright\arraybackslash}p{11em}>{\raggedright\arraybackslash}p{16em}>{\centering\arraybackslash}m{4em}>{\centering\arraybackslash}m{4em}}
\toprule
Experimento & Radianzas asimiladas & Canales ABI & Thining ABI\\
\midrule
SATWND & -- & -- & --\\
RAD & Radianzas de satélites polares & -- & --\\
RAD+ABI & Radianzas de satélites polares y ABI & 8, 9, 10 & 10 km\\
ABI\_ch890\_th10 (ABI) & Radianzas de ABI & 8, 9, 10 & 10 km\\
ABI\_ch890\_th30 & Radianzas de ABI & 8, 9, 10 & 30 km\\
\addlinespace
ABI\_ch90\_th30 & Radianzas de ABI & 9, 10 & 30 km\\
ABI\_ch0\_th30 & Radianzas de ABI & 10 & 30 km\\
\bottomrule
\end{tabu}
\end{table}
Todos los experimentos previamente descriptos usan la configuración del ensamble multifísica y del sistema de asimilación explicada en la Sección \ref{configmodelo}. De la misma manera los ciclos de análisis horarios se realizan desde las 18 UTC del 20 de noviembre hasta las 12 UTC del 23 de noviembre para cubrir el desarrollo y maduración del SCM.

Para estudiar el impacto de la asimilación de radianzas también se generaron pronósticos determinísticos en alta resolución inicializados a las 00 del 22 de noviembre (Figura \ref{fig:cycle-fcst}). Debido al altísimo costo computación que requiere una simulación en alta resolución, no fue posible generar pronósticos por ensambles como en el Capítulo \ref{cap-4-pronosticos} y entre otras cosas analizar los pronósticos de precipitación desde el punto de vista probabilístico. Sin embargo, estos pronósticos permiten analizar el desarrollo convectivo que el modelo resuelve explicitamente en mayor detalle, característica que puede mejorar con la asimilación de observaciones de ABI. En la Figura \ref{fig:dominio-det} se muestra el dominio \emph{d01} con 10 km de resolución (línea negra), idéntico al dominio utilizado para el resto de las simulaciones numéricas y el dominio anidado \emph{d02} con 2 km de resolución (línea turquesa). Las condiciones iniciales están dadas por los análisis de los distintos experimentos y las condiciones de borde son generadas a partir del GFSF determinístico. Además las variables atmosféricas para el \emph{d02} fueron inicializadas haciendo un escalado de la información del análisis. Todos los pronósticos utilizan las mismas parametrizaciones: del modelo de superficie terrestre (Noah-MP, Chen and Dudhia, 2001), de microfísica (esquema de un solo momento de 6 clases del WRF, Hong, Kim, et al., 2006) de procesos radiativos (esquema de onda corta y onda larga del RRTMG, Iacono et al., 2008), YSU (Hong, Noh, et al., 2006) para capa límite y KF (Kain, 2004) para covección (solo aplicado al \emph{d01}).

Además de los pronósticos inicializados a partir de los análisis, se generó un pronóstico inicializado a partir del análisis global de GFS, es decir, sin asimilación de datos regional. Este pronóstico tiene la misma configuración y usa el mismo conjunto de parametrizaciones que los pronósticos descriptos previamente.


\begin{figure}

{\centering \includegraphics{thesis_files/figure-latex/dominio-det-1} 

}

\caption{Dominio de baja resolución (d01, 10 km) marcado con un recuadro negro y dominio anidado de alta resolución (d02, 2 km) marcado en color turquesa.}\label{fig:dominio-det}
\end{figure}
\hypertarget{resultados-2}{%
\section{Resultados}\label{resultados-2}}

\hypertarget{canales}{%
\subsection{Sensibilidad a la combinación de canales}\label{canales}}

En este trabajo se generaron 3 experimentos que comparten la misma configuración que los experimentos previamente descriptos pero utilizando distintas combinaciones de canales de ABI para evaluar la sensibilidad a la asimilación de distintas combinaciones de canales. El experimento ABI\_ch890\_th30 asimila observaciones de los 3 canales, ABI\_ch90\_th30 asimila observaciones de los canales 9 y 10 (vapor de agua en niveles medios y bajos respectivamente) y ABI\_ch0\_th30 solo utiliza observaciones del canal 10. Estos 3 experimentos utilizan un thinning de 30 km mientras que un análisis más detallado del thinning se realizará en la siguiente Sección.


\begin{figure}
\includegraphics{thesis_files/figure-latex/area-canales-1} \caption{Porcentaje de área cubierta por precipitación a 1 hora superior a a) 1 \(mmh^{-1}\), b) 5 \(mmh^{-1}\) y c) 10 \(mmh^{-1}\) a lo largo del tiempo para el campo preliminar de los experimentos SATWND (línea naranja), ABI\_ch890\_th30 (línea verde), ABI\_ch90\_th30 (línea violeta) y ABI\_ch0\_th30 (línea celeste) entre las 00 UTC del 22 de noviembre y las 12 UTC del 23 de noviembre en el dominio de verificación (cuadro celeste en la Figura \ref{fig:dominio}a). En sombreado y para cada experimento se muestra el área máxima y mínima estimada por el ensamble del análisis. En línea discontinua negra se muestra la estimación de IMERG.}\label{fig:area-canales}
\end{figure}
La Figura \ref{fig:area-canales} muestra el porcentaje de área cubierta por precipitación superior a distintos umbrales calculado sobre el pronóstico a 1 hora (campo preliminar) y compara los experimentos para estudiar la sensibilidad al uso de canales con SATWND como experimento control y la estimación de precipitación de IMERG. Al igual que en otros experimentos, los experimentos ABI subestiman la precipitación observada aunque en menor medida que SATWND para el umbral de 1 mm (Figura \ref{fig:area-canales}a). En algunos casos, logran representar mejor la precipitación que SATWND particularmente para umbrales de precipitación mayores (Figura \ref{fig:area-canales}b-c) y algunas subregiones como la provincia de Córdoba (no se muestra). Comparando los experimentos ABI entre si, no se observan grandes diferencias y solo en algunos tiempos y/o umbrales ABI\_ch890\_th30 es mejor que los otros dos. Desde este punto de vista, la asimilación de las observaciones de los canales 8 y 9 en niveles altos y medios no parece aportar mucha información al análisis. Esto podría deberse a que los canales 8 y 9, o sus errores, están muy correlacionados y sería necesario realizar nuevos experimentos que aíslen el impacto de los canales 8 y 9 por separado para determinar si su asimilación independiente no produce mejores resultados que la asimilación del canal 10 o si la asimilación de cualquiera de los 3 canales es suficiente.

Desde el punto de vista del FSS (Figura \ref{fig:fss-canales}) los experimentos que incluyen observaciones de ABI tienen un impacto positivo mucho mayor que SATWND. Además, los 3 experimentos de ABI tienen un comportamiento similar entre si aunque en este caso ABI\_ch890\_th30 representa ligeramente mejor la precipitación por encima de 25 mm (Figuras \ref{fig:fss-canales}b y d).

Teniendo en cuenta que la asimilación de los 3 canales de ABI en conjunto no degradan el análisis y por el contrario, en algunos casos parecen tener un rendimiento mejor, para los experimentos de este capítulo se asimilaran los 3 canales asociados al vapor del agua al mismo tiempo.


\begin{figure}
\includegraphics{thesis_files/figure-latex/fss-canales-1} \caption{FSS calculado sobre la precipitación acumulada a 1 hora en una ventana móvil de 6 horas para umbrales de 1 mm (a y c) y 25 mm (b y d), en escalas de 10 km (a y b) y 100 km (c y d), para el campo preliminar de los experimentos SATWND (línea naranja), ABI\_ch890\_th30 (línea verde), ABI\_ch90\_th30 (línea violeta) y ABI\_ch0\_th30 (línea celeste).}\label{fig:fss-canales}
\end{figure}
\hypertarget{thinning}{%
\subsection{Sensibilidad al thinning}\label{thinning}}

Existen diversos algoritmos para realizar el thinning, GSI en particular utiliza una metodología simple que consiste en definir una retícula de baja resolución y seleccionar las observaciones en alta resolución de acuerdo la distancia de estás a los puntos de la retícula gruesa y según criterios de calidad descriptos en la Sección \ref{sat}.

Es importante definir entonces la resolución apropiada para la retícula de baja resolución y estudiar el impacto que esto tiene en el análisis. Para esto se realizaron 2 experimentos ABI\_ch890\_th30 con un thinning de 30 km, es decir, que lleva las observaciones con una resolución de 2 km a una retícula de 30 km siguiendo trabajos previos (por ejemplo Lee et al., 2019; Singh et al., 2016) y ABI\_ch890\_th10 con un thinning de 10 km, igual a la resolución de las simulaciones numéricas. Las 2 resoluciones elegidas tienen en cuenta por un lado el efecto de la correlación entre los errores de las observaciones y la resolución del modelo y por el otro, no perder información importante que podría incluirse en el análisis. En particular, si bien la resolución del modelo en los experimentos es de 10 km, los procesos que puede representar explicitamente son de escala mayor y por lo tanto es posible que asimilar observaciones con una resolución muy cercana al modelo, que contiene información de procesos en escalas no totalmente resueltas por el modelo, no genere el impacto esperado.


\begin{figure}
\includegraphics{thesis_files/figure-latex/area-thinning-1} \caption{Porcentaje de área cubierta por precipitación a 1 hora superior a a) 1 \(mmh^{-1}\), b) 5 \(mmh^{-1}\) y c) 10 \(mmh^{-1}\) a lo largo del tiempo para el campo preliminar de los experimentos SATWND (línea naranja), ABI\_ch890\_th30 (línea verde), y ABI\_ch890\_th10 (línea celeste) entre las 00 UTC del 22 de noviembre y las 12 UTC del 23 de noviembre en el dominio de verificación (cuadro celeste en la Figura \ref{fig:dominio}a). En sombreado y para cada experimento se muestra el área máxima y mínima estimada por el ensamble del análisis. En línea continua negra se muestra la estimación de IMERG.}\label{fig:area-thinning}
\end{figure}
La Figura \ref{fig:area-thinning} muestra el porcentaje de área cubierta por precipitación superior a distintos umbrales a lo largo del tiempo para los experimentos de thinning y tomando a SATWND como experimento control. En todos los casos ABI\_ch890\_th10, el experimento con un thinning de 10 km, es el que representa mejor el área cubierta por precipitación, particularmente en las horas de mayor actividad convectiva. Si bien todos los experimentos subestiman la precipitación en comparación con IMERG, ABI\_ch890\_th10 se acerca a esta estimación sobre todo cuando observamos el valor máximo del ensamble.


\begin{figure}
\includegraphics{thesis_files/figure-latex/thinning-fss-1} \caption{FSS calculado sobre la precipitación acumulada a 1 hora en una ventana móvil de 6 horas para umbrales de 1 mm (a y c) y 25 mm (b y d), en escalas de 10 km (a y b) y 100 km (c y d), para el campo preliminar de los experimentos SATWND (línea naranja), ABI\_ch890\_th30 (línea verde) y ABI\_ch890\_th10 (línea celeste).}\label{fig:thinning-fss}
\end{figure}
Desde el punto de vista del FSS y para 1 \(mm h^{-1}\) (Figura \ref{fig:thinning-fss}a y c) ABI\_ch890\_th10 tiene igual o mejor rendimiento que ABI\_ch890\_th30 pero ocurre lo contrario para el umbral de 25 \(mm h^{-1}\) (Figura \ref{fig:thinning-fss}b y d). Si bien las diferencias de FSS entre los experimentos de thinning son muy pequeñas, el área cubierta por precipitación y otros análisis como la comparación cualitativa con estimaciones de IMERG y cuantitativa con radiosondeos de RELAMPAGO (no mostrados), resalta la necesidad de utilizar un thinning cercano a la resolución del modelo. Por esta razón la configuración de thinning a utilizar en los experimentos posteriores será de 10 km.

\hypertarget{comparaciuxf3n-del-impacto-de-asimilar-observaciones-de-abi-con-otras-fuentes-de-informaciuxf3n}{%
\subsection{Comparación del impacto de asimilar observaciones de ABI con otras fuentes de información}\label{comparaciuxf3n-del-impacto-de-asimilar-observaciones-de-abi-con-otras-fuentes-de-informaciuxf3n}}

\hypertarget{impacto-de-las-observaciones-de-abi-en-el-anuxe1lisis}{%
\subsubsection{Impacto de las observaciones de ABI en el análisis}\label{impacto-de-las-observaciones-de-abi-en-el-anuxe1lisis}}

Al igual que en la Sección \ref{impacto-analisis}, es importante primero entender el impacto de asimilar este conjunto de observaciones en el análisis. En particular, este impacto podría variar en presencia de otros conjuntos de observaciones tales como otras radianzas de satélites polares. Por esta razón se comparará el experimento ABI (previamente ABI\_ch890\_th10) que asimila los 3 canales de vapor de agua con SATWND donde no se asimilan radianzas y es considerado el experimento control, RAD donde se asimilan radianzas de satélites en órbitas polares y RAD+ABI donde se incluyen todas las observaciones al mismo tiempo. De esta manera podremos evaluar cuál es el impacto de las observaciones de ABI de manera independiente y cuando son asimiladas en conjunto con radianzas de satélites polares.

En primer lugar calculamos perfiles verticales de temperatura y humedad específica promediando espacialmente la media del ensamble de cada experimento a lo largo de todo el periodo. Posteriormente para evaluar el impacto de la asimilación de las radianzas, calculamos la diferencia entre los perfiles verticales de los experimentos con estas observaciones y SATWND que no las incluye (Figura \ref{fig:TQ-diff-abi}). Previamente habíamos observado que la asimilación de radianzas de satélites polares producía un calentamiento en niveles bajos (Figura \ref{fig:TQ-diff-abi}a), la asimilación de observaciones de ABI en este caso produce, además, un enfriamiento en niveles medios y un calentamiento previo al paso del frente frío y desarrollo de la convección durante el 21 de noviembre lo que puede contribuir a incrementar el CAPE y por lo tanto las condiciones para el desarrollo de convección intensa.


\begin{figure}

\includegraphics{thesis_files/figure-latex/TQ-diff-abi-1} \hfill{}

\caption{Diferencia entre la media del ensamble de los análisis a) y d) RAD-SATWND, b) y e) RAD+ABI-SATWND, y c) y f) ABI-SATWND para los perfiles verticales espacialmente promediados de la temperatura (a, b y c, en \(K\)) y la humedad específica (d, e y f en \(gkg^{-1}\)) calculados sobre el dominio interior (recuadro rojo en la Figura \ref{fig:dominio}a) para cada ciclo de análisis.}\label{fig:TQ-diff-abi}
\end{figure}
A las 12 UTC del 22 de noviembre se observa un dipolo en niveles medios y altos que se intensifica con la asimilación de ABI. Esto puede estar relacionado a un menor desarrollo de la convección en ABI a esa hora respecto de SATWND. Pero a partir de las 00 UTC del 23 de noviembre el dipolo se invierte indicando que la convección se intensifica en ABI. Esto también podría explicar el enfriamiento que se observa en superficie debido a una pileta de aire frio más intensa asociada al sistema convectivo.
Al comparar la diferencia entre los perfiles verticales de cada experimento y ERA5 (Figuras \ref{fig:era5-abi}a-d), vemos que este enfriamiento en niveles medios (Figura \ref{fig:era5-abi}c-d) produce una mayor diferencia entre los análisis que incluyen ABI y ERA5.

La diferencia en la humedad específica para los distintos experimentos se muestra en las Figuras \ref{fig:TQ-diff-abi}d-f, donde vemos que la asimilación de observaciones de ABI produce un humedecimiento por encima de 850 hPa y este efecto es mayor cuando no se asimilan radianzas de satélites polares en el experimento ABI (Figura \ref{fig:TQ-diff-abi}f). Este humedecimiento junto con los cambios en la temperatura generan un aumento en el CAPE previo al desarrollo de la convección (no se muestra). El secamiento que se observa en los experimentos con radianzas y que ya se observaba previamente en RAD, no es consistente con el desarrollo de la precipitación, que ocurre primero en SATWND consumiendo la humedad presente en la capa límite, por lo que son necesarios nuevos experimentos para entender mejor el mecanismo que genera este efecto. Nuevamente, la diferencia entre los experimentos y ERA5 para esta variable (Figuras \ref{fig:era5-abi}e-h) muestra que la asimilación de observaciones de ABI aleja los análisis del estado de la atmósfera que representa ERA5.


\begin{figure}

\includegraphics{thesis_files/figure-latex/era5-abi-1} \hfill{}

\caption{Diferencia entre la media del ensamble del análisis de cada experimento y el ERA5 para los perfiles verticales espacialmente promediados de la temperatura del aire (K, a--d), la humedad específica (\(g kg^{-1}\), e--h) y el viento meridional (\(m s^{-1}\), i--l) calculados sobre el dominio interior (recuadro rojo en la Figura \ref{fig:dominio}a) para cada ciclo de análisis.}\label{fig:era5-abi}
\end{figure}
Las Figuras \ref{fig:UV-diff-abi} muestran la comparación entre los experimentos que incluyen radianzas y SATWND para las componentes del viento. En este caso las diferencias entre experimentos son más sutiles sin embargo, se aprecia una disminución en el viento zonal, y por lo tanto en la cortante de vertical de viento, por encima de la 700 hPa en los experimentos RAD+ABI y ABI. Esta disminución es más marcada cuando se asimilan radianzas de satélites polares y ABI en conjunto. En este caso la comparación con ERA5 (Figuras \ref{fig:era5-abi}i-l) muestra que el viento zonal en los experimentos RAD y ABI se acerca más a ERA5 que SATWND que no incluye radianzas o RAD+ABI que asimila todas las radianzas en conjunto. Esto indicaría que los experimentos que incluyen observaciones de ABI representan mejor el flujo zonal para este caso de estudio. En niveles bajos las diferencias son pequeñas y solo se observa un ligero aumento del viento zonal en ABI respecto de SATWND (Figura \ref{fig:UV-diff-abi}c) asociado al flujo postfrontal.

Las Figuras \ref{fig:UV-diff-abi}d-f muestran que los experimentos que asimilan ABI tienen una mayor diferencia respecto de SATWND en el viento meridional en altura a partir del 22 de noviembre a las 12 UTC cuando hay mayor convección en el dominio. Este efecto ya era notorio al analizar el impacto de las radianzas de satélites polares (Figura \ref{fig:UV-diff}f) que ahora se intensifica cuando se incluyen radianzas de ABI. Estas diferencias están asociadas a un retraso en el desarrollo de la convección en los experimentos que incluyen radianzas respecto de SATWND. Al igual que para el viento meridional, la comparación entre los experimentos y ERA5 (Figuras \ref{fig:era5-abi}m-o) muestra que, por encima de 500 hPa, los experimentos que incluyen radianzas tienen un viento meridional más parecido a ERA5 que SATWND. Esta mejora es más notoria en ABI donde no se incluyen observaciones de satélites polares.


\begin{figure}

\includegraphics{thesis_files/figure-latex/UV-diff-abi-1} \hfill{}

\caption{Diferencia entre la media del ensamble de los análisis a) y d) RAD-SATWND, b) y e) RAD+ABI-SATWND, y c) y f) ABI-SATWND para los perfiles verticales espacialmente promediados del viento zonal (a, b y c, en \(ms^{-1}\)) y viento meridional (d, e y f en \(ms^{-1}\)) calculados sobre el dominio interior (recuadro rojo en la Figura \ref{fig:dominio}a) para cada ciclo de análisis. Los contornos negros corresponden al viento zonal y meridional para (a,d) RAD, (b,e) RAD+ABI, y (c,f) ABI ya que son los experimentos tienen más observaciones asimiladas en cada panel.}\label{fig:UV-diff-abi}
\end{figure}
Habiendo analizado el impacto de la asimilación de radianzas en términos generales, la Figura \ref{fig:sondeos-abi} muestra el RMSE y BIAS calculado al comparar los experimentos con los radiosondeos de RELAMPAGO en la región que se observa la Figura \ref{fig:dominio}b. Por debajo de 5 km las observaciones de ABI generan una degradación de la temperatura de rocío durante el IOP 7 y de la temperatura en el IOP 8 con aumento tanto del RMSE como del BIAS. Pero se observa una mejora en la representación del viento zonal y meridional en niveles bajos durante el IOP 8. Sin embargo, las mayores diferencias entre SATWND o RAD y los experimentos que incluyen observaciones de ABI se observan por encima de 5 km donde se ubican los máximos de las funciones de peso de los canales de vapor de agua (Figuras \ref{fig:wf} y \ref{fig:wf-goes}). La asimilación de observaciones de ABI, en este caso, ayuda a disminuir el RMSE y el BIAS en todas las variables durante el IOP 7 por encima de la capa límite. Este IOP estuvo caracterizado por ser un periodo relativamente estable con cielos mayoritariamente despejados y un flujo del norte en niveles bajos asociado a una advección cálida y húmeda por lo que se asimilaron una mayor cantidad de observaciones de radianzas en comparación con el día siguiente. Esta es una de las razones que podrían explicar el impacto que se observa en todas las variables. Es interesante notar que el error en la humedad durante el IOP 7 mejora tanto en ABI como en RAD+ABI, mientras que RAD es el experimento con mayor RMSE. Esto muestra que la combinación de distintas fuentes de observaciones genera impactos que no son lineales. Durante el IOP 8 las mejoras se observan principalmente en la temperatura y temperatura de punto de rocío y en menor medida en el viento meridional en niveles altos. Finalmente, la característica sobresaliente de este análisis es la mejora sustancial en la representación de la humedad en niveles medios cuando se asimilan observaciones de ABI. Si bien esto es opuesto a lo observado al analizar las diferencias entre ABI y RAD+ABI con ERA5, es importante notar en este punto que la comparación con el reanálisis tiene limitaciones tanto por la resolución del reanálisis como por las observaciones disponibles para su generación.


\begin{figure}

{\centering \includegraphics{thesis_files/figure-latex/sondeos-abi-1} 

}

\caption{RMSE (línea sólida) y BIAS (línea discontinua) de a) la temperatura (\(K\)), b) la temperatura del punto de rocío (\(K\)), c) el viento zonal (\(m s^{-1}\)) y d) el viento meridional (\(m s^{-1}\)) calculados comparando el análisis de cada experimento con los radiosondeos de RELAMPAGO durante el IOP 7 y el IOP 8. La línea naranja corresponde a SATWND, la línea roja a RAD, RAD+ABI se representa con una línea verde y ABI con una línea turquesa.}\label{fig:sondeos-abi}
\end{figure}
\hypertarget{impacto-en-la-precipitaciuxf3n}{%
\paragraph{Impacto en la precipitación}\label{impacto-en-la-precipitaciuxf3n}}

En esta Sección ponemos el foco nuevamente en la precipitación para evaluar la habilidad de los análisis al representar la precipitación acumulada horaria del campo preliminar cuando incluyen o no información de las observaciones de ABI. En primer lugar analizamos el porcentaje de área cubierta por precipitación para distintos umbrales (Figuras \ref{fig:area-abi}) donde se observa una mejora considerable en la ubicación temporal del área para los experimentos ABI, RAD+ABI y RAD cuando se los compara con SATWND que no asimila radianzas. Sin embargo, la incorporación de observaciones de ABI es lo que genera el mayor impacto en la magnitud del área cubierta por precipitación que aumenta más de un 5\% para el umbral de 1 mm en comparación con RAD que solo incluye radianzas de satélites polares.

En este punto es interesante observar las posibles diferencias entre RAD+ABI y ABI para evaluar el aporte de las radianzas de satélites polares cuando se asimilan en conjunto con radianzas de ABI. Desde el punto de vista del área cubierta por precipitación, las diferencias son pequeñas, RAD+ABI subestima el área previo al máximo de precipitación respecto de ABI y lo contrario ocurre posterior al máximo. Esto podría indicar que las radianzas de ABI son suficientes para representar la precipitación y que la asimilación de radianzas de satélites polares no agrega información, al menos para este caso de estudio.


\begin{figure}
\includegraphics{thesis_files/figure-latex/area-abi-1} \caption{Porcentaje de área cubierta por precipitación a 1 hora superior a a) 1 \(mmh^{-1}\), b) 5 \(mmh^{-1}\) y c) 10 \(mmh^{-1}\) a lo largo del tiempo para el campo preliminar de los experimentos SATWND (línea naranja), RAD (línea roja), RAD+ABI (línea verde), y ABI (línea turquesa) entre las 00 UTC del 22 de noviembre y las 12 UTC del 23 de noviembre en el dominio de verificación (cuadro celeste en la Figura \ref{fig:dominio}a). En sombreado y para cada experimento se muestra el área máxima y mínima estimada por el ensamble. En línea discontinua negra se muestra la estimación de IMERG.}\label{fig:area-abi}
\end{figure}
Finalmente, la Figura \ref{fig:fss-abi} muestra el FSS calculado comparando la estimación de IMERG con cada experimento para distintos umbrales y escalas espaciales, siendo 10 km igual a la resolución del modelo. En todos los casos RAD+ABI y ABI tienen valores mayores de FSS que RAD que ya mostraba mejoras importantes respecto de SATWND para umbrales bajos de precipitación. Solo para el umbral de 25 \(mmh^{-1}\) hay un periodo alrededor de las 12 UTC del 22 de noviembre donde todos los análisis que asimilan radianzas no logran representar correctamente la convección y muestran un FSS menor a SATWND. Esto se repite para umbrales de precipitación mayores donde SATWND genera precipitación más intensa pero menos ordenada que ABI durante las primeras 12 horas del 22 de noviembre. Es particularmente notorio si se observan campos de precipitación para cada experimento, donde se ve que el área donde se produce precipitación en ABI es homogénea mientras que en SATWND es discontinua con focos de precipitación aislados (no se muestra). Comparando RAD+ABI y ABI vemos que entre las 06 y las 18 UTC del 22 de noviembre ABI representa la precipitación ligeramente mejor que RAD+ABI, lo que podría indicar que las radianzas de satélites polares tienen un impacto negativo en el análisis. Sin embargo hacia el final del periodo simulado, la relación es opuesta. RAD+ABI tiene valores de FSS mayores que ABI y cercanos a RAD. En la siguiente Sección por lo tanto, se evaluará el impacto de asimilar radianzas en pronósticos a corto plazo y se analizará si el impacto observado en los análisis perdura en los pronósticos y mejora la representación de la convección.


\begin{figure}
\includegraphics{thesis_files/figure-latex/fss-abi-1} \caption{FSS calculado sobre la precipitación acumulada a 1 hora en una ventana móvil de 6 horas para umbrales de 1 mm (a y c) y 25 mm (b y d), en escalas de 10 km (a y b) y 100 km (c y d), para el campo preliminar de los experimentos SATWND (línea naranja), RAD (línea roja), RAD+ABI (línea verde), y ABI (línea turquesa) entre las 00 UTC del 22 de noviembre y las 12 UTC del 23 de noviembre.}\label{fig:fss-abi}
\end{figure}
\hypertarget{impacto-de-las-observaciones-de-abi-en-pronuxf3sticos-determinuxedsticos-de-alta-resoluciuxf3n}{%
\subsubsection{Impacto de las observaciones de ABI en pronósticos determinísticos de alta resolución}\label{impacto-de-las-observaciones-de-abi-en-pronuxf3sticos-determinuxedsticos-de-alta-resoluciuxf3n}}

A continuación se analizará el impacto de la asimilación de observaciones de ABI en los pronósticos de alta resolución generados a partir de los análisis. También se incluye en la comparación un pronóstico inicializado a partir del análisis global de GFS (generado con el \emph{Global Data Assimilation System}, GDAS), que se utiliza como condición inicial y de borde para la generación de pronósticos operativos en Argentina.

\hypertarget{representaciuxf3n-de-la-temperatura-humedad-y-viento-en-los-pronuxf3sticos}{%
\paragraph{Representación de la temperatura, humedad y viento en los pronósticos}\label{representaciuxf3n-de-la-temperatura-humedad-y-viento-en-los-pronuxf3sticos}}

En primer lugar se analiza la representación de la temperatura, humedad y viento en los pronósticos de alta resolución. En este caso se compararan los experimentos RAD, RAD+ABI y ABI con SATWND como experimento control para evaluar el impacto de incorporar las radianzas de satélites polares y de ABI. Posteriormente se comparan los experimentos con observaciones independientes de EMA y radiosondeos de RELAMPAGO.

En primer lugar calculamos perfiles verticales de temperatura y humedad específica promediados espacialmente a partir de los pronósticos determinísticos a 36 hs, con los que generamos la diferencia entre los perfiles verticales de los experimentos con asimilación de radianzas y SATWND. La diferencia en los perfiles promediados de temperatura (Figura \ref{fig:TQ-diff-fcst}a-c) muestra que las radianzas de satélites polares producen un calentamiento en niveles bajos en comparación con SATWND y esto se repite, aunque en menor medida, con la asimilación de observaciones de ABI. De hecho luego de las 16 UTC del 22 de noviembre, los pronósticos inicializados a partir de RAD+ABI y ABI muestran una capa límite más fría que SATWND lo que podría indicar que desarrollaron un frente frío o una pileta de aire frío asociada a la convección más intensos. En niveles medios y altos las características de las diferencias, si bien de mayor magnitud, reflejan lo observado en los análisis (Figura \ref{fig:TQ-diff-abi}a-c) por lo que, nuevamente, la información aportada por la asimilación regional de las observaciones produce un impacto prolongado en el pronóstico que determinan el momento de inicio e intensificación de la convección. Respecto de la humedad (Figura \ref{fig:TQ-diff-fcst}d-f), la comparación con SATWND muestra que la asimilación de observaciones de ABI en las condiciones iniciales luego genera una disminución de la humedad en niveles bajos del pronóstico en comparación con RAD, luego de las 12 UTC del 22 de noviembre y particularmente en las últimas 12 horas de pronóstico. Esta diferencia puede estar asociada al desarrollo de la convección que consume la humedad disponible en capas bajas y que podría ser más intensa en ABI que en RAD.


\begin{figure}

\includegraphics{thesis_files/figure-latex/TQ-diff-fcst-1} \hfill{}

\caption{Diferencia entre los pronósticos determinísticos de a) y d) RAD-SATWND, b) y e) RAD+ABI-SATWND, y c) y f) ABI-SATWND para los perfiles verticales espacialmente promediados de la temperatura (a, b y c, en \(K\)) y la humedad específica (d, e y f en \(g\ kg^{-1}\)) calculados sobre el dominio interior (recuadro rojo en la Figura \ref{fig:dominio}a) para cada ciclo de análisis.}\label{fig:TQ-diff-fcst}
\end{figure}
Las diferencias para el viento zonal (Figuras \ref{fig:UV-diff-fcst}a-c) muestran que la asimilación de observaciones de satélites polares y ABI en conjunto produce el mayor impacto en los pronósticos respecto de SATWND en todos los niveles pero en particular por encima de la capa límite. En cuanto al viento meridional (Figuras \ref{fig:UV-diff-fcst}d-f), se observa que los pronósticos con condiciones iniciales que incluyen observaciones de ABI son más cercanos a SATWND que previamente había demostrado ser el pronóstico con mejor representación de la precipitación (ver Sección \ref{prono-impacto}). Por otro lado, la evolución del viento meridional a lo largo del tiempo para cada pronóstico (contornos en la Figuras \ref{fig:UV-diff-fcst}d-f) muestra que los experimentos ABI y RAD+ABI tiene un flujo en altura por detrás del sistema convectivo es más intenso (Figuras \ref{fig:UV-diff-fcst}e-f) que RAD. El dipolo que se observa en la diferencia entre los pronósticos, nuevamente indica que la convección que genera RAD o ABI está retrasada en el tiempo respecto de SATWND. Sin embargo, las diferencias entre ABI-SATWND disminuyen respecto de RAD-SATWND, lo que podría ser el resultado del uso de condiciones iniciales con mayor información del entorno convectivo.

Si bien todo lo anterior corresponde a una comparación relativa al pronóstico SATWND, en líneas generales el signo del impacto observado en los pronósticos inicializados a partir de los análisis que asimilan radianzas (Figuras \ref{fig:TQ-diff-fcst} y \ref{fig:UV-diff-fcst}) coincide con lo observado en la evaluación del impacto de los análisis (Figuras \ref{fig:TQ-diff-abi} y \ref{fig:UV-diff-abi}).


\begin{figure}

\includegraphics{thesis_files/figure-latex/UV-diff-fcst-1} \hfill{}

\caption{Diferencia entre los pronósticos determinísticos de a) y d) RAD-SATWND, b) y e) RAD+ABI-SATWND, y c) y f) ABI-SATWND para los perfiles verticales espacialmente promediados del viento zonal (a, b y c, en \(ms^{-1}\)) y viento meridional (d, e y f en \(ms^{-1}\)) calculados sobre el dominio interior (recuadro rojo en la Figura \ref{fig:dominio}a) para cada ciclo de análisis. Los contornos negros corresponden al viento zonal y meridional para (a,d) RAD, (b,e) RAD+ABI, y (c,f) ABI.}\label{fig:UV-diff-fcst}
\end{figure}
En la comparación de los pronósticos con observaciones independientes se incluye también el pronóstico determinístico inicializado a partir del análisis de GFS. En la Figura \ref{fig:sondeos-abi-fcst} se muestran los perfiles de RMSE y BIAS calculados comparando los pronósticos con los radiosondeos de RELAMPAGO. La Figura \ref{fig:sondeos-abi-fcst}a muestra que los errores asociados a la temperatura en niveles bajos son mayores en los experimentos que incluyen radianzas de satélites polares en comparación con GFS o ABI. En niveles medios y altos GFS presenta el mayor RMSE y BIAS en la temperatura, lo que pone en evidencia el impacto positivo de la asimilación de observaciones a escala regional. Esto mismo ocurre para la temperatura de punto de rocío (Figura \ref{fig:sondeos-abi-fcst}b) que muestra que el RMSE y BIAS de GFS es hasta 5\(^\circ\)K mayor que para el resto de los pronósticos en niveles medios. El RMSE del viento zonal también muestra que GFS es el pronóstico con mayor error alrededor de 5 km mientras que por encima de los 10 km el error en RAD aumenta por sobre los otros (Figura \ref{fig:sondeos-abi-fcst}c). Para el viento meridional, el RMSE de todos los pronósticos tiene un comportamiento similar aunque en algunos niveles (alrededor de 5 y 11 km) SATWND muestra un aumento en el error. En lineas generales ninguno de los pronósticos inicializados a partir de análisis con asimilación regional es consistentemente mejor para todas las variables analizadas.


\begin{figure}

{\centering \includegraphics{thesis_files/figure-latex/sondeos-abi-fcst-1} 

}

\caption{RMSE (línea sólida) y BIAS (línea discontinua) de a) la temperatura (\(K\)), b) la temperatura del punto de rocío (\(K\)), c) el viento zonal (\(m\ s^{-1}\)) y d) el viento meridional (\(m\ s^{-1}\)) calculados comparando los pronósticos determinísticos de cada experimento con los radiosondeos de RELAMPAGO. La línea naranja corresponde a SATWND, la línea roja a RAD, RAD+ABI se representa con una línea verde, ABI con una línea turquesa y en negro se muestra la comparación con el pronóstico inicializado con GFS.}\label{fig:sondeos-abi-fcst}
\end{figure}
El RMSE y BIAS calculado comparando los pronósticos con las observaciones de EMA se muestra en la Figura \ref{fig:aut-rmse-fcst} para todo el periodo pronosticado. Si bien en general todos los pronósticos tienen errores similares, se observa que los pronósticos inicializados a partir de análisis con asimilación regional tienen errores grandes al inicio del pronóstico que luego disminuyen rápidamente en la primera hora. Esto posiblemente esté asociado a que el sistema de asimilación solo actualiza las variables 3D del modelo durante la asimilación y no así la temperatura o humedad a 2 metros que se utiliza en esta verificación. Otro factor que podría influir es el uso de la media del ensamble como condición inicial, tema que se discutirá más adelante. Por otro lado el BIAS de SATWND es consistentemente menor en las primeras 10 horas de pronóstico posiblemente gracias a la asimilación de observaciones de EMA que mejoran la representación de la humedad en las condiciones iniciales. Para la temperatura, GFS mantiene un error menor al resto de los experimentos durante las últimas 12 horas de pronóstico aunque el comportamiento general de los errores no difiere del resto. Cómo fue analizado previamente, ninguno de los pronósticos inicializados a partir de análisis con asimilación regional es consistentemente mejor para todas las variables analizadas hasta ahora.


\begin{figure}

{\centering \includegraphics{thesis_files/figure-latex/aut-rmse-fcst-1} 

}

\caption{Evolución del RMSE y BIAS de a) humedad específica (\(g kg{^-1}\)), b) temperatura (\(K\)), c) el viento zonal (\(m\ s^{-1}\)) y d) el viento meridional (\(m s^{-1}\)) calculados comparando los pronósticos determinísticos de cada experimento con las observaciones de EMA. La línea naranja corresponde a SATWND, la línea roja a RAD, RAD+ABI se representa con una línea verde, ABI con una línea turquesa y en negro se muestra el pronóstico inicializado con GFS.}\label{fig:aut-rmse-fcst}
\end{figure}
\hypertarget{representaciuxf3n-de-la-convecciuxf3n-y-precipitaciuxf3n}{%
\paragraph{Representación de la convección y precipitación}\label{representaciuxf3n-de-la-convecciuxf3n-y-precipitaciuxf3n}}

Para analizar la representación de la precipitación se compararon los pronósticos determinísticos con las estimaciones de IMERG para calcular el FSS y el porcentaje de área cubierta por precipitación. En cuanto a la convección, se utilizó el modelo de transferencia radiativa CRTM para calcular la temperatura de brillo asociada al canal 13 (infrarrojo) de ABI a partir de los pronósticos y así compararlos cualitativamente con las observaciones de GOES-16. Además se calculó el FSS para distintos umbrales de temperatura de brillo para hacer una comparación cuantitativa del desarrollo de la convección observada por el satélite GOES-16.

El porcentaje de área cubierta por precipitación (Figura \ref{fig:area-abi-fcst}) muestra que los pronósticos subestiman la ocurrencia de precipitación aún cuando la resolución del modelo es alta y representa la convección de manera explicita. Si bien los pronósticos inicializados a partir de GFS y SATWND tienen un mayor porcentaje de área cubierta por precipitación mayor a 1 mm y en algunos tiempos también para precipitación mayor a 5 mm, los pronósticos que incluyen información de las observaciones de ABI representan muy bien el máximo observado para todos los umbrales. También es interesante notar que SATWND tiene un comportamiento similar al resto de los pronósticos y apenas se observan indicios de que la precipitación se adelante en el tiempo. Esto podría deberse a que los pronósticos inicializados con observaciones de ABI contribuyen a la iniciación temprana de la precipitación disminuyendo las diferencias entre SATWND y ABI. GFS es el pronóstico que genera precipitación más rápidamente, ya que el área cubierta en todos los umbrales es mayor durante las primeras 6 a 12 horas. Sin embargo, hacia el final del periodo pronosticado GFS es el que muestra un porcentaje de área cubierta menor en comparación con el resto de los pronósticos.


\begin{figure}
\includegraphics{thesis_files/figure-latex/area-abi-fcst-1} \caption{Porcentaje de área cubierta por precipitación a 1 hora superior a a) 1 \(mmh^{-1}\), b) 5 \(mmh^{-1}\) y c) 10 \(mmh^{-1}\) a lo largo del tiempo para el campo preliminar de los experimentos SATWND (línea naranja), RAD (línea roja), RAD+ABI (línea verde), ABI (línea turquesa) y GFS (línea negra continua) entre las 00 UTC del 22 de noviembre y las 12 UTC del 23 de noviembre en el dominio \emph{d02} (Figura \ref{fig:dominio-det}). En sombreado y para cada experimento se muestra el área máxima y mínima estimada por el ensamble del análisis. En línea discontinua negra se muestra la estimación de IMERG.}\label{fig:area-abi-fcst}
\end{figure}
El FSS calculado comparando los pronósticos de precipitación con la estimación de IMERG (Figura \ref{fig:fss-abi-fcst}) muestra nuevamente que GFS representa de manera más precisa la precipitación al comienzo del pronóstico, seguido por SATWND. Sin embargo con el paso de las horas de pronóstico, los experimentos con asimilación de radianzas pasan a tener valores de FSS mayores que GFS y esto se mantiene hasta el final del periodo pronosticado. Esto indica, para este caso de estudio, que si bien los pronósticos inicializados a partir de análisis con asimilación regional demoran en producir precipitación, luego mantienen la generación de la misma por más tiempo. Para esta métrica no se ven diferencias apreciables entre los distintos umbrales o escalas espaciales.


\begin{figure}
\includegraphics{thesis_files/figure-latex/fss-abi-fcst-1} \caption{FSS calculado sobre la precipitación acumulada en una ventana móvil de 6 horas para umbrales de 1 mm (a y d), 5 mm (b y e) y 25 mm (c y f), en escalas de 10 km (a-c) y 100 km (d-f), para los pronósticos determinísticos SATWND (línea naranja), RAD (línea roja), RAD+ABI (línea verde), ABI (línea turquesa) y GFS (línea negra) entre las 00 UTC del 22 de noviembre y las 12 UTC del 23 de noviembre.}\label{fig:fss-abi-fcst}
\end{figure}
Por otro lado, la Figura \ref{fig:tb} muestra la temperatura de brillo simulada para el canal 13 de GOES-16 en el tiempo inicial del pronóstico a las 00 UTC del 22 de noviembre y luego a las 12 UTC del mismo día cuando el sistema convectivo comienza a desarrollarse y a las 00 UTC del 23 de noviembre cuando el sistema convectivo ya se encuentra en el norte de Argentina y presenta mayor intensidad. A las 00 UTC del 22 de noviembre los pronósticos, particularmente SATWND, desarrollan nubosidad estratiforme y de poco espesor en gran parte del dominio que no está presente en las observaciones (Figura \ref{fig:tb}a-f). Sin embargo los pronósticos inicializados a partir de RAD+ABI y ABI ubica mejor la nubosidad y la generación de cúmulos con desarrollo vertical al norte de la provincia de Salta. Además al incluir información sobre las regiones con cielo despejado, gracias a la asimilación de radianzas de ABI, también representa mejor las zonas despejadas al este del dominio. En este sentido, el pronóstico inicializado a partir de ABI logra reducir en algunos tiempos, la convección espuria y a veces también la precipitación que genera SATWND (no se muestra). Debido a que el GFS no se inicializa con información sobre el contenido de agua de nube no muestra nubosidad para el primer tiempo de pronóstico (Figura \ref{fig:tb}e).

A las 12 UTC, cuando la convección se encuentra cruzando el dominio, todos los pronósticos ubican la nubosidad en la región correcta aunque menos organizada y asociadas a temperaturas de tope más altas que lo observado. De todos los pronósticos SATWND es el que muestra valores de temperatura de brillo más bajos (Figura \ref{fig:tb}g). Por otro lado, la nubosidad que se observa sobre Chaco también está presente en los pronósticos inicializados a partir del análisis con asimilación observaciones de ABI (Figuras \ref{fig:tb}i y j). Sin embargo no logran generar la nubosidad sobre Uruguay que si se muestra en el pronóstico de GFS (Figuras \ref{fig:tb}k).

Finalmente a las 00 UTC del 23 de noviembre la convección se extiende sobre todo el noreste del dominio con temperaturas de brillo por debajo de los -85\(^\circ\)C asociada a los topes nubosos más altos (Figura \ref{fig:tb}r). Si bien todos los pronósticos logran representar las características principales del sistema convectivo la temperatura de brillo es ligeramente mayor y solo en algunas regiones llega a valores por debajo de -85\(^\circ\)C. En este caso GFS es el único pronóstico que representa con mayor detalle la región con menores temperaturas de brillo dividida en 2 subregiones como se observa en la imagen de brillo de GOES-16.


\begin{figure}

\includegraphics{thesis_files/figure-latex/tb-1} \hfill{}

\caption{Temperatura de brillo simulada por CRTM a partir de los pronósticos determinísticos a las 00 y 12 UTC del 22 de noviembre y las 00 UTC del 23 de noviembre. Las observaciones de temperatura de brillo del canal 13 de GOES-16 se muestran en la ultima columna (f, l, r).}\label{fig:tb}
\end{figure}
Para analizar la temperatura de brillo desde un punto de vista cuantitativo, la Figura \ref{fig:fss-tb} muestra el FSS calculado para umbrales de temperatura de brillo de -30\(^\circ\)C, asociado al borde externo de la nubosidad, -70\(^\circ\)C que corresponde a la región de colores rojizos y -85\(^\circ\)C asociado a los topes nubosos. Al igual que con la precipitación, GFS tiene una mejor representación de la temperatura de brillo desde el comienzo del periodo pronosticado pero luego es igual o peor que los pronósticos inicilizados a partir de los análisis. Para el umbral de -30\(^\circ\)C no se ven diferencias apreciables entre los experimentos y tampoco en las distintas escalas. Sin embargo es destacable que los valores de FSS son cercanos a 1. Al observar el umbral de -70\(^\circ\)C, vemos que solo en escalas de 200 km hay valores de FSS para las temperaturas de brillo más bajas lo que indica que las regiones donde la nubosidad es más alta y por lo tanto más fría en los pronósticos no coincide exactamente con las regiones observadas durante las primeras horas de pronóstico. Finalmente, la temperatura de brillo por debajo de -85\(^\circ\)C, es la más difícil de pronosticar porque solo se observa en regiones muy pequeñas dentro de la nube convectiva donde se encuentran las corrientes ascendentes más intensas. Sumado a esto, cabe mencionar las dificultades del modelo de transferencia radiativa para simular estas regiones de baja temperatura. En este caso los pronósticos simulan temperaturas de brillo de esta magnitud pero en regiones muy limitadas y que no coinciden exactamente con las observaciones. Solo al mirar la escala de 200 km se ve que tanto ABI como GFS logran representar topes nubosos intensos. Es importante notar que si bien el FSS de los pronósticos inicializados a partir de la asimilación regional es alto, luego cae rápidamente debido a una disminución en el contenido de condensados en el modelo. Esto ocurre porque si bien los pronósticos son inicializados usando la información de condensados de la media del ensamble del análisis, que surgen de promediar la información de todos los miembros, estos no son campos físicos válidos. El modelo los descarta durante el ininio de la similación y genera los condensados desde cero, requiriendo tiempo de spin up. Este es solo un problema para lo pronósticos determinísicos generados a partir de la media del ensamble y deberían aplicarse técnicas para inicializar estas variables y pruebas adicionales para evitar el spin up del modelo.


\begin{figure}
\includegraphics{thesis_files/figure-latex/fss-tb-1} \caption{FSS calculado sobre la temperatura de brillo simulada para umbrales de -30\(^\circ\)C (a, d, g), -70\(^\circ\)C (b, e, h) y -85\(^\circ\)C (c, f, i), en escalas de 2 km (a-c), 20 km (d-f) y 200 km (g-i), para los pronósticos determinísticos SATWND (línea naranja), RAD (línea roja), RAD+ABI (línea verde), ABI (línea turquesa) y GFS (línea negra) entre las 00 UTC del 22 de noviembre y las 12 UTC del 23 de noviembre.}\label{fig:fss-tb}
\end{figure}
\hypertarget{conclusiones-2}{%
\section{Conclusiones}\label{conclusiones-2}}

En este capítulo evaluamos el impacto que produce la asimilación de observaciones de los canales de vapor de agua del sensor ABI a bordo del satélite geoestacionario GOES-16 tanto en el análisis como en pronósticos determinísticos de alta resolución a corto plazo. La asimilación de estas observaciones es de particular interés en Sudamérica por su alta resolución espacial y frecuencia temporal en comparación con otras fuentes de observación como los sensores a bordo de satélites polares. Además, al momento de la escritura de esta tesis, no se conocen trabajos que utilicen observaciones de sensor ABI en la región, por lo que estos resultados podrían contribuir al desarrollo de los sistemas de pronósticos en Sudamérica.

Para asimilar estas observaciones fue primero necesario incorporar las rutinas necesarias en el sistema de asimilación GSI. Por esto y por que se trata de observaciones nuevas, se generaron experimentos de sensibilidad a la combinación de canales de vapor de agua y al thinning y se estudió la distribución de los errores de las observaciones para determinar la mejor configuración del sistema de asimilación. En este trabajo, entonces, se utilizaron los 3 canales de vapor de agua que aportan información en niveles altos (\textasciitilde350 hPa), medios (\textasciitilde400 hPa) y bajos (\textasciitilde600 hPa) de la atmósfera, con una resolución de 10 km y sin aplicar ningún tipo de corrección de bias.

En primer lugar, de la comparación de los análisis con y sin asimilación de radianzas podemos ver que estas produce impactos en todos los niveles de la atmósfera. En algunos casos estos impactos son directos, es decir, correcciones introducidas a las variables del modelo durante la asimilación, y en otros casos indirectos, es decir causados por procesos físicos que se desencadenan a partir de las correcciones que introducen las observaciones. Si bien los experimentos incrementan ligeramente el bias cálido y seco presente en niveles bajos del modelo, también generan cambios en la temperatura y humedad en niveles medios que son consistentes con un aumento de la inestabilidad previo al desarrollo del sistema convectivo. Los perfiles de las componentes de viento muestran que la iniciación de la convección en los experimentos que asimilan ABI tiene un retraso con respecto del experimento control sin asimilación de radianzas. Sin embargo, la comparación con estimaciones de precipitación muestran que la asimilación de observaciones de ABI mejora sustancialmente el área cubierta por precipitación y la representación general de la precipitación particularmente para umbrales bajos. Esto último podría deberse en parte a una mejor representación de la humedad en niveles medios y altos, algo que resulta evidente al comparar los distintos experimentos con los radiosondeos de RELAMPAGO.

Los pronósticos determinísticos en alta resolución inicializados a partir de la media del análisis de cada experimento y que resuelven los procesos convectivos explicitamente nos permiten analizar el desarrollo de la convección en más detalle. Es notorio que las condiciones iniciales tienen un impacto a lo largo de todo el pronóstico e influyen en el momento de la iniciación de la convección y su intensidad. Si bien analizando el área cubierta por precipitación o la comparación con las estimaciones de precipitación, todos los pronósticos tiene un comportamiento similar, al compararlos con un pronóstico sin asimilación regional, vemos mejoras tanto en la representación de la temperatura y humedad en niveles medios y altos como en la precipitación en las últimas horas de pronóstico. Esto es de alguna manera sorprendente, ya que es esperable que el origen de las mejoras esté asociado a las condiciones iniciales de los pronósticos. Si bien no hay una razón establecida para este comportamiento, es posible que problemas en la inicialización de los condensados en los pronósticos regionales tenga un impacto negativo en las primeras horas de simulación.

Un análisis cualitativo del desarrollo de la convección a partir de la temperatura de brillo del canal de 10.3 \(\mu m\) de GOES-16 muestra que los pronósticos con condiciones iniciales que tienen asimilación de ABI representan muy bien el desarrollo convectivo y también las regiones del dominio donde hay cielos despejados, eliminando en muchos casos celdas convectivas generadas por el modelo en otros experimentos. Por otro lado, del análisis cuantitativo de la temperatura de brillo surge que solo el pronóstico con observaciones de ABI que no incluye radianzas de satélites polares y el pronóstico sin asimilación regional, logran representar los valores de más bajos de temperatura asociados a las ascendentes intensas en los sistemas convectivos.

Finalmente, de la comparación entre los experimentos es importante evaluar el impacto de las observaciones de ABI en presencia o no de otras radianzas de satélites polares. Para este caso de estudio se observó que, en general, la asimilación de radianzas de satélites polares no produjo un impacto por encima de la asimilación de radianzas de ABI cuando se las asimilaba en conjunto. Además, en algunos casos los resultados asociados al experimento que incluye solo radianzas de ABI fueron mejores que en los experimentos donde se incluían también radianzas de satélites polares. Es posible entonces, que la asimilación de radianzas de satélites polares, con todos los desafíos que implica no sea necesaria si se asimilan radianzas de ABI para generar una buena representación de procesos convectivos, aunque esta hipótesis necesita confirmarse mediante el análisis de un numero mayor de casos.

\hypertarget{conclusiones-generales-y-perspectivas-futuras}{%
\chapter{Conclusiones generales y perspectivas futuras}\label{conclusiones-generales-y-perspectivas-futuras}}

\hypertarget{conclusiones-3}{%
\section{Conclusiones}\label{conclusiones-3}}

Este trabajo representa un avance en el desarrollo de la asimilación de datos en escala regional y en consecuencia, en una mejor representación de los procesos convectivos que generan fenómenos con mayor impacto socioeconómico, como lluvias y vientos intensos, granizo y actividad eléctrica. También busca contribuir a la mejora de un sistema de alerta temprana sobre la ocurrencia de fenómenos meteorológicos extremos asociados a convección húmeda profunda. Es por esto que se pone el foco en la asimilación de observaciones de estaciones meteorológicas automáticas (EMA), vientos derivados de satélite y radianzas en cielos despejados, fuentes de observaciones con una gran resolución espacial y frecuencia temporal en una región donde la disponibilidad de observaciones convencionales es baja.

En este trabajo se utilizó el modelo numérico WRF acoplado al sistema de asimilación GSI que es capaz de asimilar observaciones de EMA y radianzas gracias al modelo de transferencias radiativa CRTM que utiliza como operador de las observaciones. Sin embargo hasta ahora la última versión de GSI no era capaz de asimilar radianzas del sensor ABI a bordo del satélite geoestacionario GOES-16 que aporta observaciones con altísima resolución espacial y temporal. La implementación de las rutinas necesarias para ampliar las capacidad del sistema de asimilación a partir de trabajos previos de Lee et al. (2019) y Liu et al. (2019), es uno de los aportes técnicos principales de este trabajo.

Los experimentos de asimilación de datos incluidos en este trabajo generan análisis horarios con 10 km de resolución horizontal utilizando la metodología de filtro de Kalman y un ensamble multifísica de 60 miembros. Además se generaron pronósticos por ensambles y determinísticos, todos para evaluar el impacto de la asimilación de distintos tipos de observaciones en la representación de un sistema convectivo de mesoescala (SCM) que se generó en el centro y norte de Argentina durante el 22 de noviembre de 2018. Este caso de estudio es de particular interés ya que forma parte de los casos de estudio de la campaña de campo RELAMPAGO donde se tomaron mediciones extraordinarias como radiosondeos con alta frecuencia temporal que son útiles para validar los experimentos.

De la comparación entre los experimentos y con observaciones independientes se puede ver que el impacto de las observaciones de EMA es de los más importantes. Una de las principales mejoras que se observa tanto en el análisis como en los pronósticos determinísticos es la representación de la temperatura y humedad en la capa límite, reduciendo el bias cálido y seco del modelo. El impacto de la asimilación también se extiende a la tropósfera principalmente en los periodos de convección más intensa. El aumento en el agua precipitable y la circulación meridional en los análisis condujo al desarrollo de convección más intensa y una mejor representación de la precipitación. Los pronósticos inicializados a partir del análisis, muestran además que el impacto se traslada a los pronósticos y perdura en el tiempo. De hecho, se observan mejoras entre el pronóstico que se inicializa a las 00 y el que se inicializa a las 06 UTC en la representación de la precipitación en parte debido a la asimilación de observaciones entre estos tiempos.

El impacto de los vientos derivados de satélite en los análisis es mínimo en comparación con el impacto que genera la asimilación de observaciones de EMA y solo es apreciable en las componente del viento, principalmente en niveles medios y altos. Esto puede deberse al bajo número de observaciones asimiladas en niveles bajos o a una sobreestimación del error de las observaciones que lleva al sistema de asimilación a darle menos peso a la hora de generar el análisis. Sin embargo, el impacto de estas observaciones es suficiente para generar una mejor representación de la precipitación tanto en el campo preliminar como en los pronósticos por ensambles, particularmente cuando se hace foco en probabilidades de precipitación para umbrales altos. En estos pronósticos y el pronóstico determinísco en alta resolución, el inicio de la precipitación intensa coincide con lo observado pero esta decae ligeramente cuando se la compara con otros pronósticos.

La asimilación de radianzas en cielos despejados generó impactos en todos los niveles de la atmósfera, particularmente en la temperatura y humedad. Esto en parte se debe a que los distintos sensores y canales aportan información sobre estas variables en distintas capas de presión. Si bien se observa un ligero aumento del bias cálido y seco del modelo, reduciendo el impacto de las observaciones de EMA, el análisis genera un desarrollo de la convección adecuado que conduce a un aumento en la precipitación mayor a lo que se observa en otros experimentos. Por otro lado, en los pronósticos por ensambles inicializados a partir de los análisis con asimilación de radianzas de satélites polares subestima la precipitación en comparación con los pronósticos cuyas condiciones iniciales tienen solo vientos derivados de satélite y EMA, aunque la ubica correctamente en el dominio. Esto podría deberse de una disminución de la humedad en niveles bajos que conduce a una atmósfera más estable.

Las observaciones de los canales de vapor de agua de ABI producen impactos también en todos los niveles de la atmósfera y en algunos casos son similares a lo que generan las radianzas de satélites polares, por ejemplo reduciendo la humedad en la capa límite. Sin embargo, también producen impactos en niveles medios que ayudan al desarrollo de mayor inestabilidad y desarrollo de la convección. De la comparación con los radiosondeos de RELAMPAGO se destaca la mejora sustancial en la representación de la humedad por encima de la capa límite donde se asimilan la mayor cantidad de observaciones. La representación de la precipitación tanto en el análisis como en el pronóstico determinístico es muy buena para umbrales bajos y se destaca la mejora respecto de otros experimentos en el área cubierta por precipitación y el desarrollo de la nubosidad. En estos experimentos, sin embargo, la precipitación se inicia ligeramente atrasada y la ubicación del SCM también se ubica por detrás de lo observado, esto puede estar asociado a la ubicación e intensidad del frente frío asociado a este sistema y la circulación en niveles bajos.

Finalmente, los experimentos con asimilación de observaciones de ABI, de satélites polares y aquel que incluye ambas fuentes de información permitieron evaluar el impacto de estas observaciones. Se observó que los resultados asociados a las observaciones de satélites polares fueron iguales o peores en comparación con los que dieron los experimentos que si incluyeron observaciones de ABI. De hecho, la representación de la localización de la precipitación y el desarrollo de la convección fue mucho mejor en los pronósticos inicializados a partir de análisis con asimilación de radianzas en cielo despejado de ABI, eliminando convección espuria que generaron otros experimentos. Teniendo en cuenta esto, y al menos para este caso de estudio, no parece ser necesaria la asimilación de radianzas de satélites polares cuando se asimilan observaciones de ABI.

Algunos de los principales desafíos encontrados durante este trabajo de tesis están asociados al uso del sistema de asimilación GSI, sistema que no había sido utilizado previamente en Argentina. En este sentido fue necesario explorar las capacidades del sistema analizando el código y generando pruebas de asimilación con observaciones convencionales para evaluar su desempeño. Este sistema de asimilación, además, ingesta las observaciones en formato bufr por lo que fue necesario convertir las observaciones de EMA a este formato binario siguiendo las especificaciones definidas por el sistema para poder incluirlas en la asimilación. Trabajar con radianzas de satélites polares en un dominio regional también trae desafíos ya que las observaciones no están disponibles en todos los ciclos de asimilación y esto puede tener impactos importantes en el control de calidad y corrección del bias. Otro aspecto que representó un gran desafío es la asimilación de radianzas del sensor ABI, tanto desde el punto de vista técnico con la implementación de esta capacidad en el sistema de asimilación y la generación de los archivos en formato bufr como en la generación de pruebas y análisis preliminares para definir la mejor configuración del sistema de asimilación. Todo esto, sumado al alto costo computacional que requiere poder generar cada uno de los experimentos y la pruebas preliminares, limitó fuertemente la generación de experimentos para múltiples casos de estudio.

Este trabajo de tesis pone en evidencia la enorme potencialidad de la asimilación regional de observaciones con alta frecuencia temporal y resolución espacial que en algunos casos, no son incluidas en los pronósticos globales como son las observaciones de EMA y que puede contribuir a mejorar los sistema de pronósticos y la representación de fenómenos severos con importantes impactos socioeconómicos. Si bien es necesario seguir avanzando en este área y los desafíos técnicos son muchos, los avances tanto en el área de la asimilación de datos como la capacidad de cómputo en Argentina generan el contexto necesario para continuar el desarrollo de un sistema de pronósticos con asimilación de datos operativo.

\hypertarget{perspectivas-futuras}{%
\section{Perspectivas futuras}\label{perspectivas-futuras}}

La asimilación de datos es un área en continuo avance. En particular la aplicación de la asimilación de datos en escala regional presenta muchos desafíos y la necesidad de seguir explorando y evaluando su impacto en la generación de pronósticos a corto plazo. En este punto es importante destacar una serie de preguntas para las que aún no se encontró respuesta o cuestiones que se desprenden del trabajo realizado y que podrían ser abordadas en trabajos futuros.

Si bien los resultados obtenidos con la asimilación de observaciones de EMA son muy prometedores, es importante continuar estudiando el impacto de la asimilación de estas observaciones en distintos casos de estudio que representen distintas situaciones atmosféricas. Teniendo en cuenta que en algunos casos se observó un impacto negativo asociado a la asimilación de observaciones de EMA, por ejemplo al comparar los pronósticos y análisis con los radiosondeos de RELAMPAGO, es importante avanzar en el desarrollo de un control de calidad robusto para estas observaciones previo a su asimilación. Por otro lado, es importante evaluar la magnitud de los errores de las observaciones usados ya que estos podrían ser mayor a lo estimado actualmente. Por un lado es posible que estas estaciones tengan menos controles de calidad, aumentando el error de las observaciones; y por el otro, que la alta frecuencia temporal y resolución espacial de la observaciones genere que sus errores estén correlacionados entre si.

Sin embargo, para extender el uso de estas observaciones en este y otros ámbitos es importante establecer vínculos con las instituciones y organismos que generan las observaciones y aquellos que las utilizan para tareas de investigación u operativas. Por ejemplo, sería muy provechoso que las observaciones pudieran ser utilizadas operacionalmente por el Servicio Meteorológico Nacional y en un futuro sean incluidas en los sistemas de información de la Organización Mundial de Meteorología para que puedan ser utilizadas a escala regional y global. Este trabajo es un ejemplo de cómo el uso de observaciones que no forman parte de las redes oficiales de información puede mejorar los pronósticos en la región y esperamos que ayude a sentar las bases necesarias para generar una política de datos que permita el intercambio de información entre las organizaciones.

También es sorprendente el impacto positivo de los vientos derivados de satélite en los pronósticos de precipitación cuando en el análisis no se ven grandes diferencias. Será importante, entonces, entender mejor los impactos que producen estas observaciones y extender su uso a otros casos de estudio. Estas observaciones podrían ser de particular interés en el pronóstico de SCM donde la presencia de nubosidad la región permite obtener mayor cantidad de estimaciones de viento. Otro aspecto que sería deseable continuar estudiando es el impacto de las radianzas de satélites polares, particularmente de los sensores multiespectrales. Estos sensores aportan un gran volumen de observaciones en distintos niveles verticales que puede ser clave para representar distintos proceso atmosféricos, sin embargo es importante estudiar las características de los canales en detalle a fin de entender el impacto que generan y evitar la degradación del análisis por efectos de la superficie o correlaciones entre sus errores.

La asimilación de observaciones de ABI mostró resultados muy prometedores en el desarrollo de la convección y la representación de la precipitación. Sería interesante en el futuro continuar realizando experimentos que aíslen el impacto de cada canal de vapor de agua para entender mejor el rol que juega cada uno. También es importante evaluar nuevamente la necesidad o no de realizar corrección de bias para estas observaciones y analizar posibles correlaciones entre los errores observacionales de los distintos canales. Un tema no desarrollado en este trabajo es la sensibilidad a la localización vertical de las radianzas. Estas observaciones abarcan toda la troposfera y parte de la estratosfera por lo que seria importante estudiar si cambios en la localización vertical, es decir, la capa de la atmósfera donde cada observación tiene influencia produce mejores resultados. Por otro lado, la asimilación de observaciones \emph{all sky}, es decir, tanto en cielos despejados como nublados es el paso natural a seguir, que si bien introduce desafíos, particularmente en la estimación de los errores de las observaciones, muchas veces no lineales, podría mejorar sustancialmente la representación de la convección en la región.

Es importante destacar un desafío que plantean el gran volumen de observaciones, particularmente de satélites, y la posible correlación entre sus errores a los sistemas de asimilación actuales. Si bien, estas correlaciones pueden ser estimadas a partir de los sistemas de asimilación (por ejemplo utilizando las estadísticas de Desroziers (Desroziers et al., 2005) para estimar la estructura de las correlaciones de los errores observacionales), el desafío también está en cómo considerar explicitamente las correlaciones tanto espaciales como entre los canales en los sistemas de asimilación.

Asimismo, sería deseable avanzar con la asimilación de otras fuentes de observaciones como las de radio ocultamiento de GPS que ya son utilizados en por Centro de Previsão de Tempo e Estudos Climáticos en Brasil (Banos et al., 2019) u observaciones de radares que ya cuenta con importantes avances en el país (Maldonado et al., 2020, 2021). El uso de otras metodologías para estimar el impacto relativo de las distintas fuentes de información, por ejemplo el \emph{Ensemble Forecast Sensitivity to Observations} (EFSO, Hotta et al., 2017), pueden contribuir a cuantificar el impacto de las observaciones con mayor eficiencia. Sumado a esto, surge la necesidad de explorar la asimilación multiescala que permite la asimilación diferencial de observaciones, de acuerdo a la escala de lo que representan procesos que representan (de Moraes et al., 2020), mejorando la representación de procesos en distintas escalas.

Por último, es importante mencionar el proyecto Pronóstico y Alerta de Eventos de Inundaciones Repentinas (PREVENIR) que comenzó en junio de 2022. Este proyecto contempla la colaboración entre instituciones de investigación de Japón y organismos científico-técnicos de
Argentina con el objetivo de desarrollar un sistema de pronóstico hidrometeorológico a corto plazo que busca mejorar el alerta temprana de eventos asociados a lluvias intensas en áreas densamente pobladas de Argentina. Entre otras cosas, el proyecto aplicará la asimilación de datos para generar pronósticos a corto plazo utilizando observaciones de EMA, radar y radianzas. En este sentido, la experiencia y conocimientos ganados a lo largo de este trabajo de tesis será de gran utilidad en el desarrollo del proyecto PREVENIR.

\backmatter

\hypertarget{referencias}{%
\chapter*{Referencias}\label{referencias}}
\addcontentsline{toc}{chapter}{Referencias}

\markboth{Referencias}{Referencias}

\noindent

\setlength{\parindent}{-0.20in}

\hypertarget{refs}{}
\leavevmode\hypertarget{ref-aksoy2010}{}%
Aksoy, A., Dowell, D.C., and Snyder, C., 2010. A Multicase Comparative Assessment of the Ensemble Kalman Filter for Assimilation of Radar Observations. Part II: Short-Range Ensemble Forecasts. Monthly Weather Review, 138, 4, 1273--1292.

\leavevmode\hypertarget{ref-allaire2019}{}%
Allaire, J., Horner, J., Xie, Y., Marti, V., and Porte, N., 2019. Markdown: Render markdown with the c library 'sundown'.

\leavevmode\hypertarget{ref-andersson1991}{}%
Andersson, E., Hollingsworth, A., Kelly, G., Lönnberg, P., Pailleux, J., and Zhang, Z., 1991. Global Observing System Experiments on Operational Statistical Retrievals of Satellite Sounding Data. Monthly Weather Review, 119, 8, 1851--1865.

\leavevmode\hypertarget{ref-arruti2021}{}%
Arruti, A., Maldonado, P., Rugna, M., Sacco, M., Ruiz, J.J., and Vidal, L., 2021. Sistema de Control de Calidad de Datos de Radar en el Servicio Meteorológico Nacional. Parte I: Descripción del algoritmo.

\leavevmode\hypertarget{ref-bae2022}{}%
Bae, J.-H., and Min, K.-H., 2022. Forecast Characteristics of Radar Data Assimilation Based on the Scales of Precipitation Systems. Remote Sensing, 14, 3, 3, 605.

\leavevmode\hypertarget{ref-banos2019}{}%
Banos, I.H., Sapucci, L.F., Cucurull, L., Bastarz, C.F., and Silveira, B.B., 2019. Assimilation of GPSRO Bending Angle Profiles into the Brazilian Global Atmospheric Model. Remote Sensing, 11, 3, 3, 256.

\leavevmode\hypertarget{ref-bao2015}{}%
Bao, Y., Xu, J., Powell Jr., A.M., Shao, M., Min, J., and Pan, Y., 2015. Impacts of AMSU-A, MHS and IASI data assimilation on temperature and humidity forecasts with GSI--WRF over the western United States. Atmospheric Measurement Techniques, 8, 10, 4231--4242.

\leavevmode\hypertarget{ref-barton2021}{}%
Barton, N., Metzger, E.J., Reynolds, C.A., Ruston, B., Rowley, C., Smedstad, O.M., Ridout, J.A., Wallcraft, A., Frolov, S., and Hogan, P. and others, 2021. The Navy's Earth System Prediction Capability: A New Global Coupled Atmosphere-Ocean-Sea Ice Prediction System Designed for Daily to Subseasonal Forecasting. Earth and Space Science, 8, 4, e2020EA001199.

\leavevmode\hypertarget{ref-baucemachado2017}{}%
Bauce Machado, V., gustavo de goncalves, luis, Vendrasco, E., Sinhori, N., Herdies, D., Sapucci, L., Levien, C., Quadro, M., Rodrigues, T., and Cardoso, C. and others, 2017. Investigating the impacts of convective scale hazardous weather events in Santa Catarina State through the CPTEC/INPE local data assimilation system. In. Presented at the Seventh International WMO Symposium on Data Assimilation.

\leavevmode\hypertarget{ref-bauer2010}{}%
Bauer, P., Geer, A.J., Lopez, P., and Salmond, D., 2010. Direct 4D-Var assimilation of all-sky radiances. Part I: Implementation. Quarterly Journal of the Royal Meteorological Society, 136, 652, 1868--1885.

\leavevmode\hypertarget{ref-boukabara2011}{}%
Boukabara, S.-A., Garrett, K., Chen, W., Iturbide-Sanchez, F., Grassotti, C., Kongoli, C., Chen, R., Liu, Q., Yan, B., and Weng, F. and others, 2011. MiRS: An All-Weather 1DVAR Satellite Data Assimilation and Retrieval System. IEEE Transactions on Geoscience and Remote Sensing, 49, 9, 3249--3272. Presented at the IEEE Transactions on Geoscience and Remote Sensing.

\leavevmode\hypertarget{ref-bousquet2008}{}%
Bousquet, O., Montmerle, T., and Tabary, P., 2008. Using operationally synthesized multiple-Doppler winds for high resolution horizontal wind forecast verification: OPERATIONAL DOPPLER RADAR NETWORKS. Geophysical Research Letters, 35, 10.

\leavevmode\hypertarget{ref-brier1950}{}%
Brier, G.W., 1950. VERIFICATION OF FORECASTS EXPRESSED IN TERMS OF PROBABILITY. Monthly Weather Review, 78, 1, 1--3.

\leavevmode\hypertarget{ref-campitelli2020}{}%
Campitelli, E., 2020, April. metR: Tools for Easier Analysis of Meteorological Fields.

\leavevmode\hypertarget{ref-candille2007}{}%
Candille, G., Côté, C., Houtekamer, P.L., and Pellerin, G., 2007. Verification of an Ensemble Prediction System against Observations. Monthly Weather Review, 135, 7, 2688--2699.

\leavevmode\hypertarget{ref-carrassi2018}{}%
Carrassi, A., Bocquet, M., Bertino, L., and Evensen, G., 2018. Data assimilation in the geosciences: An overview of methods, issues, and perspectives. WIREs Climate Change, 9, 5, e535.

\leavevmode\hypertarget{ref-casaretto2022}{}%
Casaretto, G., Dillon, M.E., Salio, P., Skabar, Y.G., Nesbitt, S.W., Schumacher, R.S., García, C.M., and Catalini, C., 2022. High-Resolution NWP Forecast Precipitation Comparison over Complex Terrain of the Sierras de Córdoba during RELAMPAGO-CACTI. Weather and Forecasting, 37, 2, 241--266.

\leavevmode\hypertarget{ref-chang2017}{}%
Chang, W., Jacques, D., Fillion, L., and Baek, S.-J., 2017. Assimilation of Hourly Surface Observations with the Canadian High-Resolution Ensemble Kalman Filter. Atmosphere-Ocean, 55, 4-5, 247--263.

\leavevmode\hypertarget{ref-chen2001}{}%
Chen, F., and Dudhia, J., 2001. Coupling an Advanced Land Surface--Hydrology Model with the Penn State--NCAR MM5 Modeling System. Part I: Model Implementation and Sensitivity. Monthly Weather Review, 129, 4, 569--585.

\leavevmode\hypertarget{ref-chen2015}{}%
Chen, Q., Fan, J., Hagos, S., Gustafson, W.I., and Berg, L.K., 2015. Roles of wind shear at different vertical levels: Cloud system organization and properties. Journal of Geophysical Research: Atmospheres, 120, 13, 6551--6574.

\leavevmode\hypertarget{ref-chen2016}{}%
Chen, X., Zhao, K., Sun, J., Zhou, B., and Lee, W.-C., 2016. Assimilating surface observations in a four-dimensional variational Doppler radar data assimilation system to improve the analysis and forecast of a squall line case. Advances in Atmospheric Sciences, 33, 10, 1106--1119.

\leavevmode\hypertarget{ref-cherubini2006}{}%
Cherubini, T., Businger, S., Velden, C., and Ogasawara, R., 2006. The Impact of Satellite-Derived Atmospheric Motion Vectors on Mesoscale Forecasts over Hawaii. Monthly Weather Review, 134, 7, 2009--2020.

\leavevmode\hypertarget{ref-clark2017}{}%
Clark, A.J., 2017. Generation of Ensemble Mean Precipitation Forecasts from Convection-Allowing Ensembles. Weather and Forecasting, 32, 4, 1569--1583.

\leavevmode\hypertarget{ref-clark2009}{}%
Clark, A.J., Gallus, W.A., Xue, M., and Kong, F., 2009. A Comparison of Precipitation Forecast Skill between Small Convection-Allowing and Large Convection-Parameterizing Ensembles. Weather and Forecasting, 24, 4, 1121--1140.

\leavevmode\hypertarget{ref-collard2007}{}%
Collard, A.D., 2007. Selection of IASI channels for use in numerical weather prediction: SELECTION OF IASI CHANNELS FOR NWP. Quarterly Journal of the Royal Meteorological Society, 133, 629, 1977--1991.

\leavevmode\hypertarget{ref-Cheyenne2019}{}%
Computational and Information Systems Laboratory, 2019. Cheyenne: HPE/SGI ICE XA System (University Community Computing). National Center for Atmospheric Research Boulder, CO.

\leavevmode\hypertarget{ref-crews2021}{}%
Crews, A., Blackwell, W.J., Leslie, R.V., Grant, M., Osaretin, I.A., DiLiberto, M., Milstein, A., Leroy, S., Gagnon, A., and Cahoy, K., 2021. Initial Radiance Validation of the Microsized Microwave Atmospheric Satellite-2A. IEEE Transactions on Geoscience and Remote Sensing, 59, 4, 2703--2714. Presented at the IEEE Transactions on Geoscience and Remote Sensing.

\leavevmode\hypertarget{ref-cutraro2021}{}%
Cutraro, F., Galligani, V.S., and Skabar, Y.G., 2021. Evaluation of synthetic satellite images computed from radiative transfer models over a region of South America using WRF and GOES-13/16 observations. Quarterly Journal of the Royal Meteorological Society, 147, 738, 2988--3003.

\leavevmode\hypertarget{ref-deelia2017}{}%
de Elía, R., Vidal, L., and Lohigorry, P., 2017. El SMN y la red argentina de radares meteorológicos (\url{http://hdl.handle.net/20.500.12160/625}).

\leavevmode\hypertarget{ref-demoraes2020}{}%
de Moraes, R.J., Hajibeygi, H., and Jansen, J.D., 2020. A multiscale method for data assimilation. Computational Geosciences, 24, 2, 425--442.

\leavevmode\hypertarget{ref-desroziers2005}{}%
Desroziers, G., Berre, L., Chapnik, B., and Poli, P., 2005. Diagnosis of observation, background and analysis-error statistics in observation space. Quarterly Journal of the Royal Meteorological Society, 131, 613, 3385--3396.

\leavevmode\hypertarget{ref-dillon2017}{}%
Dillon, L.M.E., 2017. Asimilación de datos reales a escala regional en Argentina.

\leavevmode\hypertarget{ref-dillon2021}{}%
Dillon, M.E., Maldonado, P., Corrales, P., Skabar, Y.G., Ruiz, J., Sacco, M., Cutraro, F., Mingari, L., Matsudo, C., and Vidal, L. and others, 2021. A rapid refresh ensemble based data assimilation and forecast system for the RELAMPAGO field campaign. Atmospheric Research, 105858.

\leavevmode\hypertarget{ref-dowle2020}{}%
Dowle, M., and Srinivasan, A., 2020, July. Data.Table: Extension of 'data.frame'.

\leavevmode\hypertarget{ref-sondeos}{}%
Earth Observing Laboratory, U. -, 2020. Multi-network composite highest resolution radiosonde data. Version 1.3. UCAR/NCAR - earth observing laboratory.

\leavevmode\hypertarget{ref-english2000}{}%
English, S.J., Renshaw, R.J., Dibben, P.C., Smith, A.J., Rayer, P.J., Poulsen, C., Saunders, F.W., and Eyre, J.R., 2000. A comparison of the impact of TOVS arid ATOVS satellite sounding data on the accuracy of numerical weather forecasts. Quarterly Journal of the Royal Meteorological Society, 126, 569, 2911--2931.

\leavevmode\hypertarget{ref-evensen2009}{}%
Evensen, G., 2009. Data Assimilation, Springer,

\leavevmode\hypertarget{ref-eyre2020}{}%
Eyre, J.R., English, S.J., and Forsythe, M., 2020. Assimilation of satellite data in numerical weather prediction. Part I: The early years. Quarterly Journal of the Royal Meteorological Society, 146, 726, 49--68.

\leavevmode\hypertarget{ref-eyre2022}{}%
Eyre, J.R., Bell, W., Cotton, J., English, S.J., Forsythe, M., Healy, S.B., and Pavelin, E.G., 2022. Assimilation of satellite data in numerical weather prediction. Part II: Recent years. Quarterly Journal of the Royal Meteorological Society, 148, 743, 521--556.

\leavevmode\hypertarget{ref-ferreira2017}{}%
Ferreira, R.C., Herdies, D.L., Vendrasco, É.P., Beneti, C.A.A., and Biscaro, T.S., 2017. Impacto da Assimilação de Dados de Radar em Sistemas Convectivos de Mesoescala: Um Estudo de Caso. Revista Brasileira de Meteorologia, 32, 3, 447--458.

\leavevmode\hypertarget{ref-ferreira2020}{}%
Ferreira, R.C., Alves Júnior, M.P., Vendrasco, éder P., Aravéquia, J.A., Nolasco Junior, L.R., Biscaro, T.S., Ferreira, R.C., Alves Júnior, M.P., Vendrasco, éder P., and Aravéquia, J.A. and others, 2020. The Impact of Microphysics Parameterization on Precipitation Forecast Using Radar Data Assimilation. Revista Brasileira de Meteorologia, 35, 1, 123--134.

\leavevmode\hypertarget{ref-gao2015}{}%
Gao, F., Huang, X.-Y., Jacobs, N.A., and Wang, H., 2015. Assimilation of wind speed and direction observations: Results from real observation experiments. Tellus A: Dynamic Meteorology and Oceanography, 67, 1, 27132.

\leavevmode\hypertarget{ref-garcia2019}{}%
Garcia, F., Ruiz, J., Salio, P., Bechis, H., and Nesbitt, S., 2019. Argentina mesonet data. Version 1.1. UCAR/NCAR - earth observing laboratory.

\leavevmode\hypertarget{ref-garciaskabar1997}{}%
García Skabar, Y., 1997. Análisis objetivo regional para inicializar un modelo de diez niveles en forma operativa. Tesis de licenciatura en ciencias de la atmósfera.

\leavevmode\hypertarget{ref-gasperoni2018}{}%
Gasperoni, N.A., Wang, X., Brewster, K.A., and Carr, F.H., 2018. Assessing Impacts of the High-Frequency Assimilation of Surface Observations for the Forecast of Convection Initiation on 3 April 2014 within the Dallas--Fort Worth Test Bed. Monthly Weather Review, 146, 11, 3845--3872.

\leavevmode\hypertarget{ref-goncalvesdegoncalves2015}{}%
Goncalves de Goncalves, L.G., Sapucci, L., Vendrasco, E., de Mattos, J.G., Ferreira, C., Khamis, E., and Cruz, N., 2015. A rapid update data assimilation cycle over South America using 3DVar and EnKF. In The 20th International TOVS Study Conference (ITSC-20).

\leavevmode\hypertarget{ref-grell2013}{}%
Grell, G.A., and Freitas, S.R., 2013. A scale and aerosol aware stochastic convective parameterization for weather and air quality modeling. Atmospheric Chemistry and Physics Discussions, 13, 9, 23845--23893.

\leavevmode\hypertarget{ref-gustafsson2018}{}%
Gustafsson, N., Janjić, T., Schraff, C., Leuenberger, D., Weissmann, M., Reich, H., Brousseau, P., Montmerle, T., Wattrelot, E., and Bučánek, A. and others, 2018. Survey of data assimilation methods for convective‐scale numerical weather prediction at operational centres. Quarterly Journal of the Royal Meteorological Society, 144, 713, 1218--1256.

\leavevmode\hypertarget{ref-ha2014}{}%
Ha, S.-Y., and Snyder, C., 2014. Influence of Surface Observations in Mesoscale Data Assimilation Using an Ensemble Kalman Filter. Monthly Weather Review, 142, 4, 1489--1508.

\leavevmode\hypertarget{ref-heidinger2013}{}%
Heidinger, A., and Straka III, W.C., 2013, June 11. ABI Cloud Mask.

\leavevmode\hypertarget{ref-era5pressure}{}%
Hersbach, H., Bell, B., Berrisford, P., Biavati, G., Horányi, A., Muñoz Sabater, J., Nicolas, J., Peubey, C., Radu, R., and Rozum, I. and others, 2018. ERA5 hourly data on pressure levels from 1959 to present. Copernicus Climate Change Service (C3S) Climate Data Store (CDS), (Accessed on \(<\)08-08-2022\(>\)).

\leavevmode\hypertarget{ref-hobouchian2018}{}%
Hobouchian, M.P., García Skabar, Y., Salio, P., Viale, M., and Matsudo, C.M., 2018. Evaluación de estimaciones de precipitación por satélite en el sur de Sudamérica.

\leavevmode\hypertarget{ref-hohenegger2007}{}%
Hohenegger, C., and Schar, C., 2007. Atmospheric Predictability at Synoptic Versus Cloud-Resolving Scales. Bulletin of the American Meteorological Society, 88, 11, 1783--1794.

\leavevmode\hypertarget{ref-honda2018}{}%
Honda, T., Miyoshi, T., Lien, G.-Y., Nishizawa, S., Yoshida, R., Adachi, S.A., Terasaki, K., Okamoto, K., Tomita, H., and Bessho, K., 2018. Assimilating All-Sky Himawari-8 Satellite Infrared Radiances: A Case of Typhoon Soudelor (2015). Monthly Weather Review, 146, 1, 213--229.

\leavevmode\hypertarget{ref-hong2006}{}%
Hong, S.-Y., Noh, Y., and Dudhia, J., 2006. A New Vertical Diffusion Package with an Explicit Treatment of Entrainment Processes. Monthly Weather Review, 134, 9, 2318--2341.

\leavevmode\hypertarget{ref-hong2006a}{}%
Hong, S.-Y., Kim, J.-H., Lim, J.-o., and Dudhia, J., 2006. The WRF Single Moment 6-Class Microphysics Scheme (WSM6). Journal of the Korean Meteorological Society, 42, 129--151.

\leavevmode\hypertarget{ref-hotta2017}{}%
Hotta, D., Chen, T.-C., Kalnay, E., Ota, Y., and Miyoshi, T., 2017. Proactive QC: A Fully Flow-Dependent Quality Control Scheme Based on EFSO. Monthly Weather Review, 145, 8, 3331--3354.

\leavevmode\hypertarget{ref-hu2021}{}%
Hu, H., and Han, Y., 2021. Comparing the Thermal Structures of Tropical Cyclones Derived From Suomi NPP ATMS and FY-3D Microwave Sounders. IEEE Transactions on Geoscience and Remote Sensing, 59, 10, 8073--8083. Presented at the IEEE Transactions on Geoscience and Remote Sensing.

\leavevmode\hypertarget{ref-hu2019}{}%
Hu, H., Weng, F., Han, Y., and Duan, Y., 2019. Remote Sensing of Tropical Cyclone Thermal Structure from Satellite Microwave Sounding Instruments: Impacts of Background Profiles on Retrievals. Journal of Meteorological Research, 33, 1, 89--103.

\leavevmode\hypertarget{ref-hu2018}{}%
Hu, M., Ge, G., Zhou, C., Stark, D., Shao, H., Newman, K., Beck, J., and Zhang, X., 2018. Grid-point Statistical Interpolation (GSI) User's Guide Version 3.7, Developmental Testbed Center, p. 149.

\leavevmode\hypertarget{ref-huffman2018}{}%
Huffman, G., Bolvin, D., Braithwaite, D., Hsu, K., Joyce, R., Kidd, C., Nelkin, E., Sorooshian, S., Tan, J., and Xie, P., 2018. NASA Global Precipitation Measurement (GPM) Integrated Multi-satellitE Retrievals for GPM (IMERG), National Aeronautics and Space Administration (NASA), p. 35.

\leavevmode\hypertarget{ref-hunt2007}{}%
Hunt, B.R., Kostelich, E.J., and Szunyogh, I., 2007. Efficient data assimilation for spatiotemporal chaos: A local ensemble transform Kalman filter. Physica D: Nonlinear Phenomena, 230, 1-2, 112--126.

\leavevmode\hypertarget{ref-iacono2008}{}%
Iacono, M.J., Delamere, J.S., Mlawer, E.J., Shephard, M.W., Clough, S.A., and Collins, W.D., 2008. Radiative forcing by long-lived greenhouse gases: Calculations with the AER radiative transfer models. Journal of Geophysical Research, 113, D13, D13103.

\leavevmode\hypertarget{ref-iacovazzi2020}{}%
Iacovazzi, R., Lin, L., Sun, N., and Liu, Q., 2020. NOAA Operational Microwave Sounding Radiometer Data Quality Monitoring and Anomaly Assessment Using COSMIC GNSS Radio-Occultation Soundings. Remote Sensing, 12, 5, 5, 828.

\leavevmode\hypertarget{ref-ismay2022}{}%
Ismay, C., and Solomon, N., 2022. Thesisdown: An updated r markdown thesis template using the bookdown package.

\leavevmode\hypertarget{ref-janjic2018}{}%
Janjić, T., Bormann, N., Bocquet, M., Carton, J.A., Cohn, S.E., Dance, S.L., Losa, S.N., Nichols, N.K., Potthast, R., and Waller, J.A. and others, 2018. On the representation error in data assimilation. Quarterly Journal of the Royal Meteorological Society, 144, 713, 1257--1278.

\leavevmode\hypertarget{ref-janjic1994}{}%
Janjić, Z.I., 1994. The Step-Mountain Eta Coordinate Model: Further Developments of the Convection, Viscous Sublayer, and Turbulence Closure Schemes. Monthly Weather Review, 122, 5, 927--945.

\leavevmode\hypertarget{ref-jones2013}{}%
Jones, T.A., Otkin, J.A., Stensrud, D.J., and Knopfmeier, K., 2013. Assimilation of Satellite Infrared Radiances and Doppler Radar Observations during a Cool Season Observing System Simulation Experiment. Monthly Weather Review, 141, 10, 3273--3299.

\leavevmode\hypertarget{ref-jones2014}{}%
---------, 2014. Forecast Evaluation of an Observing System Simulation Experiment Assimilating Both Radar and Satellite Data. Monthly Weather Review, 142, 1, 107--124.

\leavevmode\hypertarget{ref-jones2020}{}%
Jones, T.A., Skinner, P., Yussouf, N., Knopfmeier, K., Reinhart, A., Wang, X., Bedka, K., Smith, W., and Palikonda, R., 2020. Assimilation of GOES-16 Radiances and Retrievals into the Warn-on-Forecast System. Monthly Weather Review, 148, 5, 1829--1859.

\leavevmode\hypertarget{ref-kain2004}{}%
Kain, J.S., 2004. The Kain--Fritsch Convective Parameterization: An Update. JOURNAL OF APPLIED METEOROLOGY, 43, 12.

\leavevmode\hypertarget{ref-kalnay2002}{}%
Kalnay, E., 2002, November 6. Atmospheric Modeling, Data Assimilation and Predictability (\url{https://www.cambridge.org/highereducation/books/atmospheric-modeling-data-assimilation-and-predictability/C5FD207439132836E85027754CE9BC1A}).

\leavevmode\hypertarget{ref-kelly1978}{}%
Kelly, G.a.M., Mills, G.A., and Smith, W.L., 1978. Impact of Nimbus-6 Temperature Soundings on Australian Region Forecasts. Bulletin of the American Meteorological Society, 59, 4, 393--406.

\leavevmode\hypertarget{ref-kleist2009}{}%
Kleist, D.T., Parrish, D.F., Derber, J.C., Treadon, R., Wu, W.-S., and Lord, S., 2009. Introduction of the GSI into the NCEP Global Data Assimilation System. Weather and Forecasting, 24, 6, 1691--1705.

\leavevmode\hypertarget{ref-lazarus2010}{}%
Lazarus, S.M., Splitt, M.E., Lueken, M.D., Ramachandran, R., Li, X., Movva, S., Graves, S.J., and Zavodsky, B.T., 2010. Evaluation of Data Reduction Algorithms for Real-Time Analysis. Weather and Forecasting, 25, 3, 837--851.

\leavevmode\hypertarget{ref-lee2019}{}%
Lee, J.-R., Li, J., Li, Z., Wang, P., and Li, J., 2019. ABI Water Vapor Radiance Assimilation in a Regional NWP Model by Accounting for the Surface Impact. Earth and Space Science, 6, 9, 1652--1666.

\leavevmode\hypertarget{ref-lim2014}{}%
Lim, A.H., Jung, J.A., Huang, H.-L.A., Ackerman, S.A., and Otkin, J.A., 2014. Assimilation of clear sky Atmospheric Infrared Sounder radiances in short-term regional forecasts using community models. Journal of Applied Remote Sensing, 8, 1, 083655.

\leavevmode\hypertarget{ref-lin2017a}{}%
Lin, H., Weygandt, S.S., Benjamin, S.G., and Hu, M., 2017. Satellite Radiance Data Assimilation within the Hourly Updated Rapid Refresh. Weather and Forecasting, 32, 4, 1273--1287.

\leavevmode\hypertarget{ref-liu2019}{}%
Liu, H., Collard, A., Derber, J., Nebuda, S., and Jung, J.A., 2019. Evaluation of GOES-16 Clear-sky Radiance Data and Preliminary Assimilation Results at NCEP, 2 pags.

\leavevmode\hypertarget{ref-liu2008}{}%
Liu, Q., Weng, F., Han, Y., and van Delst, P., 2008. Community Radiative Transfer Model for Scattering Transfer and Applications. In IGARSS 2008 - 2008 IEEE International Geoscience and Remote Sensing Symposium Vol. 4, pp. IV--1193--IV--1196. Presented at the IGARSS 2008 - 2008 IEEE International Geoscience and Remote Sensing Symposium.

\leavevmode\hypertarget{ref-lorenz1965}{}%
Lorenz, E.N., 1965. A study of the predictability of a 28-variable atmospheric model. Tellus, 17, 3, 321--333.

\leavevmode\hypertarget{ref-maejima2019}{}%
Maejima, Y., Miyoshi, T., Kunii, M., Seko, H., and Sato, K., 2019. Impact of Dense and Frequent Surface Observations on 1-Minute-Update Severe Rainstorm Prediction: A Simulation Study. Journal of the Meteorological Society of Japan. Ser. II, 97, 1, 253--273.

\leavevmode\hypertarget{ref-maldonado2020}{}%
Maldonado, P., Ruiz, J., and Saulo, C., 2020. Parameter Sensitivity of the WRF--LETKF System for Assimilation of Radar Observations: Imperfect-Model Observing System Simulation Experiments. Weather and Forecasting, 35, 4, 1345--1362.

\leavevmode\hypertarget{ref-maldonado2021}{}%
---------, 2021. Sensitivity to Initial and Boundary Perturbations in Convective-Scale Ensemble-Based Data Assimilation: Imperfect-Model OSSEs. SOLA, 17, 0, 96--102.

\leavevmode\hypertarget{ref-markowski2010}{}%
Markowski, P., and Richardson, Y., 2010. Organization of Isolated Convection. In Mesoscale Meteorology in Midlatitudes pp. 201--244.

\leavevmode\hypertarget{ref-matsudo2021}{}%
Matsudo, C., Salles, M.A., and García Skabar, Y., 2021. Verificación de los pronósticos del esquema determinístico del modelo WRF para el año 2020.

\leavevmode\hypertarget{ref-nakanishi2009}{}%
Nakanishi, M., and Niino, H., 2009. Development of an Improved Turbulence Closure Model for the Atmospheric Boundary Layer. Journal of the Meteorological Society of Japan, 87, 5, 895--912.

\leavevmode\hypertarget{ref-cisl_rda_ds084.1}{}%
National Centers for Environmental Prediction, National Weather Service, NOAA, U.S. Department of Commerce, 2015. NCEP GFS 0.25 degree global forecast grids historical archive.

\leavevmode\hypertarget{ref-necker2020}{}%
Necker, T., Geiss, S., Weissmann, M., Ruiz, J., Miyoshi, T., and Lien, G., 2020. A convective‐scale 1,000‐member ensemble simulation and potential applications. Quarterly Journal of the Royal Meteorological Society, 146, 728, 1423--1442.

\leavevmode\hypertarget{ref-nesbitt2021}{}%
Nesbitt, S.W., Salio, P.V., Ávila, E., Bitzer, P., Carey, L., Chandrasekar, V., Deierling, W., Dominguez, F., Dillon, M.E., and Garcia, C.M. and others, 2021. A storm safari in Subtropical South America: Proyecto RELAMPAGO. Bulletin of the American Meteorological Society, -1, aop, 1--64.

\leavevmode\hypertarget{ref-ohring1979}{}%
Ohring, G., 1979. Impact of Satellite Temperature Sounding Data on Weather Forecasts. Bulletin of the American Meteorological Society, 60, 10, 1142--1147.

\leavevmode\hypertarget{ref-orlanski1975}{}%
Orlanski, I., 1975. A Rational Subdivision of Scales for Atmospheric Processes. Bulletin of the American Meteorological Society, 56, 5, 527--530.

\leavevmode\hypertarget{ref-ouaraini2015}{}%
Ouaraini, R.E., Berre, L., Fischer, C., and Sayouty, E.H., 2015. Sensitivity of regional ensemble data assimilation spread to perturbations of lateral boundary conditions. Tellus A: Dynamic Meteorology and Oceanography, 67, 1, 28502.

\leavevmode\hypertarget{ref-patil2001}{}%
Patil, D.J., Hunt, B.R., Kalnay, E., Yorke, J.A., and Ott, E., 2001. Local Low Dimensionality of Atmospheric Dynamics. Physical Review Letters, 86, 26, 5878--5881.

\leavevmode\hypertarget{ref-pondeca2011}{}%
Pondeca, M.S.F.V.D., Manikin, G.S., DiMego, G., Benjamin, S.G., Parrish, D.F., Purser, R.J., Wu, W.-S., Horel, J.D., Myrick, D.T., and Lin, Y. and others, 2011. The Real-Time Mesoscale Analysis at NOAA's National Centers for Environmental Prediction: Current Status and Development. Weather and Forecasting, 26, 5, 593--612.

\leavevmode\hypertarget{ref-purser2003a}{}%
Purser, R.J., Wu, W.-S., Parrish, D.F., and Roberts, N.M., 2003a. Numerical Aspects of the Application of Recursive Filters to Variational Statistical Analysis. Part II: Spatially Inhomogeneous and Anisotropic General Covariances. Monthly Weather Review, 131, 8, 1536--1548.

\leavevmode\hypertarget{ref-purser2003}{}%
---------, 2003b. Numerical Aspects of the Application of Recursive Filters to Variational Statistical Analysis. Part I: Spatially Homogeneous and Isotropic Gaussian Covariances. Monthly Weather Review, 131, 8, 1524--1535.

\leavevmode\hypertarget{ref-rabier2002}{}%
Rabier, F., Fourrié, N., Chafäi, D., and Prunet, P., 2002. Channel selection methods for Infrared Atmospheric Sounding Interferometer radiances. Quarterly Journal of the Royal Meteorological Society, 128, 581, 1011--1027.

\leavevmode\hypertarget{ref-rcoreteam2020}{}%
R Core Team, 2020. R: A language and environment for statistical computing, R Foundation for Statistical Computing,

\leavevmode\hypertarget{ref-robel2014}{}%
Robel, J., and Graumann, A., 2014, April. NOAA KLM Users Guide.

\leavevmode\hypertarget{ref-roberts2020}{}%
Roberts, B., Gallo, B.T., Jirak, I.L., Clark, A.J., Dowell, D.C., Wang, X., and Wang, Y., 2020. What Does a Convection-Allowing Ensemble of Opportunity Buy Us in Forecasting Thunderstorms? Weather and Forecasting, 35, 6, 2293--2316.

\leavevmode\hypertarget{ref-roberts2008}{}%
Roberts, N., 2008. Assessing the spatial and temporal variation in the skill of precipitation forecasts from an NWP model. Meteorological Applications, 15, 1, 163--169.

\leavevmode\hypertarget{ref-ruiz2010}{}%
Ruiz, J.J., Saulo, C., and Nogués-Paegle, J., 2010. WRF Model Sensitivity to Choice of Parameterization over South America: Validation against Surface Variables. Monthly Weather Review, 138, 8, 3342--3355.

\leavevmode\hypertarget{ref-saucedo2015}{}%
Saucedo, M.A., 2015. Estudio de los efectos de diferentes fuentes de error sobre la calidad de los análisis generados por un sistema de asimilación por filtros de Kalman.

\leavevmode\hypertarget{ref-sawada2019}{}%
Sawada, M., Ma, Z., Mehra, A., Tallapragada, V., Oyama, R., and Shimoji, K., 2019. Impacts of Assimilating High-Resolution Atmospheric Motion Vectors Derived from Himawari-8 on Tropical Cyclone Forecast in HWRF. Monthly Weather Review, 147, 10, 3721--3740.

\leavevmode\hypertarget{ref-shao2016}{}%
Shao, H., Derber, J., Huang, X.-Y., Hu, M., Newman, K., Stark, D., Lueken, M., Zhou, C., Nance, L., and Kuo, Y.-H. and others, 2016. Bridging Research to Operations Transitions: Status and Plans of Community GSI. Bulletin of the American Meteorological Society, 97, 8, 1427--1440.

\leavevmode\hypertarget{ref-singh2016}{}%
Singh, R., Ojha, S.P., Kishtawal, C.M., Pal, P.K., and Kiran Kumar, A.S., 2016. Impact of the assimilation of INSAT-3D radiances on short-range weather forecasts: Assimilation of INSAT-3D Radiances. Quarterly Journal of the Royal Meteorological Society, 142, 694, 120--131.

\leavevmode\hypertarget{ref-skamarock2008}{}%
Skamarock, W.C., Klemp, J.B., Dudhia, J., Gill, D.O., Barker, D.M., Duda, M.G., Huang, X.-Y., Wang, W., and Powers, J.G., 2008. A Description of the Advanced Research WRF Version 3 p. 125.

\leavevmode\hypertarget{ref-sobash2015}{}%
Sobash, R.A., and Stensrud, D.J., 2015. Assimilating Surface Mesonet Observations with the EnKF to Improve Ensemble Forecasts of Convection Initiation on 29 May 2012. Monthly Weather Review, 143, 9, 3700--3725.

\leavevmode\hypertarget{ref-stensrud2013}{}%
Stensrud, D.J., Wicker, L.J., Xue, M., Dawson, D.T., Yussouf, N., Wheatley, D.M., Thompson, T.E., Snook, N.A., Smith, T.M., and Schenkman, A.D. and others, 2013. Progress and challenges with Warn-on-Forecast. Atmospheric Research, 123, 2--16.

\leavevmode\hypertarget{ref-sun2014}{}%
Sun, J., Xue, M., Wilson, J.W., Zawadzki, I., Ballard, S.P., Onvlee-Hooimeyer, J., Joe, P., Barker, D.M., Li, P.-W., and Golding, B. and others, 2014. Use of NWP for Nowcasting Convective Precipitation: Recent Progress and Challenges. Bulletin of the American Meteorological Society, 95, 3, 409--426.

\leavevmode\hypertarget{ref-tong2020}{}%
Tong, M., Zhu, Y., Zhou, L., Liu, E., Chen, M., Liu, Q., and Lin, S.-J., 2020. Multiple Hydrometeors All-Sky Microwave Radiance Assimilation in FV3GFS. Monthly Weather Review, 148, 7, 2971--2995.

\leavevmode\hypertarget{ref-toshioinouye2017}{}%
Toshio Inouye, R., Calvetti, L., Gonçalves, J., Maske, B., Neundorf, R., Beneti, C., Diniz, F., Vendrasco, E., Herdies, D., and gustavo de goncalves, luis, 2017. Impact of radar data assimilation on a severe storm study in brazil. In. Presented at the 97th American Meteorological Meeting Annual Meeting.

\leavevmode\hypertarget{ref-vera1992}{}%
Vera, C.S., 1992. Un sistema de asimilación de datos para la región extratropical de Sudamérica.

\leavevmode\hypertarget{ref-wang2021}{}%
Wang, Z.Q., and Randriamampianina, R., 2021. The Impact of Assimilating Satellite Radiance Observations in the Copernicus European Regional Reanalysis (CERRA). Remote Sensing, 13, 3, 3, 426.

\leavevmode\hypertarget{ref-weng2013}{}%
Weng, F., Zou, X., Sun, N., Yang, H., Tian, M., Blackwell, W.J., Wang, X., Lin, L., and Anderson, K., 2013. Calibration of Suomi national polar-orbiting partnership advanced technology microwave sounder. Journal of Geophysical Research: Atmospheres, 118, 19, 11, 187--11, 200.

\leavevmode\hypertarget{ref-weston2019}{}%
Weston, P., Geer, A., Bormann, N., and Bormann, N., 2019. Investigations into the assimilation of AMSU-A in the presence of cloud and precipitation.

\leavevmode\hypertarget{ref-wheatley2010}{}%
Wheatley, D.M., and Stensrud, D.J., 2010. The Impact of Assimilating Surface Pressure Observations on Severe Weather Events in a WRF Mesoscale Ensemble System. Monthly Weather Review, 138, 5, 1673--1694.

\leavevmode\hypertarget{ref-whitaker2002}{}%
Whitaker, J.S., and Hamill, T.M., 2002. Ensemble Data Assimilation without Perturbed Observations. Monthly Weather Review, 130, 7, 1913--1924.

\leavevmode\hypertarget{ref-whitaker2012}{}%
---------, 2012. Evaluating Methods to Account for System Errors in Ensemble Data Assimilation. Monthly Weather Review, 140, 9, 3078--3089.

\leavevmode\hypertarget{ref-whitaker2008}{}%
Whitaker, J.S., Hamill, T.M., Wei, X., Song, Y., and Toth, Z., 2008. Ensemble Data Assimilation with the NCEP Global Forecast System. Monthly Weather Review, 136, 2, 463--482.

\leavevmode\hypertarget{ref-wickham2009}{}%
Wickham, H., 2009. Ggplot2: Elegant Graphics for Data Analysis, Springer-Verlag,

\leavevmode\hypertarget{ref-wilks2011}{}%
Wilks, D.S., 2011. Statistical Methods in the Atmospheric Sciences, 3rd ed., 676 pags Vol. 100.

\leavevmode\hypertarget{ref-wu2014}{}%
Wu, T.-C., Liu, H., Majumdar, S.J., Velden, C.S., and Anderson, J.L., 2014. Influence of Assimilating Satellite-Derived Atmospheric Motion Vector Observations on Numerical Analyses and Forecasts of Tropical Cyclone Track and Intensity. Monthly Weather Review, 142, 1, 49--71.

\leavevmode\hypertarget{ref-wu2002}{}%
Wu, W.-S., Purser, R.J., and Parrish, D.F., 2002. Three-Dimensional Variational Analysis with Spatially Inhomogeneous Covariances. Monthly Weather Review, 130, 12, 2905--2916.

\leavevmode\hypertarget{ref-xie2015}{}%
Xie, Y., 2015. Dynamic documents with R and knitr, Second., Chapman and Hall/CRC,

\leavevmode\hypertarget{ref-zhu2019}{}%
Zhu, K., Xue, M., Pan, Y., Hu, M., Benjamin, S.G., Weygandt, S.S., and Lin, H., 2019. The Impact of Satellite Radiance Data Assimilation within a Frequently Updated Regional Forecast System Using a GSI-based Ensemble Kalman Filter. Advances in Atmospheric Sciences, 36, 12, 1308--1326.

\leavevmode\hypertarget{ref-zhu2008}{}%
Zhu, Y., and Gelaro, R., 2008. Observation Sensitivity Calculations Using the Adjoint of the Gridpoint Statistical Interpolation (GSI) Analysis System. Monthly Weather Review, 136, 1, 335--351.

\leavevmode\hypertarget{ref-zhu2014}{}%
Zhu, Y., Derber, J., Collard, A., Dee, D., Treadon, R., Gayno, G., and Jung, J.A., 2014. Enhanced radiance bias correction in the National Centers for Environmental Prediction's Gridpoint Statistical Interpolation data assimilation system. Quarterly Journal of the Royal Meteorological Society, 140, 682, 1479--1492.

\leavevmode\hypertarget{ref-zhu2016}{}%
Zhu, Y., Liu, E., Mahajan, R., Thomas, C., Groff, D., Van Delst, P., Collard, A., Kleist, D., Treadon, R., and Derber, J.C., 2016. All-Sky Microwave Radiance Assimilation in NCEP's GSI Analysis System. Monthly Weather Review, 144, 12, 4709--4735.


% Index?

\end{document}
